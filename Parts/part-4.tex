%!TEX root = ../Masters.tex
\section{ГРАНИЧНІ ТРІЙКИ ДЛЯ КАНОНІЧНИХ СИСТЕМ}

\subsection{Граничні трійки для канонічних систем в регулярному випадку}

Нехай гамільтоніан $H$ задовольняє умові 
\begin{equation}\label{eq-4.1}
	H \in L^1(a,b).
\end{equation}
Тоді за Теоремою~\ref{th-relation-1} для будь-якої пари $\{f,g\}\in\calT$ і представника $f_0\in AC$ існують границі
\begin{equation}\label{eq-4.2}
	f_0(a+) = \lim_{x\downarrow a} f_0(x), \quad f_0(b-) = \lim_{x\uparrow a} f_0(x)
\end{equation}

\begin{theorem}\label{th-4.1}
	Нехай виконано умову \eqref{eq-4.1} і відображення $\Gamma_0, \Gamma_1:\calT\to\bbC^2$ задано рівностями
	\begin{equation} \label{eq-4.3}
		\Gamma_0f = 
		\begin{pmatrix}
			f_{01}(a) \\ f_{01}(b)
		\end{pmatrix},\quad
		\Gamma_1f = 
		\begin{pmatrix}
			-f_{02}(a) \\ f_{02}(b)
		\end{pmatrix}\quad
		(f,g)\in\calT,\ f_0\in f.
	\end{equation}
	Тоді сукупність $(\bbC^2,\Gamma_0,\Gamma_1)$ утворює граничну трійку для $\calT$.
\end{theorem}
\begin{proof}
	В силу Теореми~\ref{Th-CS-1} існують функції $\wt{u},\ \wt{v} \in\dom{\calT}$ такі, що 
	\begin{align}\label{eq-4.4}
	 	\wt{u}_{01}(a)&=1 & \wt{v}_{01}(a)&=0 \\
	 	\wt{u}_{02}(a)&=0 & \wt{v}_{02}(a)&=1.
	 \end{align}
	В силу Леми~\ref{lemma-7-relations} функції $\wt{u},\ \wt{v}$ можна змінити на інтервалі $(c,d)$ так, що $u_0$ і $v_0$ перетворюються на нуль на інтервалі $(c,d)$.

	Аналогічно, в силу Теореми~\ref{Th-CS-1} існують функції $h,\ k \in\dom{\calT}$ такі, що
	\begin{align}\label{eq-4.5}
	 	h_{01}(b)&=1 & k_{01}(a)&=0 \\
	 	h_{02}(b)&=0 & k_{02}(a)&=1.
	 \end{align}
	Користуючись Лемою~\ref{lemma-7-relations}, змінимо функції $h_0,\ k_0$ біля точки $a$ так, що $h_0$ і $k_0$ перетворюються на нуль в околі точки $a$. З \eqref{eq-4.4} і \eqref{eq-4.5} випливає, що відображення
	$
		\Gamma= 
		\begin{pmatrix}
			\Gamma_0 \\ \Gamma_1
		\end{pmatrix}
		:\calT\to\bbC^2\times\bbC^2
	$
	є сюр'єктивним.

	Тотожність \eqref{eq-CS-Green} випливає з \eqref{eq-relation-3}.
\end{proof}

У подальшому ми вводимо до розгляду матрицю $W(x,\lambda) = T(x,\lambda)^{T}$ і її блочной вигляд
\begin{equation}
 	W(x,\lambda)= 
 	\begin{pmatrix}
 		w_{11}(x,\lambda) & w_{12}(x,\lambda)\\
 		w_{21}(x,\lambda) & w_{22}(x,\lambda)
 	\end{pmatrix}
 \end{equation} 

\begin{theorem}\label{th-4.2}
	Функція Вейля, що відповідає граничній трійці \eqref{eq-4.3} має вигляд
	\begin{equation}\label{eq-4.7}
		M(b,\lambda)=
		\begin{pmatrix}
			-w_{11}(b,\lambda)w_{12}(b,\lambda)^{-1} & w_{12}(b,\lambda)^{-1} \\
			w_{21}(b,\lambda) - w_{22}(b,\lambda)w_{12}(b,\lambda)^{-1}w_{11}(b,\lambda) & w_{22}(b,\lambda)w_{12}(b,\lambda)^{-1}
		\end{pmatrix},
	\end{equation}
	а відповідне $\gamma$-поле має вигляд
	\begin{equation}\label{eq-4.8}
		\gamma(\lambda) = W(\cdot,\lambda)
		\begin{pmatrix}
			1 & 0 \\
			-w_{11}(b,\lambda)w_{12}(b,\lambda)^{-1} & w_{12}(b,\lambda)^{-1}
		\end{pmatrix}
	\end{equation}
\end{theorem}
\begin{proof}
	Дефектний простір $\frN_\lambda(T_0)$ складається з функцій
	\begin{equation}\label{eq-4.9}
		f(\cdot,\lambda) = W(\cdot,\lambda)
		\begin{pmatrix}
			\alpha_1 \\ \alpha_2
		\end{pmatrix}, \quad \text{ де } \alpha_1,\alpha_2\in\bbC.
	\end{equation}
	Застосовуючи оператори $\Gamma_0$, $\Gamma_1$ до $f(\cdot,\lambda)$, отримаємо
	\begin{equation*}
		\Gamma_0\widehat{f}(\cdot,\lambda) = \Phi_0(\lambda)\begin{pmatrix}
			\alpha_1 \\ \alpha_2
		\end{pmatrix}, \quad
		\Phi_0(\lambda) =
		\begin{pmatrix}
			1 & 0 \\
			w_{11}(b,\lambda) & w_{12}(b,\lambda)
		\end{pmatrix},
	\end{equation*}
	\begin{equation*}
		\Gamma_1 f(\cdot,\lambda) = \Phi_1(\lambda)\begin{pmatrix}
			\alpha_1 \\ \alpha_2
		\end{pmatrix}, \quad
		\Phi_1(\lambda) =
		\begin{pmatrix}
			0 & -1 \\
			w_{21}(b,\lambda) & w_{22}(b,\lambda)
		\end{pmatrix}.
	\end{equation*}
	Звідси отримаємо
	\begin{equation*}
		\gamma(\lambda) = W(\lambda)\Phi_0(\lambda)^{-1} = W(\lambda)
		\begin{pmatrix}
			1 & 0 \\
			-w_{11}(b,\lambda)w_{12}(b,\lambda)^{-1} & w_{12}(b,\lambda)^{-1}
		\end{pmatrix},
	\end{equation*}
	\begin{equation*}
		M(b,\lambda) = \Phi_1(\lambda)\Phi_2(\lambda)^{-1},
	\end{equation*}
	що призводить до \eqref{eq-4.7}, \eqref{eq-4.8}.
\end{proof}

\subsection{Теорія Вейля для канонічних систем} 

В роботі \cite{Weyl2010} досліджувалась поведінка коефіцієнта Вейля для оператора Штурма-Ліувілля на відрізку $[0,b]$, якщо $b\to\infty$. Зокрема, це дозволило показати, що спектральна задача для оператора Штурма-Ліувілля на прямій завжди має розв'язки в просторі $L^2(0,\infty)$.

В цьому розділі буде проведено аналогічне дослідження для системи \eqref{eq-relation-1} на півосі.

Розглянемо лінійне відношення $A(b,h)$, що продовжується в $L^2_H(0,b)$ системою \eqref{Canonical_sys} і граничними умовами
\begin{equation}\label{eq-4.10}
	f_1(0) = f_2(0) = f_2(b) + hf_1(b) = 0.
\end{equation}
З формули
% TODO ???
випливає, що спряжене лінійне відношення $A(b,h)^*$ задається системою і граничною умовою
\begin{equation}\label{eq-4.11}
	f_2(b) + hf_1(b) = 0.
\end{equation}
Гранична трійка для $A(b,h)^*$ задається рівностями
\begin{equation}\label{eq-4.12}
	\Gamma^{b,h}_0 f = f_1(0),\quad \Gamma_1^{b,h} f = f_2(0).
\end{equation}
Дефектний підпростір $\frN_\lambda(A(b,h))$ складається з вектор-функцій, пропорційних
\begin{equation}\label{eq-4.13}
	\Psi(x,\lambda) = W^T(x,\lambda)
	\begin{pmatrix}
		1 \\ -m(\lambda,b,h)
	\end{pmatrix},
\end{equation}
де коефіцієнт $m(\lambda,b,h)$ знаходиться з умови $\Psi_2(b,\lambda) + h\Psi_1(b,\lambda) = 0$, тобто
\begin{equation*}
	w_{12}(b,\lambda) - w_{22}(b,\lambda)m(\lambda,b,h) + h\{w_{11}(b,\lambda) - w_{21}(b,\lambda)m(\lambda,b,h) \} = 0.
\end{equation*}
Звідси знаходимо
\begin{equation}\label{eq-4.14}
	m(\lambda,b,h) = \frac{w_{11}(b,\lambda)h + w_{12}(b,\lambda)}{w_{21}(b,\lambda)h + w_{22}(b,\lambda)}.
\end{equation}

\begin{theorem}\label{th-4.3}
	Нехай гранична трійка для $A(b,h)^*$ задана формулою \eqref{eq-4.12}. Тоді:
	\begin{enumerate}
		\item Відповідна функція Вейля співпадає з $m(\lambda,b,h)$, а $\gamma$-поле має вигляд
		\begin{equation}\label{eq-4.15}
			\gamma(\lambda,b,h) = W(\cdot,\lambda)
			\begin{pmatrix}
				1 \\ -m(\lambda,b,h)
			\end{pmatrix}.
		\end{equation}
		\item При фіксованому $\lambda\in\bbC_+$ множина значень $m(\lambda,b,h)$ заповнює коло $C_b(\lambda)$ в $\bbC$ з центром
		\begin{equation}\label{eq-4.16}
			\wt{m}_b(\lambda) = \frac{w_2(b,\lambda)^* Jw_1(b,\lambda)}{w_2(b,\lambda)^* Jw_2(b,\lambda)}
		\end{equation}
		і радіусом
		\begin{equation}\label{eq-4.17}
			r_b(\lambda) = \left(2\Im{\lambda}\int\limits_0^b w_2(x,\lambda)^* H(x) w_2(x,\lambda)\,dx \right)^{-1}.
		\end{equation}
		\item Круг $K_b(\lambda)$, обмежений колом $C_b(\lambda)$ характеризується нерівністю
		\begin{equation}\label{eq-4.18}
			\int\limits_0^b \gamma(x,\lambda,b,h)^* H(x) \gamma(x,\lambda,b,h) \le \frac{\Im{m(\lambda,b,h)}}{\Im{\lambda}}.
		\end{equation}
	\end{enumerate}
\end{theorem}
\begin{proof}
	(1) випливає з \eqref{eq-4.12} і \eqref{eq-4.13}, оскільки
	\begin{equation*}
		\Gamma_0\Psi(\cdot,\lambda) = 1, \quad \Gamma_1\Psi(\cdot,\lambda) = m(\lambda,b,h).
	\end{equation*}
	Зауважимо, що відповідне $\gamma$-поле співпадає з $\Psi(\cdot,\lambda)$, тобто в силу \eqref{eq-4.13} виконується \eqref{eq-4.15}.

	З рівності 
	%TODO ???
	отримуємо, що $m(\lambda,b,h)$ належить до кола
	\begin{equation}
		\int\limits_0^b 
		\begin{pmatrix}
			1 & -m(\lambda,b,h)^*
		\end{pmatrix}
		H(x) W(x,\lambda)
		\begin{pmatrix}
			1 \\ -m(\lambda,b,h)
		\end{pmatrix}
		\,dx = \frac{\Im{m(\lambda,b,h)}}{\Im{\lambda}}.
	\end{equation}
	Оскільки дробово-лінійне перетворення \eqref{eq-4.14} переводить $h=-\dfrac{w_{22}(b,\lambda)}{w_{21}(b,\lambda)}$ в $\infty$, то $h=-\dfrac{\overline{w_{22}(b,\lambda)}}{\overline{w_{21}(b,\lambda)}}$ переходить у центр кола $C_b(\lambda)$
	\begin{equation*}
		\wt{m}_b(\lambda) = \frac{w_{12}(b,\lambda)w_{21}(b,\lambda)^* - w_{22}(b,\lambda)^*w_{11}(b,\lambda)}{w_{21}(b,\lambda)^*w_{22}(b,\lambda) - w_{22}(b,\lambda)^*w_{21}(b,\lambda)},
	\end{equation*}
	що співпадає з \eqref{eq-4.16}.

	Радіус кола $C_b(\lambda)$ може бути знайдений з рівності
	\begin{equation}
		r_b(\lambda) = \left| \frac{w_2(b,\lambda)^*Jw_1(b,\lambda)}{w_2(b,\lambda)^*Jw_2(b,\lambda)} - \frac{w_{21}(b,\lambda)}{w_{22}(b,\lambda)} \right|.
	\end{equation}
	Оскільки
	\begin{multline*}
		\left( w_{21}(b,\lambda)^*w_{12}(b,\lambda) - w_{22}(b,\lambda)^*w_{11}(b,\lambda) \right)w_{22}(b,\lambda) \\
		-\left( w_{21}(b,\lambda)^*w_{22}(b,\lambda) - w_{22}(b,\lambda)^*w_{21}(b,\lambda) \right) w_{21}(b,\lambda) \\
		= -w_{22}(b,\lambda)^* \left( w_{11}(b,\lambda)w_{22}(b,\lambda) - w_{21}(b,\lambda)w_{12}(b,\lambda) \right) = -w_{22}(b,\lambda)^*,
	\end{multline*}
	то
	\begin{equation}\label{eq-4.19}
		r_b(\lambda)^{-1} = \left| w_2(b,\lambda)^*Jw_2(b,\lambda) \right|.
	\end{equation}
	З тотожності \eqref{eq-relation-3} отримаємо
	\begin{equation}\label{eq-4.20}
		w_2(b,\lambda)^*Jw_2(b,\lambda) = w_2(b,\lambda)^*Jw_2(b,\lambda) - w_2(0,\lambda)^*Jw_2(0,\lambda) = (\bar{\lambda} - \lambda)\int\limits_0^b w_2(x,\lambda)^*Jw_2(x,\lambda)\,dx.
	\end{equation}
	Рівність \eqref{eq-4.17} випливає з \eqref{eq-4.19} і \eqref{eq-4.20}.
\end{proof}

\begin{corollary}
	Круги $K_b(\lambda)$ вкладені одне в одне $K_{b_2}(\lambda)\subset K_{b_1}(\lambda)$ при $b_1<b_2$.
	При цьому можливо наступне:
	\begin{enumerate}
		\item або $\bigcap\limits_{b>0} K_b(\lambda)$ містить одну точку і тоді існує єдиний розв'язок $\Psi(\cdot,\lambda)$ системи \eqref{eq-relation-1}, який належить $L_H^2(0,\infty)$
		\item або $\bigcap\limits_{b>0} K_b(\lambda)$ є граничний круг $K_\infty(\lambda)$ і тоді кожний розв'язок системи \eqref{eq-relation-1} належить до $l^2_H(0,\infty)$.
	\end{enumerate}
\end{corollary}

Зрозуміло, що в першому випадку маємо $\dim \frN_\lambda = 1$, а в другому --- $\frN_\lambda = 2$.

З загальної теорії розширень симетричних операторів випливає, що для всіх $\lambda\in\bbC_+$ одночасно має місце випадок граничної точки, якщо це трапляється для однієї точки $\lambda_0\in\bbC_+$.

\begin{definition}\label{def-4.5}
	Будемо говорити, що для системи \eqref{eq-relation-1} має місце
	\begin{enumerate}
		\item випадок граничної точки, якщо $K_\infty(\lambda)$ складається з однієї точки для всіх $\lambda\in\bbC_+$;
		\item випадок граничного круга, якщо $K_\infty(\lambda)$ --- це круг для всіх $\lambda\in\bbC$.
	\end{enumerate}
\end{definition}
В першому випадку $n_\pm(T_0) = 1$, а в другому $n_\pm(T_0) = 2$.

\begin{corollary} [Аналог теореми Вейля]
	Нехай $H\in L_{loc}^1[0,\infty)$. Тоді існує принаймні один розв'язок системи
	\begin{equation*}
		Jy'=\lambda Hy,
	\end{equation*}
	який належить до $L^2_H(0,\infty)$.
\end{corollary}

\subsection{Граничні трійки для канонічної системи у випадку граничної точки в $\infty$}

\begin{theorem}
	Система \eqref{eq-relation-1} має випадок граничної точки в $b$ тоді і тільки тоді, коли $\trace H \notin L^1(c,b)$ для $c<b$.
\end{theorem}
\begin{proof}
	Якщо $H\in L^1[c,b)$ для деякої точки $c\in [0,b)$, то система \eqref{eq-relation-1} є регулярною в точці $b$, тобто оператор $T_0$ має індекси дефекту $(2,2)$. Це означає, що для системи \eqref{eq-relation-1} має місце випадок граничної точки в $b$.

	Навпаки, припустимо, що для системи \eqref{eq-relation-1} має місце випадок граничного круга в $b$. Тоді всі точки з $\bbC$ є точками регулярного типу і індекси дефектного підпростору оператора $T_0$ дорівнюють $(2,2)$. Для точки $\lambda = 0$ маємо 2 лінійно незалежних розв'язки системи \eqref{eq-relation-1}
	\begin{equation*}
		Jy'=0
	\end{equation*}
	$y_1\equiv e_1$ і $y_2\equiv e_2$, які належать до $L^2_H(c,b)$. Тоді
	\begin{equation*}
		\int\limits_0^b \trace H(x)\,dx = 
		\int\limits_0^b (H(x)e_1,e_1) + (H(x)e_2,e_2)\,dx = 
		||y_1||^2_{L^2_H(c,b)} + ||y_2||^2_{L^2_H(c,b)} < \infty,
	\end{equation*}
	тобто $\trace{H} \in L^1(c,b)$.
\end{proof}

\begin{definition}
	Якщо для системи \eqref{eq-relation-1} має місце випадок граничної точки в $b$, то з Означення~\ref{def-4.5} випливає, що існує єдине значення $m_\infty(\lambda)$ для кожного $\lambda\in\bbC_+$ таке, що $\Psi(\cdot,\lambda) = W^T(\cdot,\lambda)
\begin{pmatrix}
	1 \\ -m_\infty(\lambda)
\end{pmatrix}
	$ належить до $L^2_H(0,b)$. Коефіцієнт $m_\infty(\lambda)$ називають коефіцієнтом Вейля-Тітчмарша системи \eqref{eq-relation-1}, а відповідний розв'язок системи $\phi(\lambda)$ називають розв'язком Вейля.

	Функція Вейля $m_\infty(\lambda)$ знаходиться за формулою
	\begin{equation}\label{eq-4.23A}
		m_\infty(z) = \lim_{x\to\infty} \frac{w_{11}(x,z)h + w_{12}(x,z)}{w_{21}(x,z)h+w_{22}(x,z)},
	\end{equation}
	причому границя в \eqref{eq-4.23A} не залежить від вибору $h\in\bbR$.
\end{definition}

\begin{lemma}\label{lemma-4.9}
	Якщо $H\notin L^1(0,b)$, то для всіх $(f,g)\in \calT$ має місце 
	\begin{equation}\label{eq-4.21}
		\lim_{x\to b} f_0^*Jf_0 = 0
	\end{equation}
	і область визначення мінімального лінійного відношення $\calT_0$ задається рівністю
	\begin{equation}\label{eq-4.22}
		\dom{\calT_0} = \{ f_0\in \dom{\calT}: f_{01}(0) = f_{02}(0)=0 \}
	\end{equation}
\end{lemma}
\begin{proof}
	Нехай $u,\ v$ --- функції, визначені в \eqref{eq-4.4}, які є фінітними в околі точки $b$. Оскільки $n_\pm(T_0)=1$, то
	\begin{equation*}
		\dim (\dom{\calT}/\dom{\calT_0}) = n_+(\calT_0) + n_-(\calT_0) = 2 
	\end{equation*}
	і тому кожна функція з $\dom{\calT}$ допускає представлення
	\begin{equation}\label{eq-4.23}
		f_0=h_0+c_1u+c_2v.
	\end{equation}
	Оскільки $u,\ v$ є фінітними в точці $b$, то
	\begin{equation}
		\lim_{x\to b} f^*Ju = \lim_{x\to b} f^*Jv, \quad \forall f\in\dom{T}.
	\end{equation}
	Далі з Теореми~\ref{th-relation-1} випливає, що для $(f,g)\in T$, $(h,k)\in T_0$
	\begin{equation*}
		0 = \left< g,h \right> - \left< f,k \right> = \lim f_0^*Jh_0.
	\end{equation*}
	Тому має місце \eqref{eq-4.21}. Рівність \eqref{eq-4.22} випливає з \eqref{eq-4.23} і \eqref{eq-relation-3}.
\end{proof}

\begin{theorem}\label{th-4.10}
	Нехай $H\notin L^1(0,b)$. Тоді для системи \eqref{eq-relation-1} має місце випадок граничної точки в $b$. При цьому:
	\begin{enumerate}
		\item Сукупність $\{\bbC,\Gamma_0,\Gamma_1\}$, в якій
		\begin{equation}\label{eq-4.24}
			\Gamma_0\widehat{f}=f_{01}(0),\quad \Gamma_1\widehat{f} = -f_{02}(0),
		\end{equation}
		утворює граничну трійку для $\calT$.
		\item Відповідна функція Вейля співпадає з коефіцієнтом Вейля-Тітчмарша $m_\infty(\lambda)$.
	\end{enumerate}
\end{theorem}
\begin{proof}
	Я було показано в Лемі~\ref{lemma-4.9}
	\begin{equation*}
		\lim_{x\to\infty} f_0^*Jh_0 = 0 \quad \forall (f,g), (h,k)\in T.
	\end{equation*}
	Тому рівність \eqref{eq-relation-3} приймає вигляд
	\begin{equation}
		(h,g)_{L^2_H} - (k,f)_{L^2_H} =\Bigl. -f^*_0Jh_0\, \Bigr|_0 = f^*_{01}(0)h_{02}(0) - f^*_{02}(0)h_{01}(0).
	\end{equation}
	Це доводить формулу \eqref{eq-CS-Green}. Сюр'єктивність відображення випливає з представлення \eqref{eq-4.23}.

	Твердження (2) випливає з рівностей
	\begin{equation*}
		\Gamma_0\Psi = \Gamma_0
		\begin{pmatrix}
			w_{11}-w_{21} & m_\infty(\lambda) \\
			w_{21}-w_{22} & m_\infty(\lambda)
		\end{pmatrix}
		=w_{11}(0,\lambda) - w_{21}(0,\lambda)m_\infty(\lambda) = 1,
	\end{equation*}
	\begin{equation*}
		\Gamma_1\Psi = -(w_{21}(0,\lambda)-w_{22}(0,\lambda)m_\infty(\lambda)) = m_\infty(\lambda).
	\end{equation*}
\end{proof}

\begin{lemma}\normalfont{\cite{Win1995}}\label{lemma-4.11}
	Нехай дані дві канонічні системи з гамільтоніанами $H(x)$ і $\wt{H}(x)=H(l+x)$ для деякого $l>0$ і $x\in [0,\infty)$. Якщо $W$ --- фундаментальна матриця системи, що відповідає $H$, а $m$ і $\wt{m}$ --- коефіцієнти Вейля, відповідні до $H(x)$ і $\wt{H}$. Тоді
	\begin{equation}\label{eq-4.35}
		m(z) = \frac{w_{11}(l,z) \wt{m}(z) + w_{12}(l,z)}{w_{21}(l,z) \wt{m}(z) + w_{22}(l,z)}.
	\end{equation}

	Зокрема, якщо $(0,l)$ --- сингулярний інтервал типу $\phi$ для $H$, тоді
	\begin{equation}\label{eq-4.36}
		Q(z) = \ctg{(\phi)} + \dfrac{1}{-zl\sin^2{\phi} + \dfrac{1}{\wt{Q}(z) - \ctg{(\phi)}}}, \text{ якщо } \phi\ne0
	\end{equation}
	і
	\begin{equation}\label{eq-4.37}
		Q(z) = lz + \wt{Q}(z), \text{ якщо } \phi = 0.
	\end{equation}
	
\end{lemma}
\begin{proof}
	Матричні функції $W(l+x,z)$ і $W(l,z)\wt{W}(x,z)$ є фундаментальними матрицями для канонічної системи \eqref{Canonical_sys}, отже $W(l+x,z) = W(l,z)\wt{W}(x,z)$ за Теоремою~\ref{th-CS-2}. Тоді з рівності
	\begin{equation*}
		m(z) = \lim_{x\to\infty} \dfrac{w_{11}(l+x,z)}{w_{12}(l+x,z)} = \lim_{x\to\infty} \dfrac{w_{11}(l,z)\wt{w}_{11}(x,z) + w_{12}(l,z)\wt{w}_{21}(x,z)}{w_{21}(l,z)\wt{w}_{11}(x,z) + w_{22}(l,z)\wt{w}_{21}(x,z)}
	\end{equation*}
	отримаємо \eqref{eq-4.35}. Якщо інтервал $(0,l)$ --- сингулярний інтервал типу $\phi$ для $H$, то
	\begin{equation*}
		W(l,z) = I - zlHJ =
		\begin{pmatrix}
			1-zl\sin{\phi}\cos{\phi} & zl\cos^2{\phi}\\
			-zl\sin^2{\phi} & 1+zl\sin{\phi}\cos{\phi}
		\end{pmatrix}.
	\end{equation*}
	Підставивши останнє у \eqref{eq-4.35}, отримаємо \eqref{eq-4.36} і \eqref{eq-4.37} в якості першого кроку неперервного розвинення $m(z)$ у неперервний дріб.
\end{proof}
