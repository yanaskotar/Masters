%!TeX root = ../Masters-talk.tex

% \section{Канонічні системи}

\begin{frame}{Канонічні системи}
	\begin{description}
		\item [Канонічною системою] називається диференціальне рівняння у формі
		\begin{equation} \label{Canonical_sys}\tag{CS}
			Ju'(x) = -zH(x)u(x), \quad 
			J=
			\begin{pmatrix}
				0 & -1\\
				1 & 0
			\end{pmatrix},
		\end{equation}
		$x\in (a,b),\  -\infty\le a<b \le +\infty$
		
		з гамільтоніаном $H(x) = \begin{pmatrix}
				h_1(x) & h_3(x)\\
				h_3(x) & h_2(x)
			\end{pmatrix}$:
		\begin{enumerate}
			\item $H(x) \in \bbR^{2\times 2}$
			\item $H \in L^1_{loc}(a,b)$
			\item $H(x)\ge 0 \text{ майже для всіх } x\in(a,b)$
			\item $H(x)\ne 0 \text{ майже для всіх } x\in(a,b)$
		\end{enumerate}
	\end{description}
\end{frame}

\begin{frame}{Канонічні системи}
		\begin{description}
			\item [Розв'язком канонічної системи] буде називатися функція $u:(a,b)\to\bbC^2$ , якщо $u\in AC(a,b)$ і \eqref{Canonical_sys} справджується майже всюди.
			\item [Матрицею переходу] $T$ називається матричний розв'язок, що приймає значення в $\bbC^{2\times 2}$ однорідного рівняння
			\begin{equation}\label{eq-Canonycal_sys_matrix}
				JT'=-zHT
			\end{equation}
			з початковою умовою $T(c) = I$.
			\item [Фундаментальна матриця] визначається як $W(x,\lambda) = T(x,\lambda)^{T}$ і має блочной вигляд
			\begin{equation} \tag{FM}
			 	W(x,\lambda)= 
			 	\begin{bmatrix}
			 		w_{11}(x,\lambda) & w_{12}(x,\lambda)\\
			 		w_{21}(x,\lambda) & w_{22}(x,\lambda)
			 	\end{bmatrix}
			 \end{equation}
		\end{description}
\end{frame}

\begin{frame}{Канонічні системи: Простір $L^2_H$}
	Нехай
\begin{equation*}
 	\calL = \left\{f:(a,b)\to \bbC^2: f \text{ є вимірною }, \int\limits_a^b f^*(x)H(x)f(x)\,dx <\infty \right\},
\end{equation*} 
тоді
\begin{equation}
 	||f|| = \left(\int\limits_a^b f^*Hf\,dx\right)^{1/2}, \quad f\in\calL,
\end{equation}
і $L_H^2(a,b)$ визначається як $\calL/\calN$, де $\calN=\{f\in\calL: ||f||=0\}$ зі скалярним добутком
\begin{equation}\tag{SP}
 	(f,g)_{L^2_H}=\int\limits_a^b g^*(x)H(x)f(x)\,dx
 \end{equation} 
\end{frame}