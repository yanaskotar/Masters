\documentclass[a4paper, 12pt]{article}
\usepackage{mystyle}
\usepackage[T2A]{fontenc}
\usepackage[utf8]{inputenc}
\usepackage[russian, french, german, english, ukrainian]{babel}
\usepackage{csquotes}
\usepackage{setspace}
% полуторный интервал
\onehalfspacing 


% Расширенные наборы математических символов.
\usepackage{amssymb}
\usepackage{amsfonts}
\usepackage{latexsym}
\usepackage{mathtext}
\usepackage{amsthm}
\usepackage{amsmath}
\usepackage{lastpage}

% \usepackage{showkeys}

\usepackage{enumerate}
\usepackage{textcomp}
\usepackage{tabularx}

\usepackage	[
			style		= gost-numeric,
			maxbibnames = 3,
			sorting		= nyt,
			doi			= true,
			url			= false,
			isbn 		= true,
			eprint		= true,
			language	= autobib,
			autolang	= other
			]{biblatex}
\addbibresource{ref.bib}
\newcounter{mycitecount}
\AtEveryBibitem{\stepcounter{mycitecount}}


\usepackage[mycitecount]{totalcount}


\theoremstyle	{plain}
\newtheorem		{theorem}					{\normalfont \textbf{Теорема}}	[section]
\newtheorem		{lemma}			[theorem]	{Лема}
\newtheorem		{corollary}		[theorem]	{Наслідок}
\newtheorem		{proposition}	[theorem]	{Пропозиція}
\newtheorem		{assertion}		[theorem]	{Твердження}
\newtheorem		{assumption}	[theorem]	{Умова}
\theoremstyle	{definition}
\newtheorem		{definition}	[theorem]	{Означення}
\theoremstyle	{remark}
\newtheorem		{remark}		[theorem]	{\normalfont \textit{Зауваження}}
\newtheorem*	{remark*}					{\normalfont \textit{Зауваження}}
\newtheorem 	{example}		[theorem]	{\normalfont \textit{Приклад}}
\newtheorem* 	{example*}					{\normalfont \textit{Приклад}}


% FRAKTUR 
% Готическое начертание
% \frq - стандартная команда для символа '>', ее нужно переопределить
% \frq is a standart command fot the symbol '>', it is necessary to redefine it.

\newcommand{\frA}{\mathfrak{A}}   \newcommand{\frB}{\mathfrak{B}}   \newcommand{\frC}{\mathfrak{C}}
\newcommand{\frD}{\mathfrak{D}}   \newcommand{\frE}{\mathfrak{E}}   \newcommand{\frF}{\mathfrak{F}}
\newcommand{\frG}{\mathfrak{G}}   \newcommand{\frH}{\mathfrak{H}}   \newcommand{\frI}{\mathfrak{I}}
\newcommand{\frJ}{\mathfrak{J}}   \newcommand{\frK}{\mathfrak{K}}   \newcommand{\frL}{\mathfrak{L}}
\newcommand{\frM}{\mathfrak{M}}   \newcommand{\frN}{\mathfrak{N}}   \newcommand{\frO}{\mathfrak{O}}
\newcommand{\frP}{\mathfrak{P}}   \newcommand{\frQ}{\mathfrak{Q}}   \newcommand{\frR}{\mathfrak{R}}
\newcommand{\frS}{\mathfrak{S}}   \newcommand{\frT}{\mathfrak{T}}   \newcommand{\frU}{\mathfrak{U}}
\newcommand{\frV}{\mathfrak{V}}   \newcommand{\frW}{\mathfrak{W}}   \newcommand{\frX}{\mathfrak{X}}
\newcommand{\frY}{\mathfrak{Y}}   \newcommand{\frZ}{\mathfrak{Z}}

\newcommand{\fra}{\mathfrak{a}}   \newcommand{\frb}{\mathfrak{b}}   \newcommand{\frc}{\mathfrak{c}}
\newcommand{\frd}{\mathfrak{d}}   \newcommand{\fre}{\mathfrak{e}}   \newcommand{\frf}{\mathfrak{f}}
\newcommand{\frg}{\mathfrak{g}}   \newcommand{\frh}{\mathfrak{h}}   \newcommand{\fri}{\mathfrak{i}}
\newcommand{\frj}{\mathfrak{j}}   \newcommand{\frk}{\mathfrak{k}}   \newcommand{\frl}{\mathfrak{l}}
\newcommand{\frm}{\mathfrak{m}}   \newcommand{\frn}{\mathfrak{n}}   \newcommand{\fro}{\mathfrak{o}}
\newcommand{\frp}{\mathfrak{p}} \renewcommand{\frq}{\mathfrak{q}}	\newcommand{\frr}{\mathfrak{r}}
\newcommand{\frs}{\mathfrak{s}}   \newcommand{\frt}{\mathfrak{t}}   \newcommand{\fru}{\mathfrak{u}}
\newcommand{\frv}{\mathfrak{v}}   \newcommand{\frw}{\mathfrak{w}}   \newcommand{\frx}{\mathfrak{x}}
\newcommand{\fry}{\mathfrak{y}}   \newcommand{\frz}{\mathfrak{z}}


% CALIGRA 
% Калиграфическое начертание

\newcommand{\calA}{\mathcal{A}}   \newcommand{\calB}{\mathcal{B}}   \newcommand{\calC}{\mathcal{C}}
\newcommand{\calD}{\mathcal{D}}   \newcommand{\calE}{\mathcal{E}}   \newcommand{\calF}{\mathcal{F}}
\newcommand{\calG}{\mathcal{G}}   \newcommand{\calH}{\mathcal{H}}   \newcommand{\calI}{\mathcal{I}}
\newcommand{\calJ}{\mathcal{J}}   \newcommand{\calK}{\mathcal{K}}   \newcommand{\calL}{\mathcal{L}}
\newcommand{\calM}{\mathcal{M}}   \newcommand{\calN}{\mathcal{N}}   \newcommand{\calO}{\mathcal{O}}
\newcommand{\calP}{\mathcal{P}}   \newcommand{\calQ}{\mathcal{Q}}   \newcommand{\calR}{\mathcal{R}}
\newcommand{\calS}{\mathcal{S}}   \newcommand{\calT}{\mathcal{T}}   \newcommand{\calU}{\mathcal{U}}
\newcommand{\calV}{\mathcal{V}}   \newcommand{\calW}{\mathcal{W}}   \newcommand{\calX}{\mathcal{X}}
\newcommand{\calY}{\mathcal{Y}}   \newcommand{\calZ}{\mathcal{Z}}


% Множини чисел

\newcommand{\bbN}{\mathbb{N}}	\newcommand{\bbZ}{\mathbb{Z}}
\newcommand{\bbQ}{\mathbb{Q}}	\newcommand{\bbR}{\mathbb{R}}
\newcommand{\bbC}{\mathbb{C}}	\newcommand{\bbD}{\mathbb{D}}


% Real and Imagine parts of a complex value
% Действительная и мнимая части комплексной величины

\renewcommand{\Re}{\operatorname{Re}}
\renewcommand{\Im}{\operatorname{Im}}


% Комплексное сопряжение
\newcommand*\conj[1]{\overline{#1}}


% Range of operator
\DeclareMathOperator{\ran}{ran}
% Domain of operator
\DeclareMathOperator{\dom}{dom}
% Multivalued part of operator (linear relation)
\DeclareMathOperator{\mul}{mul}
% Extension class
\DeclareMathOperator{\Ext}{Ext}

\DeclareMathOperator{\spn}{span}

\newcommand{\wt}{\widetilde}

\DeclareMathOperator{\sign}{sign}

\DeclareMathOperator{\Lip}{Lip}

\DeclareMathOperator{\grad}{grad}

\DeclareMathOperator{\diam}{diam}

\DeclareMathOperator{\defect}{def}

\DeclareMathOperator{\rot}{rot}

\DeclareMathOperator{\dist}{dist}

\DeclareMathOperator{\trace}{trace}

\DeclareMathOperator{\gr}{gr}

% \DeclareMathOperator{\ctg}{ctg}



\usepackage	[
	left	= 3cm,
	right	= 1cm,
	top		= 2cm,
	bottom	= 2cm
	]{geometry}


\begin{document}
%!TEX root = ../Masters.tex
\begin{titlepage}
\newpage

\begin{center}
МІНІСТЕРСТВО ОСВІТИ І НАУКИ УКРАЇНИ \\
ДОНЕЦЬКИЙ НАЦІОНАЛЬНИЙ УНІВЕРСИТЕТ ІМЕНІ ВАСИЛЯ СТУСА \\
\end{center}

\vspace{3em}

\begin{center}
	\Large СКОТАР ЯНА ЯНІВНА
\end{center}

\vspace{1em}

\flushright{
\begin{minipage}[c]{0.5\linewidth}
	Допускається до захисту:\\
	заступник завідувача кафедри\\
	прикладної математики, к.ф.-м.н.\\
	\underline{\hspace{5cm}}О.\,Д.~Трофименко\\
	<<\underline{\hspace{1cm}}>>\underline{\hspace{4cm}} 20\underline{\hspace{1cm}} р.
\end{minipage}
	
}

\vspace{2em}

\begin{center}
	\Large{\textsc{Канонічні системи і їх функції Тітчмарша-Вейля}}\\
	
	\vspace{1em}

	Спеціальність 111 Математика\\
	Магістерська робота
\end{center}

\vspace{4em}

\begin{flushleft}
	Науковий керівник:\\
	Деркач В.\,О., завідувач кафедри\\
	прикладної математики,\\
	д.ф.-м.н., професор
\end{flushleft}

\vspace{4em}

\begin{flushright}
	\begin{minipage}[c]{0.5\linewidth}
		Оцінка:\underline{\hspace{1cm}}/\underline{\hspace{1cm}}/\underline{\hspace{4cm}}

		\quad{} \quad{} \quad{}  {\tiny(бали/за шкалою CKTS/за національною шкалою)}

		Голова ЕК: \underline{\hspace{5.6cm}}
	\end{minipage}
\end{flushright}

\vspace{\fill}

\begin{center}
	Вінниця 2019
\end{center}

\end{titlepage}
\setlength{\parskip}{0.4cm}
%!TEX root = ../Masters.tex
\thispagestyle{empty}
\renewcommand{\abstractname}{АНОТАЦІЯ}
\begin{abstract}
	Скотар Я.\,Я. Канонічні системи і їх функції Тітчмарша-Вейля. Спеціальність 111 Математика, освітня програма Математика. Донецький національний університет імені Василя Стуса, 2019 --- \pageref{LastPage}\,c.

	\vspace{2.5em}

	Ключові слова: Симетричні оператори, лінійні відношення, граничні трійки, канонічні системи, функції Вейля.

	Бібліограф.: \totalmycitecounts~найм.

	\vspace{1em}

	Skotar Ya. Canonical systems and their Titchmarsh-Weyl functions. Specialty 111 Mathematics, Education program Mathematics.  Vasyl’ Stus Donetsk National University, Vinnytsia, 2019.

	\vspace{2.5em}

	Keywords: Symmetric operators, Linear relations, Boundary triples, Canonycal systems, Weyl functions.
	
	Bibliography: \totalmycitecounts~items.
\end{abstract}
% 
\setcounter{page}{3}
% %
\setlength{\parskip}{0cm}
\renewcommand{\contentsname}{ЗМІСТ}
\tableofcontents
% %
\setlength{\parskip}{0.4cm}
%!TEX root = ../Masters.tex

\section*{ВСТУП}
\addcontentsline{toc}{section}{ВСТУП}


Канонічною системою називається диференціальне рівняння у вигляді

\begin{equation*} 
	Ju'(x) = -zH(x)u(x), \quad
	J=
	\begin{pmatrix}
		0 & -1\\
		1 & 0
	\end{pmatrix}.
\end{equation*}

Ця система зі спектральним параметром $z$ розглядатиметься на відкритому, можливо нескінченному, інтервалі $x\in (a,b)$, $-\infty\le a<b \le +\infty$ в якій $H$ --- дійснозначна $2\times 2$ матриця-функція, яка належить $L^1_{loc}(a,b)$ і не дорівнює тотожньо нулю. Канонічні системи представляють великий математичний інтерес, оскільки вони в точному розумінні є найбільш загальним класом симетричних операторів другого порядку.\cite{Remling2018}

Теорія канонічних систем включає в собі всі види диференціальних операторів другого порядку такі як оператор Штурма-Ліувілля, струна Крейна-Феллера, різницеві оператори, пов'язані з матрицею Якобі, та інші. Основи спектральної теорії канонічних систем було закладено в роботах М.\,Г.~Крейна (див. також монографії \cite{KreinGohb,Atkinson}). Повний опис спектральних функцій канонічних систем другого порядку було отримано Л. де Бранжем \cite{deBranges}. Сучасна теорія канонічних систем представлена в монографіях \cite{ArovD12,Remling2018,Sakhnovich}.

Іншій важливий об'єкт, який розглядається в роботі --- це функції Вейля-Тітчмарша.  Для оператора Штурма-Ліувілля цю функцію було введено Германом Вейлем у звязку з класифікацією сингулярних точок оператора методом теорії вкладених кругів Вейля. У подальшому цю функцію було використано Тітчмаршем для обчислення спектральної функції оператора Штурма-Ліувілля. В роботах В. Деркача і М. Маламуда було введено абстрактний варіант функції Вейля-Тітчмарша і досліджено спектри довільних розширень симетричного оператора у просторі Гільберта.
Важливість функції Вейля для спектральної теорії канонічних систем  випливає з того, що їх інтегральні представлення дозволяють обчислити спектральні функції самоспряжених операторів, що відповідають канонічній системі.

В магістерській роботі застосовано теорію граничних трійок, що було розвинено в роботах \cite{Koch1975,Gorb1991} до канонічних систем. Зокрема, побудовано  теорію вкладених кругів Вейля для канонічних систем, знайдено формули для граничних трійок для канонічних систем як в регулярному, так і в сингулярному випадку  граничного кола у нескінченності. Наведено формулу про факторизацію фундаментальних матриц для зчеплення двох канонічних систем. Це дозволило обчислювати функції Вейля для граничних трійок і функцій Вейля багатьох канонічних систем у явному вигляді. В роботі розглянуто три приклади канонічних систем, для яких знайдено граничні трійки і їх функції Вейля.

%!TEX root = ../Masters.tex

\section{ТЕОРІЯ РОЗШИРЕНЬ СИМЕТРИЧНИХ ОПЕРАТОРІВ. ФУНКЦІЇ КЛАСІВ $R$ ТА $S$} % (fold)
%\label{sec:перший_розділ}

\subsection{Розширення симетричних операторів} % (fold)
%\label{sub:розширення_симетричних_операторів}

В цьому розділі наведено огляд класичної теорії розширень симетричних операторів. Ця теорія була побудована в роботах Дж.~фон~Неймана~\cite{Neumann1930,Neumann1933} й детально викладена в~\cite{AkhGlaz78}. Тут нагадується означення симетричного і самоспряженого операторів в гільбертовому просторі, визначення їх дефектних чисел і наведено дві останні формули Дж.~фон~Неймана.

Нехай $\frH$ --- гільбертів простір над полем $\bbC$, $\calD$ --- лінеал в $\frH$ і $T:\calD\mapsto \frH$ --- лінійне відображення, тобто
\begin{equation*}
	T(\lambda_1f_1+\lambda_2f_2) = \lambda_1Tf_1+\lambda_2Tf_2, \quad \forall f_1,f_2\in \calD,\ \lambda_1,\lambda_2\in\bbC.
\end{equation*}

Лінеал $\calD$ називають областю визначення оператора $T$ і позначають $\dom{T}$. Область значень оператора $T$ позначають $\ran{T}$.

\begin{definition} \label{def_adjoint operator}
	Нехай $T$ --- лінійний оператор в гільбертовому просторі $\frH$, $\overline{\dom}{T}=\frH$. Елемент $g\in\frH$ належить області визначення $\dom{T^*}$ спряженого з $T$ оператора $T^*$, якщо існує $h\in\frH$ такий, що
	\begin{equation}\label{eq-adj-oper}
		(Tf,g) = (f,h) \quad \forall f\in\dom{T}.
	\end{equation}
	В цьому випадку $T^*g=h$.
\end{definition}

\begin{remark*}
	Якщо $\overline{\dom}{T}\ne\frH$, то $T^*$ є лінійним відношенням (див. у Pозділі \ref{Subsec-Linear-relations}).
\end{remark*}

\begin{definition} \label{def_Symmetr_operator}
	Лінійний оператор $A$ називається симетричним, якщо  $\overline{\dom}{A}=\frH$ і виконується рівність
	\begin{equation*}
		(Af,g) = (f,Ag), \quad \forall f,g\in \dom{A}
	\end{equation*}
\end{definition}

З означень \ref{def_adjoint operator} і \ref{def_Symmetr_operator} випливає, що для симетричного оператора $A$ виконується включення $A\subset A^*$.

\begin{definition}\label{def_self-adj-oper}
	Оператор $A$ називається самоспряженим, якщо $A=A^*$. 
\end{definition}
% \begin{remark*}
% 	Означення \ref{def_self-adj-oper} еквівалентно тому, що оператор $A$ самоспряжений, якщо він симетричний і $\dom{A}=\dom{A^*}$.	
% \end{remark*}

\begin{definition}
	Графіком лінійного оператора $T$ називається множина
	\begin{equation*}
		\gr{T} = \{ (f,Tf): f\in\dom{T} \}\subset \frH\times\frH.
	\end{equation*}
\end{definition}

\begin{definition}
	Оператор $T$ (не обов'язково лінійний) називається замкненим, якщо з одночасного виконання умов
	\begin{equation*}
		f_n\in\dom{T}, \quad \lim_{n\to\infty}{f_n}=f, \quad \lim_{n\to\infty}{Tf_n}=g 
	\end{equation*}
	випливає, що
	\begin{equation*}
		f\in\dom{T}, \quad g=Tf.
	\end{equation*}
\end{definition}
\begin{remark*}
	Лінійний оператор $T$ є замкненим, якщо його графік замкнений в $\frH\times\frH$. 
\end{remark*}

\begin{definition}
	Оператор $\wt A$ називається розширенням $A$, якщо 
	\begin{equation*}
		\dom A \subset \dom{\wt A} \quad \text{і} \quad \wt Af = Af,\ f\in A.
	\end{equation*}
	При цьому, якщо $\dom A = \dom{\wt A}$, то $A = \wt A$.
\end{definition}

\begin{definition}
	Розширення $\wt A$ симетричного оператора $A$ називають власним розширенням, якщо $A\subsetneq \wt A \subsetneq A^*$.
\end{definition}

Якщо $\wt A$ --- симетричне розширення оператора $A$ ($A\subset \wt A$), то $(\wt A)^* = \wt{A^*} \subset A^*$ і, отже,
\begin{equation*}
	A\subset \wt A \subset \wt{A^*} \subset A^*.
\end{equation*}
А з останнього випливає, що симетричне розширення $\wt A$ оператора $A$ обов'язково є його власним розширенням.

Позначимо $\ker{A} = \{ f\in\dom{A}:Af=0 \}$ ядро оператора $A$.
 
\begin{definition}
	Оператор $V:\frH_1 \mapsto \frH_2$ ($\frH_1$ і $\frH_2$ можуть бути підпросторами одного простору) називають ізометричним, якщо для всіх $f,g\in \frH_1$ виконується
	\begin{equation*}
		(Vf,Vg)_2 = (f,g)_1
	\end{equation*}
\end{definition}

\begin{definition}
	Нехай $T$ --- довільний лінійний оператор. Число $\lambda$ називається точкою регулярного типу оператора $T$, якщо існує $k(\lambda)>0$ така, що $\forall f\in \dom{T}$ виконується
	\begin{equation*}
		||(T-\lambda I)f||\ge k||f||.
	\end{equation*}
	При цьому множину всіх точок регулярного типу оператора $T$ називають полем регулярності цього оператора і позначають $\widehat\rho(T)$. 
\end{definition}

Зрозуміло, що власні значення оператора $T$ не є його точками регулярного типу.

\begin{remark*}
	Число $\lambda$ є точкою регулярного типу оператора $T$ тоді і тільки тоді, коли оператор $(T-\lambda I)^{-1}$ існує і є обмеженим на множині $\ran{(T-\lambda I)}$ значень оператора $(T-\lambda I)$.
\end{remark*}

\begin{remark*}
	Множина точок регулярного типу завжди є відкритою множиною.
\end{remark*}

Для симетричного оператора $A$ і $z=x+iy$ $(y \ne 0)$ $\forall f\in \dom{A}$:
\begin{equation*}
	||(A-zI)f||^2 = ||(A-xI)f||^2 + y^2||f||^2 \ge y^2||f||^2.
\end{equation*}
Тобто верхня і нижня комплексні півплощини є зв'язними компонентами поля регулярності оператора $A$.

Для ізометричного оператора $V$ і $\xi \in \bbC$ зв'язними компонентами поля регулярності є внутрішня частина одиничного кола $(|\xi|<1)$ і його зовнішня частина $(|\xi|>1)$, оскільки
\begin{gather*}
	||(V-\xi I)f|| \ge ||Vf|| - |\xi|\cdot ||f|| = (1-|\xi|)\cdot ||f||,\ |\xi|<1; \\
	||(V-\xi I)f|| \ge |\xi|\cdot ||f|| - ||Vf|| = (|\xi|-1)\cdot ||f||,\ |\xi|>1.
\end{gather*}

\begin{theorem}\normalfont{\cite{AkhGlaz78}} \label{dim_for_all_lambda}
	Якщо $\Omega$ є зв'язна компонента поля регулярності лінійного оператора $T$, то розмірність підпростору $\frH \ominus \ran{(T - \lambda I)}$ однакова для всіх $\lambda \in \Omega$.
\end{theorem}

\begin{definition}
	Нехай оператор $T$ є замкненим в $\frH$.
	\begin{enumerate}
		\item Якщо $z\in\widehat\rho(T)$ і $\ran(T-zI)=\frH$, то $z$ називають регулярною точкою оператора $T$.
		\item Сукупність регулярних точок оператора $T$ називають його резольвентною множиною і позначають $\rho(T)$.
		\item Множина $\sigma(T)=\bbC\setminus\rho(T)$ називають спектром оператора $T$.
		\item Множина $\widehat\sigma(T)=\bbC\setminus\widehat\rho(T)$ називають ядром спектра оператора $T$.
	\end{enumerate}
\end{definition}

\begin{definition}
	Нехай оператор $T$ є замкненим в $\frH$.
	\begin{enumerate}
		\item Точковим спектром оператора $T$ називають множину
		\begin{equation}
		 	\sigma_p(T) = \{z\in\bbC : \ker(T-zI)\ne\{0\}\}.
		\end{equation}
		\item Неперервним спектром оператора $T$ називають множину
		\begin{equation}
			\sigma_c(T) = \{z\in\bbC\setminus\sigma_p(T) : \ran(T-zI)\ne\overline{\ran(T-zI)}\}.
		\end{equation}
		\item Залишковим спектром оператора $T$ називають множину
		\begin{equation}
			\sigma_r(T) = \sigma(T)\setminus\widehat\sigma(T).
		\end{equation}
	\end{enumerate}
\end{definition}

\begin{definition}
	Дефектним числом лінійного многовиду $\frM$ називають розмірність його ортогонального доповнення $\frN = \frH \ominus \frM$ ($\defect \frM = \dim{\frN}$).
\end{definition}

\begin{definition}
	Дефектне число лінійного многовиду $\ran{(T - \lambda I)}$ точок $\lambda \in \Omega$ поля регулярності оператора $T$ називають дефектним числом оператора $T$ в компоненті зв'язності $\Omega$ поля регулярності $T$. При цьому $\frN_\lambda = \frH \ominus \ran{(T-\bar\lambda I)}$ називають дефектним підпростором оператора $T$ для точки $\lambda$, а будь-який ненульовий елемент $\frN_\lambda$ називають дефектним елементом.
\end{definition}

Для симетричного оператора $A$:
\begin{equation*}
	\defect \ran (A-\bar{z}I) = 
	\begin{cases}
		m, & \Im z > 0,\\
		n, & \Im z < 0.
	\end{cases}
\end{equation*}

Для ізометричного оператора $V$:
\begin{equation*}
	\defect \ran (I - \bar{\xi}I) = 
	\begin{cases}
		m, & |\xi| > 1,\\
		n, & |\xi| < 1.
	\end{cases}
\end{equation*}

\begin{definition}
	Дефектні числа симетричного (ізометричного) оператора утворюють впорядковану пару $(n_+,n_-) := (m,n)$, яку називають індексами дефекту оператора.
\end{definition}

\begin{corollary} \label{cor2}\
	\begin{enumerate}
		\item Для симетричного оператора $A$: $n_+=n_-$, якщо $A$ має дійсну точку регулярного типу. \\
		Для ізометричного оператора $V$: $n_+=n_-$, якщо $V$ має точку регулярного типу, що належить одиничному колу.
		\item \label{cor2-i2} Якщо $A$ --- симетричний оператор, то будь-яке $z\notin \bbR$ є для спряженого оператора $A^*$ власним значенням:
		\begin{itemize}
			\item кратності $m$, якщо $\Im z < 0$,
			\item кратності $n$, якщо $\Im z > 0$. 
		\end{itemize}
		\item Дефектні числа ізометричного оператора $V$ можуть бути визначені за допомогою рівностей:
		\begin{equation*}
			\begin{cases}
				n_+ = \defect \dom V,\\
				n_- = \defect \ran V.
			\end{cases}
		\end{equation*}
		\item Якщо $A$ --- симетричний оператор в $\frH$, а $B$ --- обмежений самоспряжений оператор, то індекси дефекту $A$ і $A+B$ є однаковими.
	\end{enumerate}
\end{corollary}

\begin{definition}
	Наступне перетворення $V$ замкненого симетричного оператора $A$ називається перетворенням Келі:
	\begin{equation} \label{Cayley_f_cases}
		\begin{cases}
			(A - \bar zI)h = f;\\
			(A - zI)h = Vf,
	 	\end{cases}
	\end{equation}
 	якщо $z\notin \bbR$, $h\in \dom{A}$.
\end{definition}

Беручи до уваги теорему~\ref{dim_for_all_lambda}, побачимо, що
\begin{equation}
	\begin{cases}
		m = \defect \ran{A(\bar z)} = \defect \dom{V}, \\
		n = \defect \dom{A(z)} = \defect \ran{V}, 
	\end{cases}
\end{equation}
тобто індекси дефекту $(m,n)$ оператора $A$ співпадає з індексом дефекту оператора $V$.

\begin{theorem}\normalfont{\cite{AkhGlaz78}}
	Якщо $V$ --- ізометричний оператор і якщо многовид $\ran (I-V)$ є щільним в $\frH$, то оператор $A$ є симетричним оператором, а $V$ --- його перетворення Келі.
\end{theorem}

\begin{theorem}\normalfont{\cite{AkhGlaz78}}
	Нехай $A$ і $\wt A$ --- симетричні оператори, а $V$ і $\wt V$ --- їх перетворення Келі. Тоді $\wt A$ є розширенням $A$ тоді і тільки тоді, коли $\wt V$ є розширенням $V$.
\end{theorem}

Таким чином, щоб знайти деяке симетричне розширення $\wt A$ симетричного оператора $A$, необхідно спочатку перейти до його перетворення Келі $V$, а після його розширення до $\wt V$~--- назад.

Ізометричне розширення $\wt V$ оператора $V$ можна визначити наступним чином:
\begin{equation} \label{extention_V}
	\wt V = 
	\begin{cases}
		Vf, \ f\in \dom{V},\\
		V_1f,\ f\in \calF,
	\end{cases}
\end{equation}
де $\calF$ і $\calG$ --- підпростори однакової розмірності дефектних підпросторів $\frH\ominus\dom{V}$ і $\frH\ominus\ran{V}$ оператора $V$, а $V_1:\calF\mapsto\calG$ --- довільний ізометричний оператор.

\begin{theorem}\normalfont{\cite{AkhGlaz78}}
	\begin{enumerate}
		\item Для того, щоб симетричний оператор був максимальним, необхідно і достатньо, щоб одне з його дефектних чисел дорівнювало нулю. 
		\item Для того, щоб симетричний оператор був самоспряженим, необхідно і достатньо, щоб обидва його дефектних числа дорівнювали нулю. 
	\end{enumerate}
\end{theorem}

\begin{theorem}\normalfont{\cite{AkhGlaz78}}
	Нехай $A$ --- довільний симетричний оператор з індексами дефекту $n_\pm$. Оператор $A$ завжди можна розширити до максимального, але:
	\begin{itemize}
		\item якщо $n_+\ne n_-$, то серед розширень немає самоспряжених;
		\item якщо $n_+=n_-<\infty$, то будь-яке максимальне розширення оператора $A$ є самоспряженим;
		\item якщо $n_+=n_-=\infty$, то серед розширень є як самоспряжені, так і ні.
	\end{itemize}
\end{theorem}

\begin{theorem}\normalfont{\cite{AkhGlaz78}}
	Нехай $A$ --- довільний симетричний оператор з областю визначення $\dom A$, а $\frN_{\bar z}$ і $\frN_z$ $(\Im z > 0)$ --- деяка пара його дефектних підпросторів. Тоді область визначення $\dom{A^*}$ оператора $A^*$ може бути подана у наступному вигляді:
	\begin{equation} \label{Neumann_1}
		\dom{A^*} = \dom{A} \oplus \frN_{\bar z} \oplus \frN_z.
	\end{equation}
\end{theorem}

Формула~\eqref{Neumann_1} називається першою формулою Неймана і дає представлення області визначення спряженого до $A$ оператора.

З неї випливає, що симетричний оператор $A$ є самоспряженим тоді і тільки тоді, коли він має індекси дефекту $n_+=n_-=0$.

Знайдемо область визначення $\dom{\wt A}$ симетричного розширення $\wt A$ оператора $A$. Оберемо $\calF_z \subseteq \frN_{\bar z} = \frH \ominus \ran{A(\bar z)}$ і $\calG_z \subseteq \frN_{z} = \frH \ominus \ran{A(z)}$. Тоді з~\eqref{extention_V} випливає:
\begin{equation*}
 	\dom{\wt A} = (\wt V - I)\dom{\wt V} = (\wt V -I)(\dom{V} \oplus \calF_z) = (V - I)\dom{V} \oplus (V_1 - I)\calF_z = \dom{A} \oplus (V_1 - I)\calF_z.
\end{equation*}

\begin{theorem}\normalfont{\cite{AkhGlaz78}} \label{Neumann_2_Theorem}
	Формула
	\begin{equation}
 		\dom{\wt A} = \dom{A} \oplus (I - V_1)\calF_z.
	\end{equation}
	встановлює взаємно-однозначну відповідність між множиною замкнених симетричних розширень $\wt A$ оператора $A$ і множиною частково ізометричних операторів $V_1: \frN_{\bar z} \mapsto \frN_z$.
	При цьому розширення $\wt A$ оператора $A$ є самоспряженим тоді і тільки тоді, коли $V_1$ --- унітарне відображення з $\frN_{\bar z}$ на $\frN_z$.
\end{theorem}
% або, покладаючи $V_1 = -V'$,
% \begin{equation*}
% 	D_{\wt A} = D_A \oplus (V' + I)F_z.
% \end{equation*}
% Із $\wt A \subset A^*$ випливає, що при 
% \begin{equation*} 
% 	f = f_0 + g_z +V'g_z \quad (f\in D_{\wt A}, \ f_0\in D_A, \ g_z \in F_z)
% \end{equation*}
% будемо мати
% \begin{equation} \label{Neumann_2}
% 	\wt Af = Af_0 + zg_z + \bar zV'g_z.
% \end{equation}

% Формула~\eqref{Neumann_2} називається \emph{другою формулою Неймана} і описує всі розширення $\wt A$ оператора $A$. 

% Якщо для оператора $A$ його індекси дефекту $m=n\neq 0$, то $\wt A$ є самоспряженим оператором і $g_z$ пробігатиме весь простір $\frN_{\bar z}$, а $V'g_z$ --- весь простір $\frN_z$, тобто $D_{A^*} = D_{\wt A}$.

% subsection розширення_симетричних_операторів (end)

\subsection{Клас Піка-Неванлінни-Герглотца} % (fold)
%\label{sub:клас_піка_неванлінни_герглотца}

В цьому розділі наводяться необхідні відомості з теорії функцій, зокрема теорії $R$-функцій, тобто аналітичних в верхній півплощині функцій зі значеннями в $\bbC_+=\{z: \Im(z)\ge 0\}$.
Термін $R$-функції був запропонований в літературі по теорії електричних ланцюгів~\cite{LaneTom1958}. Інтегральні представлення таких функцій було отримано паралельно в роботах Р.~Неванлінyи~\cite{Nev1919}, Ф.~Ріса~\cite{Riesz1913}, Г.~Піка~\cite{Pick1916} та Г.~Герглотца~\cite{Her1911}.

\begin{definition}
	Будемо казати, що функція $f$ належить класу Піка-Неванлінни-Герглотца ($R$), якщо $f$ голоморфна в $\bbC_+$ і $\Im f(\lambda)\ge 0$ для всіх $\lambda \in \bbC_+$.
\end{definition}

\begin{theorem}\normalfont{\cite{KacKre1968}} \label{th-IntR}
	Для того, щоб $f$ належала класу $R$ необхідно і достатньо, щоб вона допускала інтегральне представлення
	\begin{equation}\label{eq-IntR}
		f(\lambda) = A+B\lambda+\int\limits_{-\infty}^{+\infty} \left( \frac{1}{t-\lambda} - \frac{t}{1+t^2} \right)\,d\sigma(t),
	\end{equation}
	де $A=\bar{A}$, $B\ge 0$, а $\sigma(t)$ --- неперервна справа неспадна функція така, що
	\begin{equation*}
		\int\limits_{-\infty}^{+\infty} \frac{d\sigma(t)}{1+t^2}<\infty.
	\end{equation*}
\end{theorem}

\begin{definition}
	Нехай $\calH$ --- допоміжний гільбертів простір. Будемо говорити, що оператор-функція $F(\lambda)$ належить до класу $R[\calH]$, якщо
	\begin{enumerate}
		\item $F(\cdot)$ голоморфна в $\bbC_+\cup\bbC_-$;
		\item $\Im F(\lambda)\ge0$ для $\lambda\in\bbC_-$;
		\item $F(\bar{\lambda}) = F(\lambda)^*$ для $\lambda\in\bbC_+\cup\bbC_-$.
	\end{enumerate}
\end{definition}

Оператор-функції $F(\cdot)\in R[\calH]$ допускають інтегральне представлення \eqref{eq-IntR}, в якому $A$ і $B$ є операторами, а $\sigma(t)$ --- неспадна оператор-функція, така що
\begin{equation*}
	\int\limits_{-\infty}^{+\infty} (1+t^2)^{-1}\,d(\sigma(t)h,h) < \infty \quad \forall h\in\calH
\end{equation*}

\begin{corollary}
	Якщо $u$ --- невід'ємна гармонійна функція в $\bbC_+$, то існують $B\ge 0$ і неперервна справа неспадна функція $\sigma(t)$ така, що
	\begin{equation} \label{eq-ImR}
		u(\lambda)=By + \int\limits_{-\infty}^{+\infty} \frac{y}{(t-x)^2+y^2}\,d\sigma(t).
	\end{equation}
\end{corollary}

\begin{remark}
	В умовах Теореми~\ref{th-IntR} $A, B$ --- єдині і визначаються рівностями:
	\begin{equation*}
		A:=\Re f(i),\quad B:=\lim_{y\to \infty} \frac{\Im f(iy)}{y}.
	\end{equation*}
\end{remark}

\begin{theorem} [Формула обертання Стілт'єса]
	В умовах Теореми~\ref{th-IntR} функція $\sigma(t)$ у своїх точках неперервності визначається рівністю
	\begin{equation} \label{eq-Stil}
		\sigma(b)-\sigma(a)=\frac{1}{\pi} \lim_{y \downarrow 0} \int\limits_a^b \Im f(x+iy)\,dx.
	\end{equation}
\end{theorem}

\begin{remark}
	Якщо $a$ і $b$ --- довільні точки, то функція розподілу приймає вигляд:
	\begin{equation*}
		\frac{\sigma(b+0)+\sigma(b-0)}{2} - \frac{\sigma(a+0)+\sigma(a-0)}{2} = \frac{1}{\pi} \lim_{y \downarrow 0} \int\limits_a^b \Im f(x+iy)\,dx.
	\end{equation*}
\end{remark}

\begin{definition}
	Функція $f\in R$ відноситься до класу $R_0$, якщо вона має інтегральне представлення 
	\begin{equation} \label{eq-R0}
		f(\lambda) = \int\limits_{-\infty}^{+\infty} \frac{d\sigma(t)}{t-\lambda},
	\end{equation} 
	де $\sigma(\lambda)$ --- обмежена неспадна функція, тобто
	\begin{equation} \label{eq-R0-sigma}
		\int\limits_{-\infty}^{+\infty} d\sigma(t)<\infty.
	\end{equation}
\end{definition}

\begin{definition}
	Будемо казати, що $f\in R$ належить класу $R_1$, якщо вона допускає інтегральне представлення
	\begin{equation} \label{eq-R1}
		f(\lambda) = \gamma + \int\limits_{-\infty}^{+\infty} \frac{d\sigma(t)}{t-\lambda},
	\end{equation}
	де
	\begin{equation} \label{eq-R1-sigma}
		\int\limits_{-\infty}^{+\infty} \frac{d\sigma(t)}{1+|t|}<\infty.
	\end{equation}
\end{definition}

\begin{theorem}\normalfont{\cite{KacKre1968}} \label{th-R0-n&s}
	Нехай $f\in R$. Тоді наступні твердження є еквівалентними:
	\begin{enumerate}
		\item $f\in R_0$;
		\item $\sup\limits_{y>0} |yf(iy)| < \infty$;
		\item $\sup\limits_{y>0} |y\Im f(iy)| < \infty$, $\lim\limits_{y\uparrow\infty} f(iy)=0$.
	\end{enumerate}
\end{theorem}

\begin{theorem}\normalfont{\cite{KacKre1968}} \label{th-R1-n&s}
	Для того, щоб $R$-функція $f(\lambda)$ належала класу $R_1$ необхідно і достатньо, щоб збігався інтеграл
	\begin{equation} \label{eq-R1-n&s}
		\int\limits_1^{+\infty} \frac{\Im f(i\eta)}{\eta}\,d\eta.
	\end{equation}
\end{theorem}


% subsection клас_піка_неванлінни_герглотца (end)

\subsection{Класи Стілт'єса. Клас функцій $S^+$}

Класи Стілт'єса було введено і досліджено М.\,Г.~Крейном в його роботах~\cite{Krein1946,Krein1951,Krein1952}. Позначення цих класів було дано на честь Т.~Стілт'єса.

\begin{definition} \label{ClassS}
	Кажуть, що функція $f$ належить класу $S^+$, якщо
	\begin{enumerate}
		\item $f\in R $;
		\item $f$ --- голоморфна в $\bbC\setminus\left[0{,}\infty\right)$;
		\item $f(x)\ge 0$, для всіх $x<0$.
	\end{enumerate}
\end{definition}

\begin{theorem}\normalfont{\cite{KacKre1968}}\label{th-IntS} 
	Для того, щоб $f\in S^+$, необхідно і достатньо, щоб функція $f$ допускала наступне інтегральне представлення
	\begin{equation}\label{eq-IntS}
	    f(\lambda)=\gamma + \int\limits_{-0}^{\infty} \frac{d\sigma(t)}{t-\lambda},
	\end{equation}
	де $\gamma \ge 0$, $\sigma(t)$ --- неспадна функція така, що
	\begin{equation}\label{eq-IntS-sigma}
		\int\limits_{0}^{\infty} \frac{d\sigma(t)}{t+1} < \infty .
	\end{equation}
\end{theorem}

\begin{theorem}\normalfont{\cite{KacKre1968}}\label{th-preMain}
	Нехай $f\in R$. Тоді наступні твердження еквівалентні:
	\begin{enumerate}
		\item $f\in S^+$;
		\item $\lambda f(\lambda)\in R$;
		%\item $  \lambda f(\lambda^2)\in R $.
	\end{enumerate}
\end{theorem}

%subsection клас_S+

\subsection{Перетворення розгортання}

Для довільної функції $f$ мероморфної в $\bbC\setminus \bbR_+$ визначено її перетворення $\wt f$, що визначається формулою
\begin{equation} \label{eq-Turn}
	\wt f(z) := zf(z^2), \quad (z\in\bbC_+).
\end{equation}
Це перетворення називають перетворенням розгортання функції $f(z)$~\cite{KalWinWor2006,DerKov2015}.

Як відомо, для функції $f$ з класу Стілт'єса її перетворення розгортання $\wt f$ належить до класу $S$.

\begin{theorem}\normalfont{\cite{KacKre1968}}\label{th-main}
	 Нехай $f\in S$. Тоді наступні твердження еквівалентні:
	 \begin{enumerate}
	 	\item $zf(z)\in R$,
	 	\item $zf(z^2)\in R$.
	 \end{enumerate}
\end{theorem}

Таким чином, з Теореми~\ref{th-main} випливає, що перетворення розгортання відображає клас $S$ в частину класу $R$.

Наступна теорема відповідає на питання які додаткові умови характеризують функції з класу $R$, що утворені перетворенням розгортання для деякої функції $f\in S$.

\begin{definition}
	Будемо казати, що функція $f\in R$ є симетричною і писати $\wt f\in R^{SYM}$, якщо
	\begin{equation} \label{eq-SYM}
	 	\wt f(-z) = -\wt f(z).
	 \end{equation} 
\end{definition}

\begin{theorem}
	 Перетворення розгортання встановлює взаємно-однозначну відповідність між класами $S$ і $R^{SYM}$.
\end{theorem}

% subsection Перетворення розгортання
% section перший_розділ (end)
%!TEX root = ../Masters.tex

\section {ГРАНИЧНІ ТРІЙКИ ТА ФУНКЦІЇ ВЕЙЛЯ}

\subsection{Лінійні відношення} % (fold)
\label{Subsec-Linear-relations}

В цьому розділі розглядаються деякі відомості про лінійні відношення в гільбертовому просторі. Поняття лінійного відношення в банаховому просторі було введено і вивчалося Р.~Аренсом~\cite{Arens1961}, хоча в іншому вигляді воно зустрічалося в більш ранніх роботах, наприклад, в~\cite{Calkin1939}.

Нехай $\frH$ --- гільбертів простір і $\frH^2 = \frH \times \frH$ --- декартовий добуток двох екземплярів простору $\frH$. 
Елементи простору $\frH^2$ будемо позначати $\widehat f = \dbinom{f_1}{f_2},\ (f_1,f_2 \in \frH)$. Для $\widehat f \in \frH^2$ та $\widehat g = \dbinom{g_1}{g_2} \in \frH^2$ покладемо
\begin{equation}
 	{\langle \widehat f,\widehat g\rangle}_{\frH^2} = (f_1,g_1)_\frH + (f_2,g_2)_\frH.
\end{equation} 

Позначимо через $\pi_1$ та $\pi_2$ проектори на першу та другу компоненту в $\calH \times \calH$ відповідно.

\begin{definition}
	Лінійний підпростір $\Theta \in \frH^2$ називається лінійним відношенням в $\frH$. Лінійне відношення називається замкненим, якщо підпростір $\Theta$ є замкненим в $\frH^2$. Сукупність замкнених лінійних відношень в $\frH$ позначимо $\wt \calC(\frH)$. 
\end{definition}

Множини
\begin{gather*}
	\dom \Theta = \left\{f_1\in \frH: \binom{f_1}{f_2}\in \Theta \text{ для деякого } f_2\in\frH\right\} = \pi_1\Theta,\\
	\ran \Theta = \left\{f_2\in \frH: \binom{f_1}{f_2}\in \Theta \text{ для деякого } f_1\in\frH\right\} = \pi_2\Theta
\end{gather*}
називаються областю визначень та областю значення лінійного відношення, а множини
\begin{equation*}
	\ker \Theta = \left\{ \pi_1\widehat f: \widehat{f}\in\Theta, \pi_2\widehat{f}=0\right\}, \quad \mul\Theta = \left\{\pi_2\widehat f: \widehat{f}\in\Theta, \pi_1\widehat{f}=0\right\}
\end{equation*}
називаються відповідно ядром і багатозначною частиною лінійного відношення $\Theta$.
Обернене до $\Theta$ лінійне відношення $\Theta^{-1}$ в $\frH$ визначається співвідношенням
\begin{equation*}
	\Theta^{-1} = \left\{\binom{f_1}{f_2}:\binom{f_2}{f_1}\in\Theta\right\}.
\end{equation*}
Спряжене лінійне відношення $\Theta^*$ визначається рівністю~\cite{Bennewitz1972,Coddington1973}
\begin{equation*}
	\Theta^*=\left\{\binom{h}{k}\in\frH'\oplus\frH:(k,f)_\frH=(h,g)_{\frH'}, \binom{f}{g}\in\Theta\right\}.
\end{equation*}

На відміну від оператора, лінійне відношення завжди можна замкнути. Більше того, в класі $\wt\calC(\calH)$ замкнених лінійних відношень завжди існують спряжене і обернене до $\Theta$ лінійні відношення. Ці переваги дозволяють після ототожнення оператора $T$ з його графіком $\Theta_T=\gr{T}$, розглядати $\bar\Theta_T$, $\Theta^*_T$, $\Theta^{-1}_T$ і роблять лінійні відношення незамінними при роботі з операторами.

Сума $\Theta_1 + \Theta_2$ і покомпонентна сума $\Theta_1 \widehat+ \Theta_2$ двох лінійних відношень $\Theta_1$ і $\Theta_2$ визначаються рівностями

\begin{equation}\label{eq_lr_sum}
	\Theta_1+\Theta_2=\left\{\binom{f}{g+k}: \binom{f}{g}\in\Theta_1, \binom{f}{k}\in\Theta_2\right\},
\end{equation}
\begin{equation*}
	\Theta_1\widehat+\Theta_2=\left\{\binom{f+h}{g+k}: \binom{f}{g}\in\Theta_1, \binom{h}{k}\in\Theta_2 \right\}.
\end{equation*}
Якщо покомпонентна сума є прямою (ортогональною), то вона позначається відповідно $\Theta_1 \dot+ \Theta_2$ ($\Theta_1 \oplus \Theta_2$).

Зрозуміло, що 
\begin{align*}
	\dom{\Theta^{-1}} &= \ran{\Theta}, & \ran{\Theta^{-1}} &= \dom{\Theta},\\
	\ker{\Theta^{-1}} &= \mul{\Theta}, & \mul{\Theta^{-1}} &= \ker{\Theta}.
\end{align*}

Ототожнюючи оператор $\lambda I\ (\lambda\in\bbC)$ з його графіком, отримаємо у відповідності з~\eqref{eq_lr_sum}
\begin{equation}
	\Theta-\lambda I=\left\{ \binom{f_1}{f_2-\lambda f_1} : \binom{f_1}{f_2}\in\Theta\right\}
\end{equation}

\begin{definition}
	Нехай $\Theta \in \wt\calC(\calH)$. Точку $\lambda \in \bbC$ називають регулярною точкою лінійного відношення $\Theta$ і пишуть $\lambda\in\rho(\Theta)$, якщо $\ker(\Theta-\lambda I) = \{0\}$ і $\ran (\Theta-\lambda I) = \frH$. Спектр лінійного відношення позначають $\sigma(\Theta):=\bbC\setminus\rho(\Theta)$. Точковий та неперервний спектри лінійного відношення $\Theta$ визначається рівностями
	\begin{equation*}
		\sigma_p(\Theta) = \left\{\lambda\in\bbC:\ker(\Theta-\lambda I)\ne \{0\} \right\},
	\end{equation*}
	\begin{equation*}
		\sigma_c(\Theta) = \left\{\lambda\in\bbC: \ker(\Theta-\lambda I)=\{0\},\ \ran(\Theta-\lambda I) \ne \overline{\ran (\Theta-\lambda I)} = \frH \right\}.
	\end{equation*}
\end{definition}


% section третій_розділ_лінійні_відношення (end)

% TODO добавить определение 5.36 (стр. 180)

\subsection{Граничні трійки для симетричних операторів} % (fold)

Підхід до теорії розширень симетричних операторів і формули Дж.~фон~Неймана виявився не зручним у застосуванні до граничних задач. У зв'язку з цим Дж.~Калкіним було запропоновано інший підхід, який базується на понятті "абстрактної граничної умови".
Надалі цей підхід застосовувався в роботах М.\,І.~Вішіка~\cite{Visik1952} з теорії розширень диференціальних операторів в частинних похідних, М.\,Л.~Горбачука~\cite{Horb1971} з теорії операторів Штурма-Ліувілля з необмеженим операторним коефіцієнтом.
В роботах А.\,Н.~Кочубея~\cite{Koch1975} і В.\,І.~Горбачук та М.\,Л.~Горбачука~\cite{Gorb1991} цей підхід трансформувався в теорію "абстрактних граничних просторів". Надалі використовується термінологія робіт В.~Деркача і М.~Маламуда, де ці об'єкти називаються граничними трійками.


Нехай $\frH$ --- гільбертів простір, $A$ --- замкнений симетричний оператор в $\frH$ із щільною областю визначення $\dom A$ і рівними індексами дефекту. 

\begin{definition}
	Сукупність $\Pi = \{\calH, \Gamma_0, \Gamma_1\}$, де $\calH$ --- гільбертів простір, $\Gamma_j:\dom{A^*} \mapsto \calH$ $(j\in \{0,1\})$ --- лінійні відображення, називається граничною трійкою для оператора $A^*$, якщо:
	\begin{enumerate}
		\item виконується формула Гріна
		\begin{equation}\label{eq-CS-Green}
			(A^*f,g)_\frH - (f,A^*g)_\frH = (\Gamma_1f,\Gamma_0g)_\calH - (\Gamma_0f,\Gamma_1g)_\calH \quad f,g \in \dom A^*;
		\end{equation}
		\item  відображення $\Gamma = \dbinom{\Gamma_0}{\Gamma_1}: \dom A^* \mapsto \calH\oplus\calH$ є сюр'єктивним.
	\end{enumerate}
\end{definition}

\begin{definition}
	Два розширення $\wt {A_1}$ і $\wt {A_2}$ оператора $A$ називаються диз'юнктними, якщо $\dom \wt {A_1} \cap \dom \wt {A_2} = \dom A$, і трансверсальними, якщо вони є диз'юнктними і $\dom \wt {A_1} + \dom \wt {A_2} = \dom A^*$.
\end{definition}

З кожною граничною трійкою пов'язані два розширення оператора $A$: 
\begin{equation}\label{eq_bt_1}
	A_j = A^* \upharpoonright \dom A_j, \qquad \dom A_j = \ker \Gamma_j,\  j\in \{0,1\}.
\end{equation}

\begin{proposition}
	Нехай розширення $A_j$ $(j\in \{0,1\})$ визначено рівностями~\eqref{eq_bt_1}. Тоді:
	\begin{enumerate}
		\item $A_j = A^*_j$, $j\in \{0,1\}$;
		\item розширення $A_0$ і $A_1$ трансверсальні.
	\end{enumerate}
\end{proposition}

\begin{proposition}
	Нехай $A$ симетричний оператор в $\frH$ з рівними дефектними числами, $\overline{\dom A} = \frH$ і $A'$ --- деяке самоспряжене розширення оператора $A$. Тоді існує гранична трійка $\Pi = \{\calH, \Gamma_0, \Gamma_1\}$ оператора $A^*$ така, що $\dom A' = \ker \Gamma_0$, тобто $A' = A_0$.
\end{proposition}

\begin{definition}
	Сукупність всіх власних розширень оператора $A$, поповнену операторами $A$ і $A^*$, позначають через $\Ext_A$.
\end{definition}

\begin{proposition}
	Відображення $\Gamma:\dom{A^*}\mapsto \calH\oplus\calH$ задає бієктивну відповідність між сукупністю $\Ext_A$ і сукупністю $\wt\calC(\calH)$ замкнених лінійних відображень в $\calH$.
	\begin{equation}
		\Ext_A \ni \wt A \mapsto \Theta := \Gamma(\dom \wt A) = \{ 
		{\begin{pmatrix} \Gamma_0 f & \Gamma_1f	\end{pmatrix}}^T : f\in \dom \wt A \} \in \wt \calC(\calH), 	
	\end{equation}
	(писатимемо $A_\Theta := \wt A$). При цьому виконуються наступні співвідношення:
	\begin{enumerate}
		\item $(A_\Theta)^* = A_{\Theta^*}$;
		\item $A_{\Theta_1} \subseteq A_{\Theta_2} \Leftrightarrow \Theta_1 \subseteq \Theta_2$;
		\item $A_{\Theta} \subseteq (A_{\Theta})^* \Leftrightarrow \Theta \subseteq \Theta^*$, зокрема $A_\Theta = (A_\Theta)^* \Leftrightarrow \Theta = \Theta^*$;
		\item $A_{\Theta_1}$ і $A_{\Theta_2}$ диз'юнктні $\Leftrightarrow \Theta_1 \cap \Theta_2 = \{0\}$;
		\item $A_{\Theta_1}$ і $A_{\Theta_2}$ трансверсальні $\Leftrightarrow \Theta_1 + \Theta_2 = \calH\oplus\calH$.
		% TODO соответствующее предложение 7.8 (стр.245) имеет еще 2 пункта, то они нужны только при наличии опеределения графика оператора.
	\end{enumerate}
\end{proposition}

\subsection{Функція Вейля}

Введемо поняття $\gamma$-поля і функції Вейля симетричного оператора з~\cite{DerMal2017}, що дозволяють досліджувати спектральні питання теорії розширень.

%TODO возможно, переставить в другое место
\begin{definition}
	Нехай $\frH_1$, $\frH_2$ --- гільбертові простори над полем $\bbC$. Позначимо через $\calB(\frH_1,\frH_2)$ множину лінійних обмежених операторів з $\frH_1$ в $\frH_2$ з областю визначення $\frH_1$. Зокрема, якщо $\frH_1=\frH_2=\frH$, покладемо $\calB(\frH)=\calB(\frH,\frH)$.
\end{definition}

\begin{definition}
	Нехай $A$ --- симетричний оператор в $\frH$, $\wt A = \wt{A^*}\in\Ext_A$ і $\calH$ --- деякий гільбертів простір, для якого $\dim \calH = n_\pm(A)$. Оператор-функцію $\gamma:\rho(\wt A) \mapsto \calB(\calH,\frH)$ називають $\gamma$-полем оператора $A$, що відповідає розширенню $\wt A$, якщо:
	\begin{enumerate}
		\item $\gamma(\lambda)$ ізоморфно відображає $\calH$ на $\frN_\lambda$ при всіх $\lambda\in\rho(\wt A)$;
		\item справджується тотожність:
		\begin{equation*}
			\gamma(\lambda) = U_{\zeta,\lambda}(\zeta) := [I+(\lambda-\zeta)(\wt A - \lambda)^{-1}]\gamma(\zeta), \qquad	\lambda,\zeta \in \rho(\wt A).
		\end{equation*}
	\end{enumerate}
\end{definition}

\begin{lemma}
	Нехай $\Pi = \{\calH, \Gamma_0, \Gamma_1\}$ --- гранична трійка для оператора $A^*$, $A_0:=A^*\upharpoonright \ker \Gamma_0$. Тоді:
	\begin{enumerate}
		\item при кожному $\lambda \in \rho(A_0)$ справедливим є розкладення
		\begin{equation*}
			\dom A^* = \dom A_0 + \frN_\lambda, \qquad \lambda\in \rho(A_0);
		\end{equation*}
		\item оператор-функція
		\begin{equation*}
			\gamma(\lambda):=(\Gamma_0\upharpoonright \frN_\lambda)^{-1}, \qquad \lambda\in\rho(A_0)
		\end{equation*}
		визначена коректно і голоморфна в $\rho(A_0)$ із значеннями в $\calB(\calH,\frN_\lambda)$;
		\item $\gamma(\lambda)$ є $\gamma$-полем оператора $A$, що відповідає розширенню $A_0$;
		\item справеджується тотожність
		\begin{equation*}
			\gamma(\bar \lambda)^* = \Gamma_1(A_0 - \lambda)^{-1}, \qquad	\lambda\in \rho(A_0).
		\end{equation*}
	\end{enumerate}
\end{lemma}

\begin{proposition}
	Нехай $\Pi = \{\calH, \Gamma_0, \Gamma_1\}$ --- гранична трійка для оператора $A^*$ і $B=B^*\in \calB(\calH)$. Тоді сукупність $\Pi^B_0 = \{\calH,\Gamma^B_0,\Gamma^B_1\}$, де
	\begin{equation*}
		\Gamma^B_0 = B\Gamma_0 - \Gamma_1, \qquad \Gamma^B_1 = \Gamma_0,
	\end{equation*}
	також є граничною трійкою для оператора $A^*$.
\end{proposition}

\begin{definition}
	Нехай $\Pi = \{\calH, \Gamma_0, \Gamma_1\}$ --- гранична трійка для оператора $A^*$. Оператор-функція $M(\cdot)$, що визначена рівністю
	\begin{equation*}
		M(\lambda)\Gamma_0f_\lambda = \Gamma_1f_\lambda, \qquad f_\lambda \in \frN_\lambda, \quad \lambda \in \rho(A_0), 
	\end{equation*}
	називається функцією Вейля оператора $A$, що відповідає граничній трійці $\Pi$.
\end{definition}

\begin{theorem} \label{th_vf_1}
	Нехай $\Pi = \{\calH, \Gamma_0, \Gamma_1\}$ --- гранична трійка для оператора $A^*$, $M(\cdot)$ --- відповідна функція Вейля. Тоді:
	\begin{enumerate}
		\item $M(\cdot)$ коректно визначена та голоморфна в $\rho(A_0)$ як оператор-функція із значеннями в $\calB(\calH)$;
		\item для всіх $\lambda, \zeta\in\rho(A_0)$ справджується тотожність 
		\begin{equation*}
			M(\lambda)-M(\zeta)^* = (\lambda-\bar\zeta)\gamma(\zeta)^*\gamma(\lambda), \qquad \lambda,\zeta \in \rho(A_0);
		\end{equation*}
		\item \label{iii} $M(\cdot)$ є $R[\calH]$-функцією та задовольняє умові
		\begin{equation}\label{eq_vf_1}
			0 \in \rho(\Im M(\lambda)), \qquad \lambda \in \bbC_+\cup\bbC_-;
		\end{equation}
		\item в кожній точці $\lambda \in \rho(A_0)$ існує (в рівномірній топології) похідна $M'(\lambda):=dM/d\lambda$ і
		\begin{equation*}
			M'(\lambda) = \gamma^*(\bar\lambda)\gamma(\lambda).
		\end{equation*}
		Якщо при цьому $\lambda \in \rho(A_0)\cap\bbR$, то $0 \in \rho(M'(\lambda))$, тобто оператор $M'(\lambda)$ є додатно визначеним.
	\end{enumerate}
\end{theorem}

\begin{theorem} \label{th_vf_2}
	Нехай $\Pi = \{\calH, \Gamma_0, \Gamma_1\}$ --- гранична трійка для оператора $A^*$, $M(\cdot)$ --- відповідна функція Вейля. Тоді справедливими є співвідношення:
	\begin{equation}\label{eq_vf_2}
		\lim_{y\uparrow\infty}y \cdot \Im(M(iy)h,h) = \infty, \qquad h \in \calH\setminus\{0\},
	\end{equation}
	\begin{equation}\label{eq_vf_3}
		s-\lim_{y\uparrow\infty}{\frac{M(iy)}{y}} = 0.
	\end{equation}
\end{theorem}

\begin{definition}
	Нехай $A^{(1)}$ і $A^{(2)}$ --- симетричні оператори в $\frH^{(1)}$ і $\frH^{(2)}$ відповідно, $\Pi^{(j)} = \{\calH, \Gamma^{(j)}_0,\Gamma^{(j)}_1\}$ --- гранична трійка для $A^{(j)*}$, $j \in \{1,2\}$. Граничні трійки $\Pi^{(1)}$ і $\Pi^{(2)}$ називають унітарно еквівалентними, якщо існує ізометричне відображення $U$ простору $\frH^{(1)}$ на $\frH^{(2)}$ таке, що
	\begin{equation}
		UA^{(1)*} = A^{(2)*}U, \qquad U\dom A^{(1)*} = \dom A^{(2)*},
	\end{equation}
	\begin{equation}
		\Gamma^{(1)}_k = \Gamma^{(2)}_k U, \qquad k \in \{0,1\}.
	\end{equation}
\end{definition}

\begin{theorem} \label{th_vf_3}
	Нехай $\calH$ --- сепарабельний гільбертів простір, $n:=\dim \calH \le \infty$, $M \in R[\calH]$ і виконуються умови~\eqref{eq_vf_2}, \eqref{eq_vf_3} і \eqref{eq_vf_1}. Тоді існує гільбертів простір $\frH$, простий щільно заданий оператор $A$ в $\frH$ з рівними індексами дефекту $(n,n)$ і гранична трійка $\Pi = \{\calH, \Gamma_0, \Gamma_1\}$ такі, що $M(z)$ є функцією Вейля оператора $A$, що відповідає граничній трійці $\Pi$.

	Гранична трійка $\Pi = \{\calH, \Gamma_0, \Gamma_1\}$ відновлюється за оператор-функцією $M(z)$ однозначно з точністю до унітарної еквівалентності. 
\end{theorem}

Таким чином, Теорема~\ref{th_vf_3}, з урахуванням Теореми~\ref{th_vf_1} \eqref{iii} і Теореми~\ref{th_vf_2}, дає повний опис всіх функцій Вейля щільно заданих симетричних операторів в гільбертовому просторі.

\begin{theorem}
	Нехай $\Pi=\{\calH,\Gamma_0,\Gamma_1\}$ --- гранична трійка для $A^*$, $M(\cdot)$ --- відповідна функція Вейля, $\Theta \in \wt\calC(\calH)$ і $\lambda \in \rho(A_0)$. Тоді правильними є наступні еквівалентності:
	\begin{equation*}
		\lambda \in \rho(A_\Theta) \Longleftrightarrow 0 \in \rho(\Theta - M(\lambda));
	\end{equation*}
	\begin{equation*}
		\lambda\in \sigma_i(A_\Theta) \Longleftrightarrow 0 \in \sigma_i(\Theta - M(\lambda)), \ i \in \{p,c,r\}.
	\end{equation*}
	При цьому мають місце рівності
	\begin{equation*}
		\ker (A_\Theta - \lambda) = \gamma(\lambda)\ker(\Theta - M(\lambda)). 
	\end{equation*}
\end{theorem}

\begin{theorem}
	Нехай $\Pi=\{\calH,\Gamma_0,\Gamma_1\}$ --- гранична трійка для $A^*$, $M(\cdot)$ --- відповідна функція Вейля, $\Theta \in \wt\calC(\calH)$, $A_\Theta$ --- відповідне власне розширення оператора $A$. Тоді:
	\begin{enumerate}
		\item формули
		\begin{equation}~\label{eq_vf_4}
			\dom (A_\Theta) = \{f\in\dom A^*:\Gamma f\in\Theta\}, \qquad \Theta:=\Gamma(\dom A_\Theta)
		\end{equation}
		встановлюють бієктивну відповідність між сукупністю всіх власних розширень $A_\Theta$ оператора $A$ і сукупністю замкнених лінійних відношень $\Theta\in\wt\calC(\calH)\setminus\{0\}$;
		\item якщо $\rho(A_\Theta)\ne\varnothing$, то для $z\in\rho(A_0)\cap\rho(A_\Theta)$ справедливою є рівність
		\begin{equation}~\label{eq_vf_5}
			(A_\Theta-z)^{-1} = (A_0-z)^{-1} + \gamma(z)(\Theta-M(z))^{-1}\gamma(\bar z)^*;
		\end{equation}
		\item рівність~\eqref{eq_vf_5} встановлює бієктивну відповідність між сукупністю резольвент власних розширень $A_\Theta$ оператора $A$, для яких $\rho(A_\Theta)\ne\varnothing$, і сукупністю замкнених лінійних відношень $\Theta\in\wt\calC(\calH)$, для яких $\{z:0\in\rho(\Theta - M(z))\}\ne\varnothing$, при цьому для кожного $g\in\frH$ вектор-функція $u_z=(A_\Theta-z)^{-1}g$, $z\in\rho(A_\Theta)$ є розв'язком граничної задачі
		\begin{equation*}
		 	(A^*-z)f = g, \qquad \{\Gamma_0f,\Gamma_1f\}\in\Theta.
		\end{equation*} 
	\end{enumerate}
\end{theorem}
% \label{sec:section_name}

% section section_name (end) 
%!TEX root = ../Masters.tex
\section{КАНОНІЧНІ СИСТЕМИ}

\subsection{Основні поняття}

Канонічні системи представляють великий математичний інтерес, оскільки вони в точному розумінні є найбільш загальним класом симетричних операторів другого порядку.\cite{Remling2018}

Канонічною системою називається диференціальне рівняння у формі

\begin{equation} \label{Canonical_sys}
	Ju'(x) = -zH(x)u(x), \quad 
	J=
	\begin{pmatrix}
		0 & -1\\
		1 & 0
	\end{pmatrix}.
\end{equation}

Ця система розглядатиметься на відкритому, можливо нескінченному, інтервалі $x\in (a,b)$, $-\infty\le a<b \le +\infty$, в якій для матриці коефіцієнтів $H$ виконуються наступні умови:
\begin{enumerate} 
	\item $H(x) \in \bbR^{2\times 2}$;
	\item $H \in L^1_{loc}(a,b)$; \label{CS-en2}
	\item $H(x)\ge 0 \text{ майже для всіх } x\in(a,b)$; \label{CS-en3}
	\item $H(x)\ne 0 \text{ майже для всіх } x\in(a,b)$. \label{CS-en4}
\end{enumerate}

Умова \eqref{CS-en2} означає, що елементи $H$ є локально інтегровними функціями, а умова \eqref{CS-en3} означає, що майже для всіх $x$ матриця $H(x)$ є симетричною і $v^*H(x)v\ge 0$ для всіх $v\in\bbC^2$.

Умова \eqref{CS-en4} також є влучною, оскільки якби існував би інтервал $(c,d)$, на якому $H = 0$ майже скрізь, то розв'язки просто залишалися б постійними на $(c, d)$, а вилучення інтервалу не впливало б на доповнення.

Параметр $z\in\bbC$ з \eqref{Canonical_sys} іноді називають спектральним параметром.

Диференціальне рівняння \eqref{Canonical_sys} має загальну структуру задачі про власне значення. А саме, якщо диференціальний оператор $\tau$ діє у $\bbC^2$ за правилом 
\begin{equation*}
 	(\tau u)(x) = -H^{-1}(x)Ju'(x),
\end{equation*}
тоді \eqref{Canonical_sys} потребує, щоб виконувалось $\tau u=zu$. Звичайно, щоб стверджувати це, необхідно визначити які умови роблять такий оператор самоспряженим, в яких гільбертових просторах діє такий оператор та що робити, якщо $H(x)$ не має оберненої матриці.

Оскільки $H(x)$ може бути нерегулярною, то потрібна деяка інтерпретація \eqref{Canonical_sys}.

\begin{definition}
	Функція $u:(a,b)\to \bbC^2$ називається локально (абсолютно) неперервною, якщо
	\begin{equation} \label{eq-AC}
	 	u(x) = u(c) + \int_c^x{f(t)\,dt}
	\end{equation}
	для деякої локально інтегровної функції $f:(a,b)\to \bbC^2$ і $c\in(a,b)$. Клас таких функцій позначається $AC(a,b)$.
\end{definition}

Якщо рівність~\eqref{eq-AC} виконується для деякого $c$, то вона виконується і для всіх $c\in(a,b)$.

Функція $u:(a,b)\to\bbC^2$ буде називатися розв'язком \eqref{Canonical_sys}, якщо $u\in AC$ і \eqref{Canonical_sys} справджується майже всюди. Також, якщо $H_1(x) = H_2(x)$ майже всюди, то два рівняння будуть мати однакові у тому ж сенсі розв'язки.

Існування та єдиність такого розв'язку системи \eqref{Canonical_sys} показує наступна теорема, сформульована для загального, неоднорідного, випадку.

\begin{theorem}\label{Th-CS-1}
	Нехай $c\in(a,b)$ і $f:(a,b)\to \bbC^2$ є локально інтегровною. Тоді для будь-якого $v\in\bbC^2$ задача
	\begin{equation}\label{eq-inhom-can-sys}
		Ju'=-zHu+f,\quad u(c) = v
	\end{equation}
	має єдиний розв'язок $u=u(x,z)$ на $x\in(a,b)$.

	Більше того, $u(x,z)$ є 
	% TODO "спільно неперервними" " jointly continuous" что это вообще такое?
	спільно неперервними на $(x,z)\in (a,b)\times\bbC$. Для кожного фіксованого $x\in(a,b)$ компоненти $u(x,z)$ є цілими функціями $z\in\bbC$. Похідні $u_n(x, z): = \partial^n u(x,z)/\partial{z^n}$ самі по собі є абсолютно неперервними функціями на $x\in(a,b)$, і вони формально рзв'язують початкові задачі, що виникають в результаті диференціювання \eqref{eq-inhom-can-sys} відносно $z$, як
	\begin{align*}
		Ju'_1 &= -zHu_1 - Hu_0, \ u_1(c)=0,\\
		Ju'_2 &= -zHu_2 - 2Hu_1, \ u_2(c)=0,\\
		&\dots \\
		Ju'_n &= -zHu_n - nHu_{n-1}, \ u_n(c)=0.
	\end{align*}
\end{theorem}

Якщо ж $f=0$, то твердження про єдиність передбачає, що множина розв'язків $u$ рівняння \eqref{Canonical_sys} є двовимірним векторним простором.

Матрицею переходу $T$ називається матричний розв'язок, що приймає значення в $\bbC^{2\times 2}$ однорідного рівняння
\begin{equation}\label{eq-Canonycal_sys_matrix}
	JT'=-zHT
\end{equation}
з початковою умовою $T(c) = I$. Таким чином, $T$ залежить від $x,c \in (a, b)$ і $z \in\bbC$ і позначається $T(x,c;z)$. При цьому, якщо другий аргумент відсутній, то вважається, що $c=0$.

\begin{theorem}\label{th-CS-2}
	Зафіксуємо $x,c\in(a,b)$, $x\ge c$. Тоді матриця переходу $T(z) = T(x,c;z)$ має наступні властивості як функція від $z\in\bbC$:
	\begin{enumerate}
		\item $T(z)$ --- ціла; \label{i-1-th-2}
		\item $T(0) = I$  і $\det{T(x,z)} = 1$ \label{i-2-th-2}
		\item $T(x,z) = T(x,c;z)T(c,z)$, де $T(x,z) = T(x,0;z)$; \label{i-3-th-2}
		\item Якщо $\Im z\ge 0$, то \label{i-4-th-2}
		\begin{equation}\label{eq-1-th-2}
			i(T^*(z)JT(z)-J)\ge 0.
		\end{equation}
	\end{enumerate}
\end{theorem}
\begin{proof}
	За другою частиною Теореми \ref{Th-CS-1}, $T$ є цілим в $z$ для фіксованих $x$ і $c$. Стовпці $T$ розв'язують \eqref{Canonical_sys} як функції від $x$. Загалом, якщо $v\in\bbC^2$, то унікальний розв'язок з початковою умовою $u (c) = v$ задається як $u (x) = T (x, c; z) v$. Отже, як випливає з назви, матриця переходу оновлює значення розв'язків, а точніше, $T (x, c; z)$ пробігає значення аргументу від $c$ до $x$. Ця властивість характеризує матрицю переходу $T (x, c; z)$.

	Оскільки $J^{-1}=-J$, маємо, що $T'= zJHT$, а матриця $JH$ має нульовий слід. Ця властивість еквівалентна тому, що $H$ симетрична. Звідси випливає, що $\det T(x)$ є постійним і, як початкова умова, $\det T = 1$, тому це справедливо для всіх $x \in (a,b)$. Зокрема, $T$ має обернену, а $T^{-1}$ нейтралізує дію $T$ і, оскільки $T$ приймає значення, пробігаючи від $c$ до $x$, то ${T(x,c;z)}^{-1} = T(c,x;z)$. Отже, $T$ є абсолютно неперервною функцією свого другого аргументу.

	Зрозуміло, що $T(0)=I$, оскільки рівняння стає $T'=0$ для $z=0$. Якщо $z=t$ є дійсними, то $T(x,c;t)$ є розв'язком задачі Коші з дійсними коефіцієнтами та дійсними початковими умовами, тому приймає дійсні значення.

	Залишилося встановити \eqref{eq-1-th-2}. Насамперед, оскільки $J^*=-J$, то спряжене від $JT'=-zHT$ дає $-T^{*}{'} J=-\bar{z}T^* H$. Позначимо $z=t+iy$, $y\ge 0$, і розглянемо
	\begin{equation*}
		\frac{d}{dx}(T^*(x,c;z)JT(x,c;z)) = (\bar{z}-z)T^*HT=-2iyT^*HT.
	\end{equation*}
	Інтегрування цього рівняння показує, що ліва частина \eqref{eq-1-th-2} дорівнює $2y\int_c^x T^*HT\,ds$ і є додатно визначена матриця, як було заявлено.
\end{proof}

Якщо задана матрична функція $T(z)$ має властивості, зазначені в теоремі, то на інтервалі $(c,x)$ буде канонічна система. Іншими словами, на такому проміжку буде функція коефіцієнтів $H$ така, що $T(z) = T(x,c;z)$. Більше того, після відповідної нормалізації, канонічна система однозначно визначається $T(z)$. Варіант розрахунку, що був використаний в останньому доведенні, дає ще одну корисну тотожність.

\begin{theorem}[Сталість Вронскіану]
	\begin{equation*}
		T^T(x)JT(x)=J,\quad T(x)\equiv T(x,c;z).
	\end{equation*}
	Як наслідок, вронскіан $W(v,w)\equiv v^T(x)Jw(x)$ є сталим для будь-яких двох розв'язків $v$, $w$ рівняння $Ju'=-zHu$.
\end{theorem}
\begin{proof}
	Перша тотожність випливає з
	\begin{equation*}
		(T^{T}JT)'=-(JT')^{T}T+T^{T}JT'=z(HT)^{T}T-zT^{T}HT=0.
	\end{equation*}

	Аналогічно для останнього виразу:
	\begin{equation*}
		(v^T(x)Jw(x))'= -(Jv'(x))^T + v^T(x)Jw'(x) = zv^T(x)Hw(x) - zv^T(x)Hw(x) = 0.
	\end{equation*}
\end{proof}

\subsection{Сингулярні інтервали}

\begin{definition}\label{def-sing-point&int}
	Точка $x\in(a,b)$ називається сингулярною, якщо знайдуться $\delta>0$ і вектор $v\in\bbR^2$, $v\ne0$ такі, що $H(t)v=0$ майже для всіх $|t-x|<\delta$. В іншому випадку точка $x\in(a,b)$ називається регулярною.

	Множина $S$ сингулярних точок є відкритою, а її компоненти зв'язності $(c,d)$ називаються сингулярними інтервалами. 
\end{definition}

Відкритість множини $S$ одразу випливає з означення. Можна записати $S=\bigcup(c_j,d_j)$ як зліченне об'єднання відкритих інтервалів, що не перетинаються, які є сингулярними інтервалами, визначеними у \ref{def-sing-point&int}.

Нехай тепер $x\in(a,b)$ --- сингулярна точка. Оскільки $H\ge0$ і $H\ne0$ майже скрізь на $(x-\delta,x+\delta)$, ця функція може бути представлена у вигляді $H(t)=h(t)P_\alpha$ на цьому інтервалі майже скрізь для деякого $\alpha\in[0,\pi)$ та деякої функції $h\in L^1(x-\delta,x+\delta)$, $h>0$, де
\begin{equation}\label{eq-si-1}
	P_\alpha = 
	\begin{pmatrix}
		\cos^2{\alpha} & \sin{\alpha}\cos{\alpha} \\
		\sin{\alpha}\cos{\alpha} & \sin^2{\alpha}
	\end{pmatrix}
	= e_\alpha e^*_\alpha, \quad e_\alpha = \binom{\cos{\alpha}}{\sin{\alpha}}
\end{equation}
позначає проекцію на $e_\alpha$. Ті ж зауваження стосуються всього сингулярного інтервалу, якому належить $x$. 

\begin{definition}
	Кут $\alpha$ називається типом сингулярного інтервалу $(c,d)$.
\end{definition}

% Іноді для зручності вектор $e_\alpha$ також називається типом сингулярного інтервалу.

Розв'язання \eqref{Canonical_sys} через сингулярний інтервал $(c,d)$ типу $\alpha$ дає більше розуміння сенсу Означення \ref{def-sing-point&int}. Тож нехай $H(x)=h(x)P_\alpha \equiv h(x)P$ на $(c,d)$. Оскільки лише скалярна $h$ залежить від $x$, то будь-які дві матриці $JH(x)$, $JH(x')$ є комутативними, тому рівняння $u'=zJHu$ має розв'язок
\begin{equation*}
	u(x) = e^{z\left(\int\limits_c^x h(t)\,dt\right)JP}u(c).
\end{equation*}
Розкладемо останнє за степенями. Отримаємо, що $(JP)^2=JPJP=0$, оскільки $J$ діє як обертання на 90 градусів, що дає $PJP=0$. Таким чином, експоненціальний ряд для $u(x)$ закінчується після перших двох доданків:
\begin{equation*}
	u(x) = \left(1+z\left(\int\limits_c^x h(t)\,dt\right)JP\right)u(c).
\end{equation*}
Зокрема, $u(d)=(1+zJH)u(c)$ з $H=\int_c^d H(x)\,dx$, але це теж саме, що і зробити один крок рекурсії в у різницевому рівнянні
\begin{equation}\label{eq-Canon_sys_dif}
 	J(u_{n+1} - u_n) = -zH_nu_n,
\end{equation}
аналогічному \eqref{Canonical_sys}, якщо $H_n=H$.

З більш абстрактної точки зору, властивістю матриці переходу $T=I+zJH$ через сингулярний інтервал є її поліноміальна залежність від $z$, причому ступеня 1. Це випливає з \eqref{eq-Canon_sys_dif} тоді, як диференціальні рівняння зазвичай призводять до складніших функцій $z$.

Отже, канонічна система через сингулярний інтервал імітує різницеве рівняння. Здатність канонічної системи робити це має вирішальне значення. Вже було згадано результат, що будь-які спектральні дані можуть бути реалізовані канонічною системою, і ці спектральні дані можуть виходити з різницевого рівняння, тому є необхідність варіанту їх моделювання.

Сингулярні інтервали також використовуються для реалізації граничних умов, і вони відповідають за багатозначну частину відношень, які асоціюються з канонічними системами.

\subsection{Гільбертів простір $L^2$}

У наступному розділі буде розглянуто самоспряжені реалізації \eqref{Canonical_sys} та їх спектральну теорію і з'явиться необхідність у гільбертовому просторі, в якому вони діють, і відповідним простором для цього є $L^2_H(a,b)$, визначений наступним чином. Припустимо, що $H(x)$ задовольняє умови для матриці коефіцієнтів канонічної системи. Нехай
\begin{equation*}
 	\calL = \left\{f:(a,b)\to \bbC^2: f \text{ є вимірною }, \int\limits_a^b f^*(x)H(x)f(x)\,dx <\infty \right\},
\end{equation*} 
тоді
\begin{equation}
 	||f|| = \left(\int\limits_a^b f^*Hf\,dx\right)^{1/2}, \quad f\in\calL,
\end{equation}
і $L_H^2(a,b)$ визначається як $\calL/\calN$, де $\calN=\{f\in\calL: ||f||=0\}$.

Це є звичайною процедурою визначення просторів $L^p$, за винятком, того, що функції приймають значення в $\bbC^2$, а не в $\bbC$. $L^2_H$ --- сепарабельний, нескінченновимірний простір Гільберта. Насправді, відображення
\begin{equation*}
	V:L_H^2(a,b)\to L^2_I(a,b), \quad (Vf)(x) = H^{1/2}(x)f(x),
\end{equation*}
де $H^{1/2}(x)$ визначено як єдиний додатний квадратний корінь з $H(x)$, що забезпечує вкладання $L_H^2$ в $L_I^2$, де $I$ --- одинична матриця $2\times2$. А, оскільки $L_I^2(a,b)\cong L^2(a,b)\oplus L^2(a,b)$, то всі питання можна звести до класичного простору $L^2$.

Оскільки $f^*Hf=(H^{1/2}f)^*H^{1/2}f$, то функції $f,g$ будуть являти собою один і той самий елемент в $L^2_H$ тоді і тільки тоді, коли $H(x)f(x) = H(x)g(x)$ майже для всіх $x$, але якщо $H(x)$ має ядро для деякого $x$ таке, що $H(x)v(x)=0$, $v(x)\ne0$, то це просто означає, що $f(x)-g(x)=c(x)v(x)$ для цього $x$. Зокрема, цілком можливо, що $f\in L^2_H$ має декілька неперервних представників, не рівних як функції.

Корисним є наступний факт.
\begin{lemma}
	Якщо $f\in L^2_H(a,b)$, то $Hf\in L^1_{loc}(a,b)$.
\end{lemma}
\begin{proof}
	Це випливає з нерівності Гельдера, якщо взяти додатний квадратний корінь $H^{1/2}(x)$ і записати $Hf=H^{1/2}(H^{1/2}f)$. Тоді обидва $H^{1/2}$ і $H^{1/2}f$ належать $L^2_{loc}(a,b)$ і їх добуток $Hf\in L^1_{loc}(a,b)$.
\end{proof}

\subsection{Мінімальні і максимальні відношення канонічних систем}

В цьому розділі буде розглянуто питання як канонічна система
\begin{equation*}
	Ju'(x) = -zH(x)u(x),\quad x\in(a,b)
\end{equation*}
генерує самоспряжені оператори у гільбертовому просторі $L^2_H(a,b)$. Ці оператори мають діяти як $-H^{-1}Jf'$ на функції $f$ з їх областей визначення і необхідно, щоб $H(x)$ мала обернену. Цієї проблеми можна уникнути, перемістивши $H$ назад. Припустимо, є пара $f,g\in L^2_H$ і $g=\tau f$ як результат застосування оператора над $f$, який необхідно побудувати. Формально це можна записати як
\begin{equation}\label{eq-relation-1}
	Jf'(x) = -H(x)g(x).
\end{equation}
Загалом ця умова визначатиме лінійні відношення, а не оператор.

Лінійні оператори $T$ стають {особливими} відношеннями після ототожнення їх зі своїми графіками $\{(x,Tx)\}$. І навпаки, відношення можна вважати операторами, за винятком того, що $f\in\calH$ може мати кілька зображень. Відношення $\calT$ є оператором, якщо $(f,g_1), (f,g_2)\in \calT$ означає, що $g_1=g_2$. За лінійністю це еквівалентно умові, що $(0,g)\in\calT$ тоді і тільки тоді, коли $g=0$.

Визначимо максимальне відношення $\calT$ канонічної системи як сукупність усіх пар $(f,g)$, для яких виконується \eqref{eq-relation-1}:
\begin{multline} \label{eq-relation-2}
	\calT=\{(f,g): f,g\in L^2_H(a,b), f \text{ має представника } f_0\in AC \text{ такого, що }\\ Jf'_0(x)=-H(x)g(x) \text{ майже для всіх } x\in(a,b)\}.
\end{multline}
Останнє чітко визначає лінійний підпростір, або відношення.

Як вже було показано, $f\in L^2_H$ може мати декілька неперервних представників, тому не можна реально очікувати, що $f_0$ однозначно визначається $f$. Тому особлива увага буде приділятися розрізненню елементів простору Гільберта (класів еквівалентності функцій) та функцій.

Хоча $f_0$ з \eqref{eq-relation-2} не має визначатися функцією $f$, але $f_0$ визначається парою $(f,g)$, якщо не розглядається тривіальний сценарій, де $(a,b)$ є лише одиничним сингулярним інтервалом. Отже, необхідне припущення: інтервал $(a,b)$ містить щонайменше одну регулярну точку. Це діятиме відтепер, якщо прямо не зазначено інше. Якщо $(a,b)$ --- єдиний сингулярний інтервал, то все можна опрацювати явно. Більше того, багато результатів потребують модифікації в цьому випадку, тому набагато зручніше просто виключати цей тривіальний сценарій з розвитку загальної теорії.

Повернемося до твердження, що якщо $(f,g)\in\calT$, то $f_0$ з \eqref{eq-relation-2} однозначно визначається. Дійсно, це показує інтегрування $Jf'_0=-Hg$: 
\begin{equation}
	f_0(x)=f_0(c) + J\int\limits_c^x H(t)g(t)\,dt.
\end{equation}
Слід зауважити, що тут $H(t)g(t)$ може змінюватися лише на нульовому наборі, якщо обрати іншого представника $g$, тому інтеграл визначається елементом гільбертового простору $g\in L^2_H$. Звідси випливає, що два таких представники $f$ могли б відрізнятись лише постійною функцією $v$, але тоді матимемо $H(x)v=0$ майже для кожного $x$, інакше вони б не представляли б один і той самий елемент гільбертового простору. Якщо ж $v\ne0$, тоді це означає, що $(a,b)$ --- сингулярний інтервал, який був виключений з розгляду.

\begin{lemma} \label{lemma-1-relations}
	Елемент $(f,g)\in\calT$ максимального відношення однозначно визначає абсолютно неперервну функцію $f_0:(a,b)\to\bbC^2$ з наступними двома додатковими властивостями: 
	\begin{enumerate}
		\item $f_0\in L^2_H(a,b)$, і він є представником елемента $f$;
		\item $Jf'_0=-Hg$.
	\end{enumerate}
\end{lemma}

Індекс у $f_0$ означає, що він є представником $f$, який визначається не лише самим $f$, але й парою $(f,g)\in\calT$.

Приклад, коли $f_0$ дійсно не визначається просто $f$, можна побачити, розглянувши випадок, де $(a,b)$ --- сингулярний інтервал. Покладемо $a=0$, $b = 1$, 
$H(x) =
\begin{pmatrix}
	1 & 0 \\ 0 & 0
\end{pmatrix}
$.

Тоді $(f,g)\in\calT$ тоді і тільки тоді, коли $f'_{0,2}=g_1$, $f'_{0,1}=0$. З точки зору простору Гільберта, важливий лише перший компонент функції. Таким чином, ми можемо прийняти будь-яку абсолютно неперервну функцію як $f_{0,2}$, і з цього випливає, що $g\in L^2_H(0,1)$ є довільною. Зрозуміло, що $f_{0,1}$ має бути постійною. Отже, позначаючи $e_1$ перший одиничний вектор, знаходиться наступне
\begin{equation}
	\calT = L(e_1)\oplus L^2_H(0,1) =\{(f,g): f(x) = ce_1, g\in L^2_H(0,1)\}. 
\end{equation}

Далі наведені необхідні визначення для відношень.

\begin{definition}
	Нехай $\calT\subseteq\calH\times\calH$ є відношенням. $\calT$ називається замкненим, якщо $\calT$ є замкненим підпростором $\calH\times\calH$. Замиканням $\overline{\calT}$ відношення $\calT$ називається замикання підпростору $\calT$.

	Області визначення та значень, ядро і багатозначна частина відношення $\calT$ визначаються наступним чином:
	\begin{align*}
	 	\dom \calT &= \{f\in\calH:(f,g)\in\calT \text{ для деякого } g\} &
	 	\ker \calT &= \{f\in\calH:(f,0)\in\calT \}\\
	 	\ran \calT &= \{g\in\calH:(f,g)\in\calT \text{ для деякого } f\} &
	 	\mul \calT &= \{g\in\calH:(0,g)\in\calT \}.
	\end{align*}

	Оберненим до $\calT$ є відношення
	\begin{equation*}
		\calT^{-1} = \{(g,f):(f,g)\in\calT\},
	\end{equation*}
	а спряжене відношення до $\calT$ визначається як
	\begin{equation*}
		\calT^* = \{(h,k):\left<h,g\right>=\left<k,f\right> \text{ для всіх } (f,g)\in\calT\}.
	\end{equation*}
	Відношення $\calT$ називається симетричним $\calT\subseteq \calT^*$, якщо, і самоспряженим, якщо $\calT=\calT^*$.
\end{definition}

Тож, на відміну від операторів, відношення завжди мають замикання, обернені до них та унікальні спряження.

Для того, щоб знайти спряження $\calT_0:=\calT^*$ максимального відношення $\calT$ канонічної системи, необхідно визначити предмінімальне відношення:
\begin{equation*}
	\calT_{00} = \{(f,g)\in\calT:f_0(x) \text{ має компактний носій на } (a,b) \}.
\end{equation*}

\begin{definition}
	Замикання $\calT_0=\overline{\calT_{00}}$ лінійного відношення $\calT_{00}$ називають мінімальним.
\end{definition}

Неважко помітити, що $\calT_{00}\subseteq\calT^*$. Дійсно, для фіксованого $(f,k)\in\calT_{00}$ і довільного $(h,k)\in\calT$ покажемо, що $\left<f,k\right>=\left<g,h\right>$, або
\begin{equation*}
	\int\limits_a^b f^*(x)H(x)k(x)\,dx = \int\limits_a^b g^*(x)H(x)h(x)\,dx.
\end{equation*}
Підключаючи до цього $Hk=-Jh'_0$, $Hg=-Jf'_0$, отримаємо
\begin{equation*}
	\int\limits_a^b \left(f^*_0(x)Jh'_0(x)+f^{*}{'}_0(x)Jh_0(x)\right)\,dx = 0,
\end{equation*}
що є очевидним, оскільки $f_0$ є нулем близько до $a$ і $b$.

\begin{proposition} \label{prop-1-relations}
	$\calT^*_{00}\subseteq\calT$
\end{proposition}
\begin{proof}
	Нехай $(f,g)\in\calT^*_{00}$ і функція $f_1$ визначається як
	\begin{equation*}
		f_1(x) = J\int\limits_c^x H(t)g(t)\,dt,
	\end{equation*}
	для деякого фіксованого $c\in(a,b)$. Тоді $f_1$ є абсолютно неперервною і $Jf'_1=-Hg$, але її квадрат не обов'язково є інтегрованим, тобто $f_1$ може не бути елементом гільбертового простору.

	Нехай $(h,k)\in\calT_{00}$ є довільним. Інтегрування за частинами показує, що
	\begin{equation*}
		\left<h,g\right> = \int\limits_a^b h^*_0(x)H(x)g(x)\,dx = \int\limits_a^b h^*_0(x)Jf'_1(x)\,dx = \int\limits_a^b k^*(x)H(x)f_1(x)\,dx.
	\end{equation*}
	Функції $h_0$ і $Hk$ мають компактний носій, тому факт, що $f_1$ може не лежати в $L^2_H$, не може зробить останній інтеграл розбіжним. З цієї ж причини інтегрування частинами не вносить граничні умови. 

	З іншого боку, $\left<h,g\right>=\left<k,f\right>=\int_a^b k^*Hf$, тому
	\begin{equation} \label{eq-1-prop1}
		\int\limits_a^b k^*(x)H(x)(f_1(x)-f(x))\,dx = 0 \quad \forall k\in \ran(\calT_{00}).
	\end{equation}
	Потрібно звернути увагу, що $k\in L^2_H(a,b)$ буде точно в $\ran \calT_{00}$, якщо він задовольняє наступним двом умовам: (1) $Hk$ має компактний носій; (2) $\int_a^b HK = 0$. Оскільки $Hk$ локально інтегрується та має компактний носій, останній інтеграл є визначеним. Позначимо через $X$ лінійний підпростір $L^2_H$, визначений умовою (1), і розглянемо на $X$ функціонали
	\begin{equation*}
		F_j(k) = e^*_j \int\limits_a^b H(x)k(x)\,dx, \quad F(k) = \int\limits_a^b (f_1(x)-f(x))^*H(x)k(x)\,dx.
	\end{equation*}
	Тепер \ref{eq-1-prop1} можна перефразувати як твердження, що якщо $F_1(k)=F_2(k)=0$ для $k\in X$, то $F(k)=0$. Тоді $F$ має бути лінійною комбінацією $F_1$ і $F_2$. Отже, існує вектор $v\in\bbC$ такий, що 
	\begin{equation*}
		\int\limits_a^b (f_1(x)-f(x)-v)^*H(x)K(x)\,dx = 0
	\end{equation*}
	для всіх $k\in X$. Оскільки $f_1(x)-f(x)-v$ локально в $L^2_H$, це можливо лише в тому випадку, коли $H(x)(f_1(x)-f(x)-v)=0$ майже скрізь. Отже, $f$ має абсолютно неперервного представника $f_1(x)-v$, а $J(f_1-v)'=-Hg$ за побудовою $f_1$. Це говорить про те, що $(f,g)\in\calT$, що доводить пропозицію.
\end{proof}

\begin{proposition} \label{prop-2-relations}
	Нехай $\calT\subseteq\calH\times\calH$ є відношенням. Тоді:
	\begin{enumerate}
		\item $\calT^*$ є замкненим;
		\item $\calT^{**}=\overline{\calT}$;
		\item $\overline{\calT}^* = \calT^*$
	\end{enumerate}
\end{proposition}

\begin{theorem}
	\begin{enumerate}
		\
		\item Максимальне відношення $\calT$ є замкненим;
		\item Мінімальне відношення $\calT^*_0=\calT$ є замкненим і симетричним, і $\calT^*_0=\calT$.
	\end{enumerate}
\end{theorem}
\begin{proof}
	1. Припустимо, що $(f_n,g_n)\in\calT$, $(f_n,g_n)\to (f,g)\in\calH\oplus\calH$. Необхідно довести, що $(f,g)\in\calT$.

	Переходячи до підпослідовності, можна припустити, що $H(x)f_{n,0}(x)\to H(x)f(x)$ поточково майже скрізь для представників із Леми~\ref{lemma-1-relations}. Також для кожного фіксованого $x\in(a,b)$ послідовність $f_{n,0}(x)$ повинна бути обмежена. Це випливає з того, що похідні $f'_{n,0}=JHg_n$ обмежені в $L^1(c,d)$ для будь-якої компактної підмножини $[c,d]\subseteq(a,b)$, тому, якщо б $|e\cdot f_{n,0}(x)|$ були великими для деякого напрямку $e\in\bbC^2$, $||e||=1$, то те саме було б справедливим для будь-якої компактної підмножини $(a,b)$, але це зробило б норму $f_n$ великою, оскільки $(a,b)$ не є a сингулярним інтервалом і, таким чином, $H(x)$ не може занулити $e$ скрізь.

	Тож можна обрати підпослідовність таку, що в додаток $f_{n,0}(c)\to v$ для фіксованого $c\in(a,b)$, що був обраний заздалегідь. Тепер можна просто перейти до поточкової границі в 
	\begin{equation*}
		f_{n,0}(x) = f_{n,0}(c) + J\int\limits_c^x H(t)g_n(t)\,dt.
	\end{equation*}
	Видно, що $f_{n,0}(x)$ сам збігається (не тільки після застосування $H(x)$), і його границя буде представляти f. Отже, був знайдений абсолютно неперервний представник $f$, який задовольняє $Jf'=-Hg$, отже $(f,g)\in\calT$.

	2. Зрозуміло, що спряжений оператор $\calT_0$ є замкненим, а симетрія випливатиме з двох заявлених рівностей, тому їх достатньо довести. Раніше було показано, що $\calT_{00}\subseteq\calT^*$ і $\calT^*_{00}\subseteq\calT$ (Пропозиція~\ref{prop-1-relations}), а спряження другого включення дає це $\calT^{**}_{00}\supseteq\calT^*$. А якщо взяти замикання першого включення і використати Пропозицію~\ref{prop-2-relations}, то отримаємо $\overline{\calT_{00}} = \calT^* = \calT_0$. Ще одне спряження дає останню рівність.
\end{proof}

Можна дати точніший опис мінімального відношення $\calT_0$. Принаймні, як частина результату: $\calT_0$ можна отримати, взявши замикання $\calT_{00}$, яке було визначене як ті елементи максимального відношення, для яких $f_0$ має компактний носій.

\begin{definition}\label{def-reg-endpoint}
	Кінцеву точку $a$ називають регулярною, якщо $H\in L^1(a,c)$ для деяких (і тоді всіх) $c\in(a,b)$, і аналогічно для $b$.

	Тут точки $a=-\infty$ і $b=\infty$ можуть бути звичайними кінцевими точками.
\end{definition}

Визначення~\ref{def-reg-endpoint} дає Лему~\ref{lemma2-relations}.

\begin{lemma} \label{lemma2-relations}
	Якщо $a$ є регулярною, то для будь-яких $(f,g)\in\calT$ представлення $f_0$ має неперервне продовження на $[a,b)$, і $f_0\in AC[a,b)$. Більше того, розв'язки однорідного рівняння $Ju'=-zHu$ мають ті ж самі властивості.

	Для регулярної точки $b$ результати аналогічні.
\end{lemma}

Вже відомо, що $f_0\in AC(a,b)$ і це означає, що $f_0(x)=f_0(c) + \int_c^x h(t)\,dt$ для деякої $h\in L^1_{loc}(a,b)$. Твердження, що $f_0\in AC[a,b)$, створює додаткове твердження, що $h\in L^1(a,c)$ для $c\in(a,b)$. Це означає, що $f_0$ має неперервне продовження до $x=a$, але не випливає з цієї властивості.

Ніякі зміни цих тверджень не потрібні у випадку $a=-\infty$, якщо дати розширеному інтервалу $[a,c)=[-\infty,c)$ його очевидну топологію.

\begin{proof}
	Нерівність Коші-Шварца показує, що для будь-якого $g\in L^2_H(a,c)$, маємо, що $Hg=H^{1/2}H^{1/2}g\in L^1(a,c)$, тож твердження для $f_0$ випливають з
	\begin{equation*}
	 	f_0(x) = f_0(c) + J\int\limits_c^x H(t)g(t)\,dt.
	 \end{equation*}
	Як і для розв'язків $u$ рівняння $Ju'=-zHu$, застосуємо теорію звичайних диференціальних рівнянь, узагальнену в Теоремі~\ref{Th-CS-1}, до початкової задачі значення $u(a)=v$ для загального $v\in\bbC^2$, щоб підтвердити, що $u$ є абсолютно неперервними на $[a,b)$.

	Початкова задача $Ju'=-zHu$, $u(-\infty)=v$ може бути записана як інтегральне рівняння $u(x)=v+zJ\int_{-\infty}^x H(t)u(t)\,dt$, і, якщо $H\in L^1(-\infty,c)$, то це можна розв'язати так само, як на обмеженому проміжку, за допомогою ітерації Пікарда. Або, можна провести перетворення, щоб зробити $a$ кінцевою точкою $A\in\bbR$, щоб взагалі уникнути цих питань.
\end{proof}

\begin{lemma}\label{lemma-7-relations}
	Нехай $(c,d)\subseteq(a,b)$, і жоден з $(a,c)$, $(d,b)$ не є порожнім інтервалом, що міститься в одному сингулярному інтервалі. Нехай $(h,k)\in\calT_{(c,d)}$. Тоді існує $(f,g)\in\calT$ з $f_0=h_0$ на $(c,d)$, $f_0(x)=0$ для $x\in(a,c)$ близько до $a$ і $x\in(d,b)$ близько до $b$.
\end{lemma}
\begin{proof}
	Нехай $d<b$. Оскільки $d$ є регулярною кінцевою точкою $(c,d)$, застосовується Лема~\ref{lemma2-relations}, тоді $h_0$ абсолютно неперервна на $(c,d]$. Потрібно знайти абсолютно неперервну функцію $f_0$ на $[d,b)$ таку, що $Jf'_0=-Hg$ для деяких $g\in L^2_H(d,b)$ і $f_0(d)=h_0(d)$, щоб зробити функцію абсолютно неперервною при з'єднанні двох частин. Також бажано, щоб $f_0(x)=0$ для всіх великих $x$, і достатньо одного разу досягти цього значення, адже з цього моменту можна застосувати нульову функцію. Отже, щоб $f_0(d)=h_0(d)$ і $f_0(t)=0$, оберемо $t\in(d,b)$ настільки великим, щоб $(d,t)$ не міститься в сингулярному інтервалі. Якщо все це сказати для $g$, то тепер потрібно знайти $g\in L^2_H(d,t)$ таке, що функція $f_0$ визначається як
	\begin{equation*}
		f_0 = h_0(d) +J\int\limits_d^x H(s)g(s)\,ds
	\end{equation*}
	і задовольняє умові $f_0(t)=0$. Це працює, якщо лінійне відображення
	\begin{equation*}
		F:L^2_H(d,t)\to\bbC^2, \quad F(g) = \int\limits_d^t H(s)g(s)\,ds
	\end{equation*}
	є сюр'єктивним і легко зрозуміти, що це буде в тому випадку, якщо $(d,t)$ не міститься в сингулярному інтервалі, тому що діапазон $H(x)$ не може бути однаково рівним фіксованому одновимірному підпростору $\bbC^2$.

	Нарешті, якщо також $c>a$, то застосовуємо ту саму процедуру зліва від $(c,d)$.
\end{proof}

\begin{theorem}\label{th-relation-1}
	Нехай $(f,g), (h,k)\in \calT$. Тоді для $f^*_0(x)Jh_0(x)$ існують границі при $x\to a+$ і $x\to b-$. Більше того,
	\begin{equation}\label{eq-relation-3}
	 	\left<g,h\right> - \left<f,k\right> =\Bigl. f^*_0Jh_0\,\Bigl|_a^b.
	\end{equation}
\end{theorem}

Для цих меж будуть використані позначення $(f^*_0Jh_0)(a)$ і $(f^*_0Jh_0)(b)$. Якщо кінцева точка (скажімо, $a$) є регулярною, то існування їх стає безпосереднім наслідком Леми~\ref{lemma2-relations}, і в цьому випадку $(f^*_0Jh_0)(a) = f^*_0(a)Jh_0(a)$. Тут використовуються наводить позначення $f_0(a)$, $h_0(a)$ для неперервних продовжень цих функцій до $x=a$.

\begin{proof}
	Обидва твердження випливають із наступного розрахунку:
	\begin{align*}
		\left<g,h\right> - \left<f,k\right> &= \lim_{\substack{\alpha\to a+\\ \beta\to b-}} \int\limits_\alpha^\beta (g^*(x)H(x)h_0(x) - f^*_0(x)H(x)k(x))\,dx \\
		&= \lim_{\substack{\alpha\to a+\\ \beta\to b-}} \int\limits_\alpha^\beta (f_0^{*}{'}(x)Jh_0(x) - f^*_0(x)Jh'_0(x))\,dx \\
		&= \lim_{\substack{\alpha\to a+\\ \beta\to b-}} \Bigl. f^*_0Jh_0\Bigr|_\alpha^\beta.
	\end{align*}
\end{proof}
%!TEX root = ../Masters.tex
\section{ГРАНИЧНІ ТРІЙКИ ДЛЯ КАНОНІЧНИХ СИСТЕМ}

\subsection{Граничні трійки для канонічних систем в регулярному випадку}

Нехай гамільтоніан $H$ задовольняє умові 
\begin{equation}\label{eq-4.1}
	H \in L^1(a,b).
\end{equation}
Тоді за Теоремою~\ref{th-relation-1} для будь-якої пари $\{f,g\}\in\calT$ і представника $f_0\in AC$ існують границі
\begin{equation}\label{eq-4.2}
	f_0(a+) = \lim_{x\downarrow a} f_0(x), \quad f_0(b-) = \lim_{x\uparrow a} f_0(x)
\end{equation}

\begin{theorem}\label{th-4.1}
	Нехай виконано умову \eqref{eq-4.1} і відображення $\Gamma_0, \Gamma_1:\calT\to\bbC^2$ задано рівностями
	\begin{equation} \label{eq-4.3}
		\Gamma_0f = 
		\begin{pmatrix}
			f_{01}(a) \\ f_{01}(b)
		\end{pmatrix},\quad
		\Gamma_1f = 
		\begin{pmatrix}
			-f_{02}(a) \\ f_{02}(b)
		\end{pmatrix}\quad
		(f,g)\in\calT,\ f_0\in f.
	\end{equation}
	Тоді сукупність $(\bbC^2,\Gamma_0,\Gamma_1)$ утворює граничну трійку для $\calT$.
\end{theorem}
\begin{proof}
	В силу Теореми~\ref{Th-CS-1} існують функції $\wt{u},\ \wt{v} \in\dom{\calT}$ такі, що 
	\begin{align}\label{eq-4.4}
	 	\wt{u}_{01}(a)&=1 & \wt{v}_{01}(a)&=0 \\
	 	\wt{u}_{02}(a)&=0 & \wt{v}_{02}(a)&=1.
	 \end{align}
	В силу Леми~\ref{lemma-7-relations} функції $\wt{u},\ \wt{v}$ можна змінити на інтервалі $(c,d)$ так, що $u_0$ і $v_0$ перетворюються на нуль на інтервалі $(c,d)$.

	Аналогічно, в силу Теореми~\ref{Th-CS-1} існують функції $h,\ k \in\dom{\calT}$ такі, що
	\begin{align}\label{eq-4.5}
	 	h_{01}(b)&=1 & k_{01}(a)&=0 \\
	 	h_{02}(b)&=0 & k_{02}(a)&=1.
	 \end{align}
	Користуючись Лемою~\ref{lemma-7-relations}, змінимо функції $h_0,\ k_0$ біля точки $a$ так, що $h_0$ і $k_0$ перетворюються на нуль в околі точки $a$. З \eqref{eq-4.4} і \eqref{eq-4.5} випливає, що відображення
	$
		\Gamma= 
		\begin{pmatrix}
			\Gamma_0 \\ \Gamma_1
		\end{pmatrix}
		:\calT\to\bbC^2\times\bbC^2
	$
	є сюр'єктивним.

	Тотожність \eqref{eq-CS-Green} випливає з \eqref{eq-relation-3}.
\end{proof}

У подальшому ми вводимо до розгляду матрицю $W(x,\lambda) = T(x,\lambda)^{T}$ і її блочной вигляд
\begin{equation}
 	W(x,\lambda)= 
 	\begin{pmatrix}
 		w_{11}(x,\lambda) & w_{12}(x,\lambda)\\
 		w_{21}(x,\lambda) & w_{22}(x,\lambda)
 	\end{pmatrix}
 \end{equation} 

\begin{theorem}\label{th-4.2}
	Функція Вейля, що відповідає граничній трійці \eqref{eq-4.3} має вигляд
	\begin{equation}\label{eq-4.7}
		M(b,\lambda)=
		\begin{pmatrix}
			-w_{11}(b,\lambda)w_{12}(b,\lambda)^{-1} & w_{12}(b,\lambda)^{-1} \\
			w_{21}(b,\lambda) - w_{22}(b,\lambda)w_{12}(b,\lambda)^{-1}w_{11}(b,\lambda) & w_{22}(b,\lambda)w_{12}(b,\lambda)^{-1}
		\end{pmatrix},
	\end{equation}
	а відповідне $\gamma$-поле має вигляд
	\begin{equation}\label{eq-4.8}
		\gamma(\lambda) = W(\cdot,\lambda)
		\begin{pmatrix}
			1 & 0 \\
			-w_{11}(b,\lambda)w_{12}(b,\lambda)^{-1} & w_{12}(b,\lambda)^{-1}
		\end{pmatrix}
	\end{equation}
\end{theorem}
\begin{proof}
	Дефектний простір $\frN_\lambda(T_0)$ складається з функцій
	\begin{equation}\label{eq-4.9}
		f(\cdot,\lambda) = W(\cdot,\lambda)
		\begin{pmatrix}
			\alpha_1 \\ \alpha_2
		\end{pmatrix}, \quad \text{ де } \alpha_1,\alpha_2\in\bbC.
	\end{equation}
	Застосовуючи оператори $\Gamma_0$, $\Gamma_1$ до $f(\cdot,\lambda)$, отримаємо
	\begin{equation*}
		\Gamma_0\widehat{f}(\cdot,\lambda) = \Phi_0(\lambda)\begin{pmatrix}
			\alpha_1 \\ \alpha_2
		\end{pmatrix}, \quad
		\Phi_0(\lambda) =
		\begin{pmatrix}
			1 & 0 \\
			w_{11}(b,\lambda) & w_{12}(b,\lambda)
		\end{pmatrix},
	\end{equation*}
	\begin{equation*}
		\Gamma_1 f(\cdot,\lambda) = \Phi_1(\lambda)\begin{pmatrix}
			\alpha_1 \\ \alpha_2
		\end{pmatrix}, \quad
		\Phi_1(\lambda) =
		\begin{pmatrix}
			0 & -1 \\
			w_{21}(b,\lambda) & w_{22}(b,\lambda)
		\end{pmatrix}.
	\end{equation*}
	Звідси отримаємо
	\begin{equation*}
		\gamma(\lambda) = W(\lambda)\Phi_0(\lambda)^{-1} = W(\lambda)
		\begin{pmatrix}
			1 & 0 \\
			-w_{11}(b,\lambda)w_{12}(b,\lambda)^{-1} & w_{12}(b,\lambda)^{-1}
		\end{pmatrix},
	\end{equation*}
	\begin{equation*}
		M(b,\lambda) = \Phi_1(\lambda)\Phi_2(\lambda)^{-1},
	\end{equation*}
	що призводить до \eqref{eq-4.7}, \eqref{eq-4.8}.
\end{proof}

\subsection{Теорія Вейля для канонічних систем} 

В роботі \cite{Weyl2010} досліджувалась поведінка коефіцієнта Вейля для оператора Штурма-Ліувілля на відрізку $[0,b]$, якщо $b\to\infty$. Зокрема, це дозволило показати, що спектральна задача для оператора Штурма-Ліувілля на прямій завжди має розв'язки в просторі $L^2(0,\infty)$.

В цьому розділі буде проведено аналогічне дослідження для системи \eqref{eq-relation-1} на півосі.

Розглянемо лінійне відношення $A(b,h)$, що продовжується в $L^2_H(0,b)$ системою \eqref{Canonical_sys} і граничними умовами
\begin{equation}\label{eq-4.10}
	f_1(0) = f_2(0) = f_2(b) + hf_1(b) = 0.
\end{equation}
З формули
% TODO ???
випливає, що спряжене лінійне відношення $A(b,h)^*$ задається системою і граничною умовою
\begin{equation}\label{eq-4.11}
	f_2(b) + hf_1(b) = 0.
\end{equation}
Гранична трійка для $A(b,h)^*$ задається рівностями
\begin{equation}\label{eq-4.12}
	\Gamma^{b,h}_0 f = f_1(0),\quad \Gamma_1^{b,h} f = f_2(0).
\end{equation}
Дефектний підпростір $\frN_\lambda(A(b,h))$ складається з вектор-функцій, пропорційних
\begin{equation}\label{eq-4.13}
	\Psi(x,\lambda) = W^T(x,\lambda)
	\begin{pmatrix}
		1 \\ -m(\lambda,b,h)
	\end{pmatrix},
\end{equation}
де коефіцієнт $m(\lambda,b,h)$ знаходиться з умови $\Psi_2(b,\lambda) + h\Psi_1(b,\lambda) = 0$, тобто
\begin{equation*}
	w_{12}(b,\lambda) - w_{22}(b,\lambda)m(\lambda,b,h) + h\{w_{11}(b,\lambda) - w_{21}(b,\lambda)m(\lambda,b,h) \} = 0.
\end{equation*}
Звідси знаходимо
\begin{equation}\label{eq-4.14}
	m(\lambda,b,h) = \frac{w_{11}(b,\lambda)h + w_{12}(b,\lambda)}{w_{21}(b,\lambda)h + w_{22}(b,\lambda)}.
\end{equation}

\begin{theorem}\label{th-4.3}
	Нехай гранична трійка для $A(b,h)^*$ задана формулою \eqref{eq-4.12}. Тоді:
	\begin{enumerate}
		\item Відповідна функція Вейля співпадає з $m(\lambda,b,h)$, а $\gamma$-поле має вигляд
		\begin{equation}\label{eq-4.15}
			\gamma(\lambda,b,h) = W(\cdot,\lambda)
			\begin{pmatrix}
				1 \\ -m(\lambda,b,h)
			\end{pmatrix}.
		\end{equation}
		\item При фіксованому $\lambda\in\bbC_+$ множина значень $m(\lambda,b,h)$ заповнює коло $C_b(\lambda)$ в $\bbC$ з центром
		\begin{equation}\label{eq-4.16}
			\wt{m}_b(\lambda) = \frac{w_2(b,\lambda)^* Jw_1(b,\lambda)}{w_2(b,\lambda)^* Jw_2(b,\lambda)}
		\end{equation}
		і радіусом
		\begin{equation}\label{eq-4.17}
			r_b(\lambda) = \left(2\Im{\lambda}\int\limits_0^b w_2(x,\lambda)^* H(x) w_2(x,\lambda)\,dx \right)^{-1}.
		\end{equation}
		\item Круг $K_b(\lambda)$, обмежений колом $C_b(\lambda)$ характеризується нерівністю
		\begin{equation}\label{eq-4.18}
			\int\limits_0^b \gamma(x,\lambda,b,h)^* H(x) \gamma(x,\lambda,b,h) \le \frac{\Im{m(\lambda,b,h)}}{\Im{\lambda}}.
		\end{equation}
	\end{enumerate}
\end{theorem}
\begin{proof}
	(1) випливає з \eqref{eq-4.12} і \eqref{eq-4.13}, оскільки
	\begin{equation*}
		\Gamma_0\Psi(\cdot,\lambda) = 1, \quad \Gamma_1\Psi(\cdot,\lambda) = m(\lambda,b,h).
	\end{equation*}
	Зауважимо, що відповідне $\gamma$-поле співпадає з $\Psi(\cdot,\lambda)$, тобто в силу \eqref{eq-4.13} виконується \eqref{eq-4.15}.

	З рівності 
	%TODO ???
	отримуємо, що $m(\lambda,b,h)$ належить до кола
	\begin{equation}
		\int\limits_0^b 
		\begin{pmatrix}
			1 & -m(\lambda,b,h)^*
		\end{pmatrix}
		H(x) W(x,\lambda)
		\begin{pmatrix}
			1 \\ -m(\lambda,b,h)
		\end{pmatrix}
		\,dx = \frac{\Im{m(\lambda,b,h)}}{\Im{\lambda}}.
	\end{equation}
	Оскільки дробово-лінійне перетворення \eqref{eq-4.14} переводить $h=-\dfrac{w_{22}(b,\lambda)}{w_{21}(b,\lambda)}$ в $\infty$, то $h=-\dfrac{\overline{w_{22}(b,\lambda)}}{\overline{w_{21}(b,\lambda)}}$ переходить у центр кола $C_b(\lambda)$
	\begin{equation*}
		\wt{m}_b(\lambda) = \frac{w_{12}(b,\lambda)w_{21}(b,\lambda)^* - w_{22}(b,\lambda)^*w_{11}(b,\lambda)}{w_{21}(b,\lambda)^*w_{22}(b,\lambda) - w_{22}(b,\lambda)^*w_{21}(b,\lambda)},
	\end{equation*}
	що співпадає з \eqref{eq-4.16}.

	Радіус кола $C_b(\lambda)$ може бути знайдений з рівності
	\begin{equation}
		r_b(\lambda) = \left| \frac{w_2(b,\lambda)^*Jw_1(b,\lambda)}{w_2(b,\lambda)^*Jw_2(b,\lambda)} - \frac{w_{21}(b,\lambda)}{w_{22}(b,\lambda)} \right|.
	\end{equation}
	Оскільки
	\begin{multline*}
		\left( w_{21}(b,\lambda)^*w_{12}(b,\lambda) - w_{22}(b,\lambda)^*w_{11}(b,\lambda) \right)w_{22}(b,\lambda) \\
		-\left( w_{21}(b,\lambda)^*w_{22}(b,\lambda) - w_{22}(b,\lambda)^*w_{21}(b,\lambda) \right) w_{21}(b,\lambda) \\
		= -w_{22}(b,\lambda)^* \left( w_{11}(b,\lambda)w_{22}(b,\lambda) - w_{21}(b,\lambda)w_{12}(b,\lambda) \right) = -w_{22}(b,\lambda)^*,
	\end{multline*}
	то
	\begin{equation}\label{eq-4.19}
		r_b(\lambda)^{-1} = \left| w_2(b,\lambda)^*Jw_2(b,\lambda) \right|.
	\end{equation}
	З тотожності \eqref{eq-relation-3} отримаємо
	\begin{equation}\label{eq-4.20}
		w_2(b,\lambda)^*Jw_2(b,\lambda) = w_2(b,\lambda)^*Jw_2(b,\lambda) - w_2(0,\lambda)^*Jw_2(0,\lambda) = (\bar{\lambda} - \lambda)\int\limits_0^b w_2(x,\lambda)^*Jw_2(x,\lambda)\,dx.
	\end{equation}
	Рівність \eqref{eq-4.17} випливає з \eqref{eq-4.19} і \eqref{eq-4.20}.
\end{proof}

\begin{corollary}
	Круги $K_b(\lambda)$ вкладені одне в одне $K_{b_2}(\lambda)\subset K_{b_1}(\lambda)$ при $b_1<b_2$.
	При цьому можливо наступне:
	\begin{enumerate}
		\item або $\bigcap\limits_{b>0} K_b(\lambda)$ містить одну точку і тоді існує єдиний розв'язок $\Psi(\cdot,\lambda)$ системи \eqref{eq-relation-1}, який належить $L_H^2(0,\infty)$
		\item або $\bigcap\limits_{b>0} K_b(\lambda)$ є граничний круг $K_\infty(\lambda)$ і тоді кожний розв'язок системи \eqref{eq-relation-1} належить до $l^2_H(0,\infty)$.
	\end{enumerate}
\end{corollary}

Зрозуміло, що в першому випадку маємо $\dim \frN_\lambda = 1$, а в другому --- $\frN_\lambda = 2$.

З загальної теорії розширень симетричних операторів випливає, що для всіх $\lambda\in\bbC_+$ одночасно має місце випадок граничної точки, якщо це трапляється для однієї точки $\lambda_0\in\bbC_+$.

\begin{definition}\label{def-4.5}
	Будемо говорити, що для системи \eqref{eq-relation-1} має місце
	\begin{enumerate}
		\item випадок граничної точки, якщо $K_\infty(\lambda)$ складається з однієї точки для всіх $\lambda\in\bbC_+$;
		\item випадок граничного круга, якщо $K_\infty(\lambda)$ --- це круг для всіх $\lambda\in\bbC$.
	\end{enumerate}
\end{definition}
В першому випадку $n_\pm(T_0) = 1$, а в другому $n_\pm(T_0) = 2$.

\begin{corollary} [Аналог теореми Вейля]
	Нехай $H\in L_{loc}^1[0,\infty)$. Тоді існує принаймні один розв'язок системи
	\begin{equation*}
		Jy'=\lambda Hy,
	\end{equation*}
	який належить до $L^2_H(0,\infty)$.
\end{corollary}

\subsection{Граничні трійки для канонічної системи у випадку граничної точки в $\infty$}

\begin{theorem}
	Система \eqref{eq-relation-1} має випадок граничної точки в $b$ тоді і тільки тоді, коли $\trace H \notin L^1(c,b)$ для $c<b$.
\end{theorem}
\begin{proof}
	Якщо $H\in L^1[c,b)$ для деякої точки $c\in [0,b)$, то система \eqref{eq-relation-1} є регулярною в точці $b$, тобто оператор $T_0$ має індекси дефекту $(2,2)$. Це означає, що для системи \eqref{eq-relation-1} має місце випадок граничної точки в $b$.

	Навпаки, припустимо, що для системи \eqref{eq-relation-1} має місце випадок граничного круга в $b$. Тоді всі точки з $\bbC$ є точками регулярного типу і індекси дефектного підпростору оператора $T_0$ дорівнюють $(2,2)$. Для точки $\lambda = 0$ маємо 2 лінійно незалежних розв'язки системи \eqref{eq-relation-1}
	\begin{equation*}
		Jy'=0
	\end{equation*}
	$y_1\equiv e_1$ і $y_2\equiv e_2$, які належать до $L^2_H(c,b)$. Тоді
	\begin{equation*}
		\int\limits_0^b \trace H(x)\,dx = 
		\int\limits_0^b (H(x)e_1,e_1) + (H(x)e_2,e_2)\,dx = 
		||y_1||^2_{L^2_H(c,b)} + ||y_2||^2_{L^2_H(c,b)} < \infty,
	\end{equation*}
	тобто $\trace{H} \in L^1(c,b)$.
\end{proof}

\begin{definition}
	Якщо для системи \eqref{eq-relation-1} має місце випадок граничної точки в $b$, то з Означення~\ref{def-4.5} випливає, що існує єдине значення $m_\infty(\lambda)$ для кожного $\lambda\in\bbC_+$ таке, що $\Psi(\cdot,\lambda) = W^T(\cdot,\lambda)
\begin{pmatrix}
	1 \\ -m_\infty(\lambda)
\end{pmatrix}
	$ належить до $L^2_H(0,b)$. Коефіцієнт $m_\infty(\lambda)$ називають коефіцієнтом Вейля-Тітчмарша системи \eqref{eq-relation-1}, а відповідний розв'язок системи $\phi(\lambda)$ називають розв'язком Вейля.

	Функція Вейля $m_\infty(\lambda)$ знаходиться за формулою
	\begin{equation}\label{eq-4.23A}
		m_\infty(z) = \lim_{x\to\infty} \frac{w_{11}(x,z)h + w_{12}(x,z)}{w_{21}(x,z)h+w_{22}(x,z)},
	\end{equation}
	причому границя в \eqref{eq-4.23A} не залежить від вибору $h\in\bbR$.
\end{definition}

\begin{lemma}\label{lemma-4.9}
	Якщо $H\notin L^1(0,b)$, то для всіх $(f,g)\in \calT$ має місце 
	\begin{equation}\label{eq-4.21}
		\lim_{x\to b} f_0^*Jf_0 = 0
	\end{equation}
	і область визначення мінімального лінійного відношення $\calT_0$ задається рівністю
	\begin{equation}\label{eq-4.22}
		\dom{\calT_0} = \{ f_0\in \dom{\calT}: f_{01}(0) = f_{02}(0)=0 \}
	\end{equation}
\end{lemma}
\begin{proof}
	Нехай $u,\ v$ --- функції, визначені в \eqref{eq-4.4}, які є фінітними в околі точки $b$. Оскільки $n_\pm(T_0)=1$, то
	\begin{equation*}
		\dim (\dom{\calT}/\dom{\calT_0}) = n_+(\calT_0) + n_-(\calT_0) = 2 
	\end{equation*}
	і тому кожна функція з $\dom{\calT}$ допускає представлення
	\begin{equation}\label{eq-4.23}
		f_0=h_0+c_1u+c_2v.
	\end{equation}
	Оскільки $u,\ v$ є фінітними в точці $b$, то
	\begin{equation}
		\lim_{x\to b} f^*Ju = \lim_{x\to b} f^*Jv, \quad \forall f\in\dom{T}.
	\end{equation}
	Далі з Теореми~\ref{th-relation-1} випливає, що для $(f,g)\in T$, $(h,k)\in T_0$
	\begin{equation*}
		0 = \left< g,h \right> - \left< f,k \right> = \lim f_0^*Jh_0.
	\end{equation*}
	Тому має місце \eqref{eq-4.21}. Рівність \eqref{eq-4.22} випливає з \eqref{eq-4.23} і \eqref{eq-relation-3}.
\end{proof}

\begin{theorem}\label{th-4.10}
	Нехай $H\notin L^1(0,b)$. Тоді для системи \eqref{eq-relation-1} має місце випадок граничної точки в $b$. При цьому:
	\begin{enumerate}
		\item Сукупність $\{\bbC,\Gamma_0,\Gamma_1\}$, в якій
		\begin{equation}\label{eq-4.24}
			\Gamma_0\widehat{f}=f_{01}(0),\quad \Gamma_1\widehat{f} = -f_{02}(0),
		\end{equation}
		утворює граничну трійку для $\calT$.
		\item Відповідна функція Вейля співпадає з коефіцієнтом Вейля-Тітчмарша $m_\infty(\lambda)$.
	\end{enumerate}
\end{theorem}
\begin{proof}
	Я було показано в Лемі~\ref{lemma-4.9}
	\begin{equation*}
		\lim_{x\to\infty} f_0^*Jh_0 = 0 \quad \forall (f,g), (h,k)\in T.
	\end{equation*}
	Тому рівність \eqref{eq-relation-3} приймає вигляд
	\begin{equation}
		(h,g)_{L^2_H} - (k,f)_{L^2_H} =\Bigl. -f^*_0Jh_0\, \Bigr|_0 = f^*_{01}(0)h_{02}(0) - f^*_{02}(0)h_{01}(0).
	\end{equation}
	Це доводить формулу \eqref{eq-CS-Green}. Сюр'єктивність відображення випливає з представлення \eqref{eq-4.23}.

	Твердження (2) випливає з рівностей
	\begin{equation*}
		\Gamma_0\Psi = \Gamma_0
		\begin{pmatrix}
			w_{11}-w_{21} & m_\infty(\lambda) \\
			w_{21}-w_{22} & m_\infty(\lambda)
		\end{pmatrix}
		=w_{11}(0,\lambda) - w_{21}(0,\lambda)m_\infty(\lambda) = 1,
	\end{equation*}
	\begin{equation*}
		\Gamma_1\Psi = -(w_{21}(0,\lambda)-w_{22}(0,\lambda)m_\infty(\lambda)) = m_\infty(\lambda).
	\end{equation*}
\end{proof}

\begin{lemma}\normalfont{\cite{Win1995}}\label{lemma-4.11}
	Нехай дані дві канонічні системи з гамільтоніанами $H(x)$ і $\wt{H}(x)=H(l+x)$ для деякого $l>0$ і $x\in [0,\infty)$. Якщо $W$ --- фундаментальна матриця системи, що відповідає $H$, а $m$ і $\wt{m}$ --- коефіцієнти Вейля, відповідні до $H(x)$ і $\wt{H}$. Тоді
	\begin{equation}\label{eq-4.35}
		m(z) = \frac{w_{11}(l,z) \wt{m}(z) + w_{12}(l,z)}{w_{21}(l,z) \wt{m}(z) + w_{22}(l,z)}.
	\end{equation}

	Зокрема, якщо $(0,l)$ --- сингулярний інтервал типу $\phi$ для $H$, тоді
	\begin{equation}\label{eq-4.36}
		Q(z) = \ctg{(\phi)} + \dfrac{1}{-zl\sin^2{\phi} + \dfrac{1}{\wt{Q}(z) - \ctg{(\phi)}}}, \text{ якщо } \phi\ne0
	\end{equation}
	і
	\begin{equation}\label{eq-4.37}
		Q(z) = lz + \wt{Q}(z), \text{ якщо } \phi = 0.
	\end{equation}
	
\end{lemma}
\begin{proof}
	Матричні функції $W(l+x,z)$ і $W(l,z)\wt{W}(x,z)$ є фундаментальними матрицями для канонічної системи \eqref{Canonical_sys}, отже $W(l+x,z) = W(l,z)\wt{W}(x,z)$ за Теоремою~\ref{th-CS-2}. Тоді з рівності
	\begin{equation*}
		m(z) = \lim_{x\to\infty} \dfrac{w_{11}(l+x,z)}{w_{12}(l+x,z)} = \lim_{x\to\infty} \dfrac{w_{11}(l,z)\wt{w}_{11}(x,z) + w_{12}(l,z)\wt{w}_{21}(x,z)}{w_{21}(l,z)\wt{w}_{11}(x,z) + w_{22}(l,z)\wt{w}_{21}(x,z)}
	\end{equation*}
	отримаємо \eqref{eq-4.35}. Якщо інтервал $(0,l)$ --- сингулярний інтервал типу $\phi$ для $H$, то
	\begin{equation*}
		W(l,z) = I - zlHJ =
		\begin{pmatrix}
			1-zl\sin{\phi}\cos{\phi} & zl\cos^2{\phi}\\
			-zl\sin^2{\phi} & 1+zl\sin{\phi}\cos{\phi}
		\end{pmatrix}.
	\end{equation*}
	Підставивши останнє у \eqref{eq-4.35}, отримаємо \eqref{eq-4.36} і \eqref{eq-4.37} в якості першого кроку неперервного розвинення $m(z)$ у неперервний дріб.
\end{proof}

%!TEX root = ../Masters.tex
\section{ПРИКЛАДИ КАНОНІЧНИХ СИСТЕМ}

\begin{example}
	Розглянемо лінійну систему
	\begin{equation}\label{eq-4.25}
		Jy'=-zy \text{ на інтервалі } [0,b].
	\end{equation}
	Тут $H(x)\equiv I$ і $\calH = L^2_{I_2}(0,b) = L^2(0,b)\oplus L^2(0,b)$.

	Фундаментальна матриця $W(x,z)$ приймає вигляд
	\begin{equation}\label{eq-4.26}
	  	W(x,z) = 
	  	\begin{pmatrix}
	  		\cos{zx} & \sin{zx}\\
	  		-\sin{zx} & \cos{zx}
	  	\end{pmatrix}.
	\end{equation}
	Дійсно,
	\begin{equation*}
		W'(x,z)J = z
		\begin{pmatrix}
	  		-\sin{zx} & \cos{zx}\\
	  		-\cos{zx} & -\sin{zx}
	  	\end{pmatrix}
	  	\begin{pmatrix}
	  		0 & -1\\
	  		1 & 0
	  	\end{pmatrix}
	  	= z
	  	\begin{pmatrix}
	  		-\cos{zx} & \sin{zx}\\
	  		-\sin{zx} & \cos{zx}
	  	\end{pmatrix}
	  	=zW(x,z)
	\end{equation*}
	і $W(0,z) = I_2$.

	Знайдемо функцію Вейля, що відповідає граничній трійці \eqref{eq-4.3}:
	\begin{equation*}
		\Gamma_0W(z,x)^T c = \Gamma_0
		\begin{pmatrix}
	  		c_1 \cos{zx} - c_2 \sin{zx}\\
	  		c_1 \sin{zx} + c_2 \cos{zx}
	  	\end{pmatrix}
	  	=
	  	\begin{pmatrix}
	  		c_1\\
	  		c_1 \sin{zb} - c_2 \sin{zb}
	  	\end{pmatrix};
	\end{equation*}
	\begin{equation*}
		\Gamma_1W(x,z)^T c = 
		\begin{pmatrix}
	  		-c_2\\
	  		c_1\sin{zb} + c_2 \cos{zb}
	  	\end{pmatrix}.
	\end{equation*}
	Звідси знаходимо
	\begin{multline}\label{eq-4.27}
		M(z) = 
		\begin{pmatrix}
	  		0 & -1\\
	  		\sin{zb} & \cos{zb}
	  	\end{pmatrix}
	  	\cdot
	  	\begin{pmatrix}
	  		1 & 0\\
	  		\cos{zb} & -\sin{zb}
	  	\end{pmatrix}^{-1}
	  	\\=
	  	\begin{pmatrix}
	  		0 & -1\\
	  		\sin{zb} & \cos{zb}
	  	\end{pmatrix}
	  	\cdot
	  	\begin{pmatrix}
	  		1 & 0\\
	  		\frac{\cos{zb}}{\sin{zb}} & -\frac{1}{\sin{zb}}
	  	\end{pmatrix}
	  	=
	  	\begin{pmatrix}
 	  		-\frac{\cos{zb}}{\sin{zb}} & \frac{1}{\sin{zb}}\\
	  		\frac{1}{\sin{zb}} & -\frac{\cos{zb}}{\sin{zb}}
	  	\end{pmatrix}
	\end{multline}
\end{example}

\begin{example}
	Розглянемо систему
	\begin{equation}\label{eq-4.28}
		Jy'=-zy \text{ на інтервалі } (0,\infty).
	\end{equation}
	Тут $H(x)=I_2\in L^1_{loc}[0,\infty)$, але $H\notin L^1(0,\infty)$.

	Тому за Теоремою~\ref{th-4.10} для системи \eqref{eq-4.28} має місце випадок граничної точки. Фундаментальна матриця системи \eqref{eq-4.28} має вигляд \eqref{eq-4.26}, а відповідна функція Вейля знаходиться за формулою
	\begin{equation*}
		m_\infty(z) = \lim_{x\to\infty} \frac{w_{11}(x,z)}{w_{21}(x,z)}.
	\end{equation*}
	Тому для $z\in\bbC_+$ отримаємо
	\begin{equation*}
		m_\infty(z) = \lim_{x\to\infty} \frac{\cos{zx}}{-\sin{zx}} = \lim_{x\to\infty} i\frac{e^{-ixz}+e^{ixz}}{e^{-ixz}-e^{ixz}} = i.
	\end{equation*}
	Таким чином відповідна функція Вейля має вигляд
	\begin{equation*}
		m_\infty(z) = 
		\begin{cases}
			i, & z\in\bbC_+\\
			-i, & z\in\bbC_-
		\end{cases}
	\end{equation*}
\end{example}

\begin{example}
	Розглянемо систему \eqref{Canonical_sys} на інтервалі $(0,\infty)$ з гамільтоніаном $H(x)$, який задається наступним чином:
	 \begin{gather}\label{eq-4.31}
	 	H(x) = H_j = c_{\alpha_j}c^*_{\alpha_j}, \ x\in [x_{j-1},x_j], \ j=1,\ldots,n\\
	 	H(x) \equiv I, \ x\in[x_n,\infty). \notag
	 \end{gather}
	Тут $c_{\alpha_j}$ мають вигляд $c_{\alpha_j} = \dbinom{\cos{\alpha j}}{\sin{\alpha j}}$, див.~\eqref{eq-si-1}, а $x_j$ --- точки на півосі $[0,\infty)$:
	\begin{equation*}
		0=x_0<x_1<\ldots<x_{n-1}<x_n.
	\end{equation*}

	Матрицант на кожному інтервалі $[x_{j-1},x_j]$ має вигляд
	\begin{equation}\label{eq-4.32}
		W_j(x,z) = I - zH_jJx,
	\end{equation}
	оскільки $W'_j = -zH_jJ$, тобто $W'_j J = zH_j$.

	Зауважимо, що $H_jJH_j =0$ і тому
	\begin{equation*}
		W_jH_j = (1-zH_jJx)H_j = H_j,
	\end{equation*}
	тобто $W'_j J = $ задовольняє системі
	% TODO вставить ссілку на систему
	\begin{equation*}
		W'_j J = zH_j = zW_jH.
	\end{equation*}
	
	За теоремою
	% TODO вставить ссылку на теорему
	матрицант системи \eqref{Canonical_sys} має вигляд
	\begin{equation}\label{eq-4.33}
		W(x,z) = (I-zl_1H_1J)(I-zl_2H_2J)\ldots(I-zl_nH_nJ)i, \ z\in\bbC_+,
	\end{equation}
	де $l_j=x_j-x_{j-1}$, $j=1,\ldots,n$.

	Оскільки $H\notin L^1(0,\infty)$, то для системи \eqref{Canonical_sys} має місце випадок граничної точки в $\infty$ і відповідна функція Вейля знаходиться за формулою \eqref{eq-4.23A}.

	За теоремою
	% TODO вставить ссылку на теорему
	функція Вейля канонічної системи з гамільтоніаном \eqref{eq-4.31} приймає вигляд неперервного дробу
	\begin{equation}
		Q(z) = \ctg{\alpha_1} + \dfrac{1}
									{-zb_1 + \dfrac{1}
												{a_2 + \dfrac{1}
															{-zb_2 + \ldots \dfrac{1}
																				{-zb_n + \dfrac{1}
																							{i-\ctg{\alpha_n}}}}}}
	\end{equation}
	де $b_j = l_j\sin^2{\alpha_j}$, $a_j = \ctg{\alpha_j} - \ctg{\alpha_{j-1}}$, $j=1,\ldots,n$.
\end{example}

%!TEX root = ../Masters.tex

\section*{ВИСНОВКИ}
\addcontentsline{toc}{section}{ВИСНОВКИ}

\begin{enumerate}
	\item Знайдено вигляд граничної трійки канонічної системи у регулярному випадку.
	\item Наведено класифікацію сингулярних точок канонічних систем, аналогічну класифікації Вейля для операторів Штурма-Ліувілля.
	\item Знайдено вигляд граничної трійки канонічної системи у сингулярному випадку.
	\item Знайдено вигляд фундаментальної матриці і функцій Вейля для зчеплення двох канонічних систем.
	\item Розглянуто приклади канонічних систем як у регулярному, так і у сингулярному випадку.
	\item Знайдено функцію Вейля для зчеплення кількох канонічних систем, що відповідають сингулярним інтервалам.
\end{enumerate}

\setlength{\parskip}{0cm}
%
% %
\nocite{*}
\renewcommand{\refname}{СПИСОК ВИКОРИСТАНИХ ПОСИЛАНЬ}
\printbibliography
\addcontentsline{toc}{section}{\refname}
%

%!TEX root = ../Masters.tex
\thispagestyle{empty}
\section*{ДЕКЛАРАЦІЯ ЩОДО УНІКАЛЬНОСТІ ТЕКСТІВ РОБОТИ ТА НЕВИКОРИСТАННЯ МАТЕРІАЛІВ ІНШИХ АВТОРІВ БЕЗ ПОСИЛАНЬ}

\vspace{2em}

\begin{minipage}[l]{0.5\linewidth}
	\hrulefill \\
	{\tiny{Прізвище, ім’я, по батькові}}

	\hrulefill \\
	{\tiny{Факультет}}

	\hrulefill \\
	{\tiny{Шифр і назва спеціальності}}

	\hrulefill \\
	{\tiny{Освітня програма}}
\end{minipage}

\vspace{2em}

\begin{center}
	\textbf{ДЕКЛАРАЦІЯ}
\end{center}

Усвідомлюючи свою відповідальність за надання неправдивої інформації,
стверджую, що подана магістерська робота на тему: <<\underline{\hspace{9cm}} \underline{\hspace{15cm}}>> є написаною мною особисто.

Одночасно заявляю, що ця робота:
\begin{itemize}
	\item[--] не передавалась іншим особам і подається до захисту вперше;
	\item[--] не порушує авторських та суміжних прав, закріплених статтями 21–25 Закону України «Про авторське право та суміжні права»;
	\item[--] не отримувались іншими особами, а також дані та інформація не отримувались у недозволений спосіб.
\end{itemize}

Я усвідомлюю, що у разі порушення цього порядку моя магістерська
робота буде відхилена без права її захисту, або під час захисту за неї буде
поставлена оцінка «незадовільно».

\vspace{2em}

\flushright{
	\begin{minipage}[c]{0.5\linewidth}
	\hrulefill \\
	{\tiny{Дата і підпис студента}}
	\end{minipage}
}
\end{document}
