%!TEX root = ../Masters.tex

\section*{ВСТУП}
\addcontentsline{toc}{section}{ВСТУП}


Канонічною системою називається диференціальне рівняння у вигляді

\begin{equation*} 
	Ju'(x) = -zH(x)u(x), \quad
	J=
	\begin{pmatrix}
		0 & -1\\
		1 & 0
	\end{pmatrix}.
\end{equation*}

Ця система зі спектральним параметром $z$ розглядатиметься на відкритому, можливо нескінченному, інтервалі $x\in (a,b)$, $-\infty\le a<b \le +\infty$ в якій $H$ --- дійснозначна $2\times 2$ матриця-функція, яка належить $L^1_{loc}(a,b)$ і не дорівнює тотожньо нулю. Канонічні системи представляють великий математичний інтерес, оскільки вони в точному розумінні є найбільш загальним класом симетричних операторів другого порядку.\cite{Remling2018}

Теорія канонічних систем включає в собі всі види диференціальних операторів другого порядку такі як оператор Штурма-Ліувілля, струна Крейна-Феллера, різницеві оператори, пов'язані з матрицею Якобі, та інші. Основи спектральної теорії канонічних систем було закладено в роботах М.\,Г.~Крейна (див. також монографії \cite{KreinGohb,Atkinson}). Повний опис спектральних функцій канонічних систем другого порядку було отримано Л. де Бранжем \cite{deBranges}. Сучасна теорія канонічних систем представлена в монографіях \cite{ArovD12,Remling2018,Sakhnovich}.

Іншій важливий об'єкт, який розглядається в роботі --- це функції Вейля-Тітчмарша.  Для оператора Штурма-Ліувілля цю функцію було введено Германом Вейлем у звязку з класифікацією сингулярних точок оператора методом теорії вкладених кругів Вейля. У подальшому цю функцію було використано Тітчмаршем для обчислення спектральної функції оператора Штурма-Ліувілля. В роботах В. Деркача і М. Маламуда було введено абстрактний варіант функції Вейля-Тітчмарша і досліджено спектри довільних розширень симетричного оператора у просторі Гільберта.
Важливість функції Вейля для спектральної теорії канонічних систем  випливає з того, що їх інтегральні представлення дозволяють обчислити спектральні функції самоспряжених операторів, що відповідають канонічній системі.

В магістерській роботі застосовано теорію граничних трійок, що було розвинено в роботах \cite{Koch1975,Gorb1991} до канонічних систем. Зокрема, побудовано  теорію вкладених кругів Вейля для канонічних систем, знайдено формули для граничних трійок для канонічних систем як в регулярному, так і в сингулярному випадку  граничного кола у нескінченності. Наведено формулу про факторизацію фундаментальних матриц для зчеплення двох канонічних систем. Це дозволило обчислювати функції Вейля для граничних трійок і функцій Вейля багатьох канонічних систем у явному вигляді. В роботі розглянуто три приклади канонічних систем, для яких знайдено граничні трійки і їх функції Вейля.