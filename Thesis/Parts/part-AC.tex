%!TEX root = ../Masters.tex

\subsection{Клас абсолютно неперервних функцій}

\begin{definition}
	Функція $u:(a,b)\to \bbC^2$ називається локально (абсолютно) неперервною, якщо
	\begin{equation} \label{eq-AC}
	 	u(x) = u(c) + \int_c^x{f(t)\,dt}
	\end{equation}
	для деакої локально інтегровної функції $f:(a,b)\to \bbC^2$ і $c\in(a,b)$. Клас таких функцій позначається $AC(a,b)$.
\end{definition}

Якщо рівність~\eqref{eq-AC} виконується для деякого $c$, то вона виконується і для всіх $c\in(a,b)$.

\begin{theorem}
	Функція $u\in AC$ диференційовна в класичному сенсі майже скрізь на $x\in (a,b)$.
\end{theorem}

Якщо $u\in AC(a,b)$, то $f$ однозначно визначається $U$ з точністю до зміни її значень на множині міри нуль. При цому $f$ називають похідною від $u$.

Альтернативна інтерпретація абсолютно безперервних функцій, яка іноді корисна, надається теорією розподілів: локально інтеговна функція $u$ буде в $AC$ тоді і тільки тоді, коли її узагальнена похідна $u'$ є локально інтегровною.