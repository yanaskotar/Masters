%!TEX root = ../Masters.tex
\section{КАНОНІЧНІ СИСТЕМИ}

\subsection{Основні поняття}

Канонічні системи представляють великий математичний інтерес, оскільки вони в точному розумінні є найбільш загальним класом симетричних операторів другого порядку.\cite{Remling2018}

Канонічною системою називається диференціальне рівняння у формі

\begin{equation} \label{Canonical_sys}
	Ju'(x) = -zH(x)u(x), \quad 
	J=
	\begin{pmatrix}
		0 & -1\\
		1 & 0
	\end{pmatrix}.
\end{equation}

Ця система розглядатиметься на відкритому, можливо нескінченному, інтервалі $x\in (a,b)$, $-\infty\le a<b \le +\infty$, в якій для матриці коефіцієнтів $H$ виконуються наступні умови:
\begin{enumerate} 
	\item $H(x) \in \bbR^{2\times 2}$;
	\item $H \in L^1_{loc}(a,b)$; \label{CS-en2}
	\item $H(x)\ge 0 \text{ майже для всіх } x\in(a,b)$; \label{CS-en3}
	\item $H(x)\ne 0 \text{ майже для всіх } x\in(a,b)$. \label{CS-en4}
\end{enumerate}

Умова \eqref{CS-en2} означає, що елементи $H$ є локально інтегровними функціями, а умова \eqref{CS-en3} означає, що майже для всіх $x$ матриця $H(x)$ є симетричною і $v^*H(x)v\ge 0$ для всіх $v\in\bbC^2$.

Умова \eqref{CS-en4} також є влучною, оскільки якби існував би інтервал $(c,d)$, на якому $H = 0$ майже скрізь, то розв'язки просто залишалися б постійними на $(c, d)$, а вилучення інтервалу не впливало б на доповнення.

Параметр $z\in\bbC$ з \eqref{Canonical_sys} іноді називають спектральним параметром.

Диференціальне рівняння \eqref{Canonical_sys} має загальну структуру задачі про власне значення. А саме, якщо диференціальний оператор $\tau$ діє у $\bbC^2$ за правилом 
\begin{equation*}
 	(\tau u)(x) = -H^{-1}(x)Ju'(x),
\end{equation*}
тоді \eqref{Canonical_sys} потребує, щоб виконувалось $\tau u=zu$. Звичайно, щоб стверджувати це, необхідно визначити які умови роблять такий оператор самоспряженим, в яких гільбертових просторах діє такий оператор та що робити, якщо $H(x)$ не має оберненої матриці.

Оскільки $H(x)$ може бути нерегулярною, то потрібна деяка інтерпретація \eqref{Canonical_sys}.

\begin{definition}
	Функція $u:(a,b)\to \bbC^2$ називається локально (абсолютно) неперервною, якщо
	\begin{equation} \label{eq-AC}
	 	u(x) = u(c) + \int_c^x{f(t)\,dt}
	\end{equation}
	для деякої локально інтегровної функції $f:(a,b)\to \bbC^2$ і $c\in(a,b)$. Клас таких функцій позначається $AC(a,b)$.
\end{definition}

Якщо рівність~\eqref{eq-AC} виконується для деякого $c$, то вона виконується і для всіх $c\in(a,b)$.

Функція $u:(a,b)\to\bbC^2$ буде називатися розв'язком \eqref{Canonical_sys}, якщо $u\in AC$ і \eqref{Canonical_sys} справджується майже всюди. Також, якщо $H_1(x) = H_2(x)$ майже всюди, то два рівняння будуть мати однакові у тому ж сенсі розв'язки.

Існування та єдиність такого розв'язку системи \eqref{Canonical_sys} показує наступна теорема, сформульована для загального, неоднорідного, випадку.

\begin{theorem}\label{Th-CS-1}
	Нехай $c\in(a,b)$ і $f:(a,b)\to \bbC^2$ є локально інтегровною. Тоді для будь-якого $v\in\bbC^2$ задача
	\begin{equation}\label{eq-inhom-can-sys}
		Ju'=-zHu+f,\quad u(c) = v
	\end{equation}
	має єдиний розв'язок $u=u(x,z)$ на $x\in(a,b)$.

	Більше того, $u(x,z)$ є 
	% TODO "спільно неперервними" " jointly continuous" что это вообще такое?
	спільно неперервними на $(x,z)\in (a,b)\times\bbC$. Для кожного фіксованого $x\in(a,b)$ компоненти $u(x,z)$ є цілими функціями $z\in\bbC$. Похідні $u_n(x, z): = \partial^n u(x,z)/\partial{z^n}$ самі по собі є абсолютно неперервними функціями на $x\in(a,b)$, і вони формально рзв'язують початкові задачі, що виникають в результаті диференціювання \eqref{eq-inhom-can-sys} відносно $z$, як
	\begin{align*}
		Ju'_1 &= -zHu_1 - Hu_0, \ u_1(c)=0,\\
		Ju'_2 &= -zHu_2 - 2Hu_1, \ u_2(c)=0,\\
		&\dots \\
		Ju'_n &= -zHu_n - nHu_{n-1}, \ u_n(c)=0.
	\end{align*}
\end{theorem}

Якщо ж $f=0$, то твердження про єдиність передбачає, що множина розв'язків $u$ рівняння \eqref{Canonical_sys} є двовимірним векторним простором.

Матрицею переходу $T$ називається матричний розв'язок, що приймає значення в $\bbC^{2\times 2}$ однорідного рівняння
\begin{equation}\label{eq-Canonycal_sys_matrix}
	JT'=-zHT
\end{equation}
з початковою умовою $T(c) = I$. Таким чином, $T$ залежить від $x,c \in (a, b)$ і $z \in\bbC$ і позначається $T(x,c;z)$. При цьому, якщо другий аргумент відсутній, то вважається, що $c=0$.

\begin{theorem}\label{th-CS-2}
	Зафіксуємо $x,c\in(a,b)$, $x\ge c$. Тоді матриця переходу $T(z) = T(x,c;z)$ має наступні властивості як функція від $z\in\bbC$:
	\begin{enumerate}
		\item $T(z)$ --- ціла; \label{i-1-th-2}
		\item $T(0) = I$  і $\det{T(x,z)} = 1$ \label{i-2-th-2}
		\item $T(x,z) = T(x,c;z)T(c,z)$, де $T(x,z) = T(x,0;z)$; \label{i-3-th-2}
		\item Якщо $\Im z\ge 0$, то \label{i-4-th-2}
		\begin{equation}\label{eq-1-th-2}
			i(T^*(z)JT(z)-J)\ge 0.
		\end{equation}
	\end{enumerate}
\end{theorem}
\begin{proof}
	За другою частиною Теореми \ref{Th-CS-1}, $T$ є цілим в $z$ для фіксованих $x$ і $c$. Стовпці $T$ розв'язують \eqref{Canonical_sys} як функції від $x$. Загалом, якщо $v\in\bbC^2$, то унікальний розв'язок з початковою умовою $u (c) = v$ задається як $u (x) = T (x, c; z) v$. Отже, як випливає з назви, матриця переходу оновлює значення розв'язків, а точніше, $T (x, c; z)$ пробігає значення аргументу від $c$ до $x$. Ця властивість характеризує матрицю переходу $T (x, c; z)$.

	Оскільки $J^{-1}=-J$, маємо, що $T'= zJHT$, а матриця $JH$ має нульовий слід. Ця властивість еквівалентна тому, що $H$ симетрична. Звідси випливає, що $\det T(x)$ є постійним і, як початкова умова, $\det T = 1$, тому це справедливо для всіх $x \in (a,b)$. Зокрема, $T$ має обернену, а $T^{-1}$ нейтралізує дію $T$ і, оскільки $T$ приймає значення, пробігаючи від $c$ до $x$, то ${T(x,c;z)}^{-1} = T(c,x;z)$. Отже, $T$ є абсолютно неперервною функцією свого другого аргументу.

	Зрозуміло, що $T(0)=I$, оскільки рівняння стає $T'=0$ для $z=0$. Якщо $z=t$ є дійсними, то $T(x,c;t)$ є розв'язком задачі Коші з дійсними коефіцієнтами та дійсними початковими умовами, тому приймає дійсні значення.

	Залишилося встановити \eqref{eq-1-th-2}. Насамперед, оскільки $J^*=-J$, то спряжене від $JT'=-zHT$ дає $-T^{*}{'} J=-\bar{z}T^* H$. Позначимо $z=t+iy$, $y\ge 0$, і розглянемо
	\begin{equation*}
		\frac{d}{dx}(T^*(x,c;z)JT(x,c;z)) = (\bar{z}-z)T^*HT=-2iyT^*HT.
	\end{equation*}
	Інтегрування цього рівняння показує, що ліва частина \eqref{eq-1-th-2} дорівнює $2y\int_c^x T^*HT\,ds$ і є додатно визначена матриця, як було заявлено.
\end{proof}

Якщо задана матрична функція $T(z)$ має властивості, зазначені в теоремі, то на інтервалі $(c,x)$ буде канонічна система. Іншими словами, на такому проміжку буде функція коефіцієнтів $H$ така, що $T(z) = T(x,c;z)$. Більше того, після відповідної нормалізації, канонічна система однозначно визначається $T(z)$. Варіант розрахунку, що був використаний в останньому доведенні, дає ще одну корисну тотожність.

\begin{theorem}[Сталість Вронскіану]
	\begin{equation*}
		T^T(x)JT(x)=J,\quad T(x)\equiv T(x,c;z).
	\end{equation*}
	Як наслідок, вронскіан $W(v,w)\equiv v^T(x)Jw(x)$ є сталим для будь-яких двох розв'язків $v$, $w$ рівняння $Ju'=-zHu$.
\end{theorem}
\begin{proof}
	Перша тотожність випливає з
	\begin{equation*}
		(T^{T}JT)'=-(JT')^{T}T+T^{T}JT'=z(HT)^{T}T-zT^{T}HT=0.
	\end{equation*}

	Аналогічно для останнього виразу:
	\begin{equation*}
		(v^T(x)Jw(x))'= -(Jv'(x))^T + v^T(x)Jw'(x) = zv^T(x)Hw(x) - zv^T(x)Hw(x) = 0.
	\end{equation*}
\end{proof}

\subsection{Сингулярні інтервали}

\begin{definition}\label{def-sing-point&int}
	Точка $x\in(a,b)$ називається сингулярною, якщо знайдуться $\delta>0$ і вектор $v\in\bbR^2$, $v\ne0$ такі, що $H(t)v=0$ майже для всіх $|t-x|<\delta$. В іншому випадку точка $x\in(a,b)$ називається регулярною.

	Множина $S$ сингулярних точок є відкритою, а її компоненти зв'язності $(c,d)$ називаються сингулярними інтервалами. 
\end{definition}

Відкритість множини $S$ одразу випливає з означення. Можна записати $S=\bigcup(c_j,d_j)$ як зліченне об'єднання відкритих інтервалів, що не перетинаються, які є сингулярними інтервалами, визначеними у \ref{def-sing-point&int}.

Нехай тепер $x\in(a,b)$ --- сингулярна точка. Оскільки $H\ge0$ і $H\ne0$ майже скрізь на $(x-\delta,x+\delta)$, ця функція може бути представлена у вигляді $H(t)=h(t)P_\alpha$ на цьому інтервалі майже скрізь для деякого $\alpha\in[0,\pi)$ та деякої функції $h\in L^1(x-\delta,x+\delta)$, $h>0$, де
\begin{equation}\label{eq-si-1}
	P_\alpha = 
	\begin{pmatrix}
		\cos^2{\alpha} & \sin{\alpha}\cos{\alpha} \\
		\sin{\alpha}\cos{\alpha} & \sin^2{\alpha}
	\end{pmatrix}
	= e_\alpha e^*_\alpha, \quad e_\alpha = \binom{\cos{\alpha}}{\sin{\alpha}}
\end{equation}
позначає проекцію на $e_\alpha$. Ті ж зауваження стосуються всього сингулярного інтервалу, якому належить $x$. 

\begin{definition}
	Кут $\alpha$ називається типом сингулярного інтервалу $(c,d)$.
\end{definition}

% Іноді для зручності вектор $e_\alpha$ також називається типом сингулярного інтервалу.

Розв'язання \eqref{Canonical_sys} через сингулярний інтервал $(c,d)$ типу $\alpha$ дає більше розуміння сенсу Означення \ref{def-sing-point&int}. Тож нехай $H(x)=h(x)P_\alpha \equiv h(x)P$ на $(c,d)$. Оскільки лише скалярна $h$ залежить від $x$, то будь-які дві матриці $JH(x)$, $JH(x')$ є комутативними, тому рівняння $u'=zJHu$ має розв'язок
\begin{equation*}
	u(x) = e^{z\left(\int\limits_c^x h(t)\,dt\right)JP}u(c).
\end{equation*}
Розкладемо останнє за степенями. Отримаємо, що $(JP)^2=JPJP=0$, оскільки $J$ діє як обертання на 90 градусів, що дає $PJP=0$. Таким чином, експоненціальний ряд для $u(x)$ закінчується після перших двох доданків:
\begin{equation*}
	u(x) = \left(1+z\left(\int\limits_c^x h(t)\,dt\right)JP\right)u(c).
\end{equation*}
Зокрема, $u(d)=(1+zJH)u(c)$ з $H=\int_c^d H(x)\,dx$, але це теж саме, що і зробити один крок рекурсії в у різницевому рівнянні
\begin{equation}\label{eq-Canon_sys_dif}
 	J(u_{n+1} - u_n) = -zH_nu_n,
\end{equation}
аналогічному \eqref{Canonical_sys}, якщо $H_n=H$.

З більш абстрактної точки зору, властивістю матриці переходу $T=I+zJH$ через сингулярний інтервал є її поліноміальна залежність від $z$, причому ступеня 1. Це випливає з \eqref{eq-Canon_sys_dif} тоді, як диференціальні рівняння зазвичай призводять до складніших функцій $z$.

Отже, канонічна система через сингулярний інтервал імітує різницеве рівняння. Здатність канонічної системи робити це має вирішальне значення. Вже було згадано результат, що будь-які спектральні дані можуть бути реалізовані канонічною системою, і ці спектральні дані можуть виходити з різницевого рівняння, тому є необхідність варіанту їх моделювання.

Сингулярні інтервали також використовуються для реалізації граничних умов, і вони відповідають за багатозначну частину відношень, які асоціюються з канонічними системами.

\subsection{Гільбертів простір $L^2$}

У наступному розділі буде розглянуто самоспряжені реалізації \eqref{Canonical_sys} та їх спектральну теорію і з'явиться необхідність у гільбертовому просторі, в якому вони діють, і відповідним простором для цього є $L^2_H(a,b)$, визначений наступним чином. Припустимо, що $H(x)$ задовольняє умови для матриці коефіцієнтів канонічної системи. Нехай
\begin{equation*}
 	\calL = \left\{f:(a,b)\to \bbC^2: f \text{ є вимірною }, \int\limits_a^b f^*(x)H(x)f(x)\,dx <\infty \right\},
\end{equation*} 
тоді
\begin{equation}
 	||f|| = \left(\int\limits_a^b f^*Hf\,dx\right)^{1/2}, \quad f\in\calL,
\end{equation}
і $L_H^2(a,b)$ визначається як $\calL/\calN$, де $\calN=\{f\in\calL: ||f||=0\}$.

Це є звичайною процедурою визначення просторів $L^p$, за винятком, того, що функції приймають значення в $\bbC^2$, а не в $\bbC$. $L^2_H$ --- сепарабельний, нескінченновимірний простір Гільберта. Насправді, відображення
\begin{equation*}
	V:L_H^2(a,b)\to L^2_I(a,b), \quad (Vf)(x) = H^{1/2}(x)f(x),
\end{equation*}
де $H^{1/2}(x)$ визначено як єдиний додатний квадратний корінь з $H(x)$, що забезпечує вкладання $L_H^2$ в $L_I^2$, де $I$ --- одинична матриця $2\times2$. А, оскільки $L_I^2(a,b)\cong L^2(a,b)\oplus L^2(a,b)$, то всі питання можна звести до класичного простору $L^2$.

Оскільки $f^*Hf=(H^{1/2}f)^*H^{1/2}f$, то функції $f,g$ будуть являти собою один і той самий елемент в $L^2_H$ тоді і тільки тоді, коли $H(x)f(x) = H(x)g(x)$ майже для всіх $x$, але якщо $H(x)$ має ядро для деякого $x$ таке, що $H(x)v(x)=0$, $v(x)\ne0$, то це просто означає, що $f(x)-g(x)=c(x)v(x)$ для цього $x$. Зокрема, цілком можливо, що $f\in L^2_H$ має декілька неперервних представників, не рівних як функції.

Корисним є наступний факт.
\begin{lemma}
	Якщо $f\in L^2_H(a,b)$, то $Hf\in L^1_{loc}(a,b)$.
\end{lemma}
\begin{proof}
	Це випливає з нерівності Гельдера, якщо взяти додатний квадратний корінь $H^{1/2}(x)$ і записати $Hf=H^{1/2}(H^{1/2}f)$. Тоді обидва $H^{1/2}$ і $H^{1/2}f$ належать $L^2_{loc}(a,b)$ і їх добуток $Hf\in L^1_{loc}(a,b)$.
\end{proof}

\subsection{Мінімальні і максимальні відношення канонічних систем}

В цьому розділі буде розглянуто питання як канонічна система
\begin{equation*}
	Ju'(x) = -zH(x)u(x),\quad x\in(a,b)
\end{equation*}
генерує самоспряжені оператори у гільбертовому просторі $L^2_H(a,b)$. Ці оператори мають діяти як $-H^{-1}Jf'$ на функції $f$ з їх областей визначення і необхідно, щоб $H(x)$ мала обернену. Цієї проблеми можна уникнути, перемістивши $H$ назад. Припустимо, є пара $f,g\in L^2_H$ і $g=\tau f$ як результат застосування оператора над $f$, який необхідно побудувати. Формально це можна записати як
\begin{equation}\label{eq-relation-1}
	Jf'(x) = -H(x)g(x).
\end{equation}
Загалом ця умова визначатиме лінійні відношення, а не оператор.

Лінійні оператори $T$ стають {особливими} відношеннями після ототожнення їх зі своїми графіками $\{(x,Tx)\}$. І навпаки, відношення можна вважати операторами, за винятком того, що $f\in\calH$ може мати кілька зображень. Відношення $\calT$ є оператором, якщо $(f,g_1), (f,g_2)\in \calT$ означає, що $g_1=g_2$. За лінійністю це еквівалентно умові, що $(0,g)\in\calT$ тоді і тільки тоді, коли $g=0$.

Визначимо максимальне відношення $\calT$ канонічної системи як сукупність усіх пар $(f,g)$, для яких виконується \eqref{eq-relation-1}:
\begin{multline} \label{eq-relation-2}
	\calT=\{(f,g): f,g\in L^2_H(a,b), f \text{ має представника } f_0\in AC \text{ такого, що }\\ Jf'_0(x)=-H(x)g(x) \text{ майже для всіх } x\in(a,b)\}.
\end{multline}
Останнє чітко визначає лінійний підпростір, або відношення.

Як вже було показано, $f\in L^2_H$ може мати декілька неперервних представників, тому не можна реально очікувати, що $f_0$ однозначно визначається $f$. Тому особлива увага буде приділятися розрізненню елементів простору Гільберта (класів еквівалентності функцій) та функцій.

Хоча $f_0$ з \eqref{eq-relation-2} не має визначатися функцією $f$, але $f_0$ визначається парою $(f,g)$, якщо не розглядається тривіальний сценарій, де $(a,b)$ є лише одиничним сингулярним інтервалом. Отже, необхідне припущення: інтервал $(a,b)$ містить щонайменше одну регулярну точку. Це діятиме відтепер, якщо прямо не зазначено інше. Якщо $(a,b)$ --- єдиний сингулярний інтервал, то все можна опрацювати явно. Більше того, багато результатів потребують модифікації в цьому випадку, тому набагато зручніше просто виключати цей тривіальний сценарій з розвитку загальної теорії.

Повернемося до твердження, що якщо $(f,g)\in\calT$, то $f_0$ з \eqref{eq-relation-2} однозначно визначається. Дійсно, це показує інтегрування $Jf'_0=-Hg$: 
\begin{equation}
	f_0(x)=f_0(c) + J\int\limits_c^x H(t)g(t)\,dt.
\end{equation}
Слід зауважити, що тут $H(t)g(t)$ може змінюватися лише на нульовому наборі, якщо обрати іншого представника $g$, тому інтеграл визначається елементом гільбертового простору $g\in L^2_H$. Звідси випливає, що два таких представники $f$ могли б відрізнятись лише постійною функцією $v$, але тоді матимемо $H(x)v=0$ майже для кожного $x$, інакше вони б не представляли б один і той самий елемент гільбертового простору. Якщо ж $v\ne0$, тоді це означає, що $(a,b)$ --- сингулярний інтервал, який був виключений з розгляду.

\begin{lemma} \label{lemma-1-relations}
	Елемент $(f,g)\in\calT$ максимального відношення однозначно визначає абсолютно неперервну функцію $f_0:(a,b)\to\bbC^2$ з наступними двома додатковими властивостями: 
	\begin{enumerate}
		\item $f_0\in L^2_H(a,b)$, і він є представником елемента $f$;
		\item $Jf'_0=-Hg$.
	\end{enumerate}
\end{lemma}

Індекс у $f_0$ означає, що він є представником $f$, який визначається не лише самим $f$, але й парою $(f,g)\in\calT$.

Приклад, коли $f_0$ дійсно не визначається просто $f$, можна побачити, розглянувши випадок, де $(a,b)$ --- сингулярний інтервал. Покладемо $a=0$, $b = 1$, 
$H(x) =
\begin{pmatrix}
	1 & 0 \\ 0 & 0
\end{pmatrix}
$.

Тоді $(f,g)\in\calT$ тоді і тільки тоді, коли $f'_{0,2}=g_1$, $f'_{0,1}=0$. З точки зору простору Гільберта, важливий лише перший компонент функції. Таким чином, ми можемо прийняти будь-яку абсолютно неперервну функцію як $f_{0,2}$, і з цього випливає, що $g\in L^2_H(0,1)$ є довільною. Зрозуміло, що $f_{0,1}$ має бути постійною. Отже, позначаючи $e_1$ перший одиничний вектор, знаходиться наступне
\begin{equation}
	\calT = L(e_1)\oplus L^2_H(0,1) =\{(f,g): f(x) = ce_1, g\in L^2_H(0,1)\}. 
\end{equation}

Далі наведені необхідні визначення для відношень.

\begin{definition}
	Нехай $\calT\subseteq\calH\times\calH$ є відношенням. $\calT$ називається замкненим, якщо $\calT$ є замкненим підпростором $\calH\times\calH$. Замиканням $\overline{\calT}$ відношення $\calT$ називається замикання підпростору $\calT$.

	Області визначення та значень, ядро і багатозначна частина відношення $\calT$ визначаються наступним чином:
	\begin{align*}
	 	\dom \calT &= \{f\in\calH:(f,g)\in\calT \text{ для деякого } g\} &
	 	\ker \calT &= \{f\in\calH:(f,0)\in\calT \}\\
	 	\ran \calT &= \{g\in\calH:(f,g)\in\calT \text{ для деякого } f\} &
	 	\mul \calT &= \{g\in\calH:(0,g)\in\calT \}.
	\end{align*}

	Оберненим до $\calT$ є відношення
	\begin{equation*}
		\calT^{-1} = \{(g,f):(f,g)\in\calT\},
	\end{equation*}
	а спряжене відношення до $\calT$ визначається як
	\begin{equation*}
		\calT^* = \{(h,k):\left<h,g\right>=\left<k,f\right> \text{ для всіх } (f,g)\in\calT\}.
	\end{equation*}
	Відношення $\calT$ називається симетричним $\calT\subseteq \calT^*$, якщо, і самоспряженим, якщо $\calT=\calT^*$.
\end{definition}

Тож, на відміну від операторів, відношення завжди мають замикання, обернені до них та унікальні спряження.

Для того, щоб знайти спряження $\calT_0:=\calT^*$ максимального відношення $\calT$ канонічної системи, необхідно визначити предмінімальне відношення:
\begin{equation*}
	\calT_{00} = \{(f,g)\in\calT:f_0(x) \text{ має компактний носій на } (a,b) \}.
\end{equation*}

\begin{definition}
	Замикання $\calT_0=\overline{\calT_{00}}$ лінійного відношення $\calT_{00}$ називають мінімальним.
\end{definition}

Неважко помітити, що $\calT_{00}\subseteq\calT^*$. Дійсно, для фіксованого $(f,k)\in\calT_{00}$ і довільного $(h,k)\in\calT$ покажемо, що $\left<f,k\right>=\left<g,h\right>$, або
\begin{equation*}
	\int\limits_a^b f^*(x)H(x)k(x)\,dx = \int\limits_a^b g^*(x)H(x)h(x)\,dx.
\end{equation*}
Підключаючи до цього $Hk=-Jh'_0$, $Hg=-Jf'_0$, отримаємо
\begin{equation*}
	\int\limits_a^b \left(f^*_0(x)Jh'_0(x)+f^{*}{'}_0(x)Jh_0(x)\right)\,dx = 0,
\end{equation*}
що є очевидним, оскільки $f_0$ є нулем близько до $a$ і $b$.

\begin{proposition} \label{prop-1-relations}
	$\calT^*_{00}\subseteq\calT$
\end{proposition}
\begin{proof}
	Нехай $(f,g)\in\calT^*_{00}$ і функція $f_1$ визначається як
	\begin{equation*}
		f_1(x) = J\int\limits_c^x H(t)g(t)\,dt,
	\end{equation*}
	для деякого фіксованого $c\in(a,b)$. Тоді $f_1$ є абсолютно неперервною і $Jf'_1=-Hg$, але її квадрат не обов'язково є інтегрованим, тобто $f_1$ може не бути елементом гільбертового простору.

	Нехай $(h,k)\in\calT_{00}$ є довільним. Інтегрування за частинами показує, що
	\begin{equation*}
		\left<h,g\right> = \int\limits_a^b h^*_0(x)H(x)g(x)\,dx = \int\limits_a^b h^*_0(x)Jf'_1(x)\,dx = \int\limits_a^b k^*(x)H(x)f_1(x)\,dx.
	\end{equation*}
	Функції $h_0$ і $Hk$ мають компактний носій, тому факт, що $f_1$ може не лежати в $L^2_H$, не може зробить останній інтеграл розбіжним. З цієї ж причини інтегрування частинами не вносить граничні умови. 

	З іншого боку, $\left<h,g\right>=\left<k,f\right>=\int_a^b k^*Hf$, тому
	\begin{equation} \label{eq-1-prop1}
		\int\limits_a^b k^*(x)H(x)(f_1(x)-f(x))\,dx = 0 \quad \forall k\in \ran(\calT_{00}).
	\end{equation}
	Потрібно звернути увагу, що $k\in L^2_H(a,b)$ буде точно в $\ran \calT_{00}$, якщо він задовольняє наступним двом умовам: (1) $Hk$ має компактний носій; (2) $\int_a^b HK = 0$. Оскільки $Hk$ локально інтегрується та має компактний носій, останній інтеграл є визначеним. Позначимо через $X$ лінійний підпростір $L^2_H$, визначений умовою (1), і розглянемо на $X$ функціонали
	\begin{equation*}
		F_j(k) = e^*_j \int\limits_a^b H(x)k(x)\,dx, \quad F(k) = \int\limits_a^b (f_1(x)-f(x))^*H(x)k(x)\,dx.
	\end{equation*}
	Тепер \ref{eq-1-prop1} можна перефразувати як твердження, що якщо $F_1(k)=F_2(k)=0$ для $k\in X$, то $F(k)=0$. Тоді $F$ має бути лінійною комбінацією $F_1$ і $F_2$. Отже, існує вектор $v\in\bbC$ такий, що 
	\begin{equation*}
		\int\limits_a^b (f_1(x)-f(x)-v)^*H(x)K(x)\,dx = 0
	\end{equation*}
	для всіх $k\in X$. Оскільки $f_1(x)-f(x)-v$ локально в $L^2_H$, це можливо лише в тому випадку, коли $H(x)(f_1(x)-f(x)-v)=0$ майже скрізь. Отже, $f$ має абсолютно неперервного представника $f_1(x)-v$, а $J(f_1-v)'=-Hg$ за побудовою $f_1$. Це говорить про те, що $(f,g)\in\calT$, що доводить пропозицію.
\end{proof}

\begin{proposition} \label{prop-2-relations}
	Нехай $\calT\subseteq\calH\times\calH$ є відношенням. Тоді:
	\begin{enumerate}
		\item $\calT^*$ є замкненим;
		\item $\calT^{**}=\overline{\calT}$;
		\item $\overline{\calT}^* = \calT^*$
	\end{enumerate}
\end{proposition}

\begin{theorem}
	\begin{enumerate}
		\
		\item Максимальне відношення $\calT$ є замкненим;
		\item Мінімальне відношення $\calT^*_0=\calT$ є замкненим і симетричним, і $\calT^*_0=\calT$.
	\end{enumerate}
\end{theorem}
\begin{proof}
	1. Припустимо, що $(f_n,g_n)\in\calT$, $(f_n,g_n)\to (f,g)\in\calH\oplus\calH$. Необхідно довести, що $(f,g)\in\calT$.

	Переходячи до підпослідовності, можна припустити, що $H(x)f_{n,0}(x)\to H(x)f(x)$ поточково майже скрізь для представників із Леми~\ref{lemma-1-relations}. Також для кожного фіксованого $x\in(a,b)$ послідовність $f_{n,0}(x)$ повинна бути обмежена. Це випливає з того, що похідні $f'_{n,0}=JHg_n$ обмежені в $L^1(c,d)$ для будь-якої компактної підмножини $[c,d]\subseteq(a,b)$, тому, якщо б $|e\cdot f_{n,0}(x)|$ були великими для деякого напрямку $e\in\bbC^2$, $||e||=1$, то те саме було б справедливим для будь-якої компактної підмножини $(a,b)$, але це зробило б норму $f_n$ великою, оскільки $(a,b)$ не є a сингулярним інтервалом і, таким чином, $H(x)$ не може занулити $e$ скрізь.

	Тож можна обрати підпослідовність таку, що в додаток $f_{n,0}(c)\to v$ для фіксованого $c\in(a,b)$, що був обраний заздалегідь. Тепер можна просто перейти до поточкової границі в 
	\begin{equation*}
		f_{n,0}(x) = f_{n,0}(c) + J\int\limits_c^x H(t)g_n(t)\,dt.
	\end{equation*}
	Видно, що $f_{n,0}(x)$ сам збігається (не тільки після застосування $H(x)$), і його границя буде представляти f. Отже, був знайдений абсолютно неперервний представник $f$, який задовольняє $Jf'=-Hg$, отже $(f,g)\in\calT$.

	2. Зрозуміло, що спряжений оператор $\calT_0$ є замкненим, а симетрія випливатиме з двох заявлених рівностей, тому їх достатньо довести. Раніше було показано, що $\calT_{00}\subseteq\calT^*$ і $\calT^*_{00}\subseteq\calT$ (Пропозиція~\ref{prop-1-relations}), а спряження другого включення дає це $\calT^{**}_{00}\supseteq\calT^*$. А якщо взяти замикання першого включення і використати Пропозицію~\ref{prop-2-relations}, то отримаємо $\overline{\calT_{00}} = \calT^* = \calT_0$. Ще одне спряження дає останню рівність.
\end{proof}

Можна дати точніший опис мінімального відношення $\calT_0$. Принаймні, як частина результату: $\calT_0$ можна отримати, взявши замикання $\calT_{00}$, яке було визначене як ті елементи максимального відношення, для яких $f_0$ має компактний носій.

\begin{definition}\label{def-reg-endpoint}
	Кінцеву точку $a$ називають регулярною, якщо $H\in L^1(a,c)$ для деяких (і тоді всіх) $c\in(a,b)$, і аналогічно для $b$.

	Тут точки $a=-\infty$ і $b=\infty$ можуть бути звичайними кінцевими точками.
\end{definition}

Визначення~\ref{def-reg-endpoint} дає Лему~\ref{lemma2-relations}.

\begin{lemma} \label{lemma2-relations}
	Якщо $a$ є регулярною, то для будь-яких $(f,g)\in\calT$ представлення $f_0$ має неперервне продовження на $[a,b)$, і $f_0\in AC[a,b)$. Більше того, розв'язки однорідного рівняння $Ju'=-zHu$ мають ті ж самі властивості.

	Для регулярної точки $b$ результати аналогічні.
\end{lemma}

Вже відомо, що $f_0\in AC(a,b)$ і це означає, що $f_0(x)=f_0(c) + \int_c^x h(t)\,dt$ для деякої $h\in L^1_{loc}(a,b)$. Твердження, що $f_0\in AC[a,b)$, створює додаткове твердження, що $h\in L^1(a,c)$ для $c\in(a,b)$. Це означає, що $f_0$ має неперервне продовження до $x=a$, але не випливає з цієї властивості.

Ніякі зміни цих тверджень не потрібні у випадку $a=-\infty$, якщо дати розширеному інтервалу $[a,c)=[-\infty,c)$ його очевидну топологію.

\begin{proof}
	Нерівність Коші-Шварца показує, що для будь-якого $g\in L^2_H(a,c)$, маємо, що $Hg=H^{1/2}H^{1/2}g\in L^1(a,c)$, тож твердження для $f_0$ випливають з
	\begin{equation*}
	 	f_0(x) = f_0(c) + J\int\limits_c^x H(t)g(t)\,dt.
	 \end{equation*}
	Як і для розв'язків $u$ рівняння $Ju'=-zHu$, застосуємо теорію звичайних диференціальних рівнянь, узагальнену в Теоремі~\ref{Th-CS-1}, до початкової задачі значення $u(a)=v$ для загального $v\in\bbC^2$, щоб підтвердити, що $u$ є абсолютно неперервними на $[a,b)$.

	Початкова задача $Ju'=-zHu$, $u(-\infty)=v$ може бути записана як інтегральне рівняння $u(x)=v+zJ\int_{-\infty}^x H(t)u(t)\,dt$, і, якщо $H\in L^1(-\infty,c)$, то це можна розв'язати так само, як на обмеженому проміжку, за допомогою ітерації Пікарда. Або, можна провести перетворення, щоб зробити $a$ кінцевою точкою $A\in\bbR$, щоб взагалі уникнути цих питань.
\end{proof}

\begin{lemma}\label{lemma-7-relations}
	Нехай $(c,d)\subseteq(a,b)$, і жоден з $(a,c)$, $(d,b)$ не є порожнім інтервалом, що міститься в одному сингулярному інтервалі. Нехай $(h,k)\in\calT_{(c,d)}$. Тоді існує $(f,g)\in\calT$ з $f_0=h_0$ на $(c,d)$, $f_0(x)=0$ для $x\in(a,c)$ близько до $a$ і $x\in(d,b)$ близько до $b$.
\end{lemma}
\begin{proof}
	Нехай $d<b$. Оскільки $d$ є регулярною кінцевою точкою $(c,d)$, застосовується Лема~\ref{lemma2-relations}, тоді $h_0$ абсолютно неперервна на $(c,d]$. Потрібно знайти абсолютно неперервну функцію $f_0$ на $[d,b)$ таку, що $Jf'_0=-Hg$ для деяких $g\in L^2_H(d,b)$ і $f_0(d)=h_0(d)$, щоб зробити функцію абсолютно неперервною при з'єднанні двох частин. Також бажано, щоб $f_0(x)=0$ для всіх великих $x$, і достатньо одного разу досягти цього значення, адже з цього моменту можна застосувати нульову функцію. Отже, щоб $f_0(d)=h_0(d)$ і $f_0(t)=0$, оберемо $t\in(d,b)$ настільки великим, щоб $(d,t)$ не міститься в сингулярному інтервалі. Якщо все це сказати для $g$, то тепер потрібно знайти $g\in L^2_H(d,t)$ таке, що функція $f_0$ визначається як
	\begin{equation*}
		f_0 = h_0(d) +J\int\limits_d^x H(s)g(s)\,ds
	\end{equation*}
	і задовольняє умові $f_0(t)=0$. Це працює, якщо лінійне відображення
	\begin{equation*}
		F:L^2_H(d,t)\to\bbC^2, \quad F(g) = \int\limits_d^t H(s)g(s)\,ds
	\end{equation*}
	є сюр'єктивним і легко зрозуміти, що це буде в тому випадку, якщо $(d,t)$ не міститься в сингулярному інтервалі, тому що діапазон $H(x)$ не може бути однаково рівним фіксованому одновимірному підпростору $\bbC^2$.

	Нарешті, якщо також $c>a$, то застосовуємо ту саму процедуру зліва від $(c,d)$.
\end{proof}

\begin{theorem}\label{th-relation-1}
	Нехай $(f,g), (h,k)\in \calT$. Тоді для $f^*_0(x)Jh_0(x)$ існують границі при $x\to a+$ і $x\to b-$. Більше того,
	\begin{equation}\label{eq-relation-3}
	 	\left<g,h\right> - \left<f,k\right> =\Bigl. f^*_0Jh_0\,\Bigl|_a^b.
	\end{equation}
\end{theorem}

Для цих меж будуть використані позначення $(f^*_0Jh_0)(a)$ і $(f^*_0Jh_0)(b)$. Якщо кінцева точка (скажімо, $a$) є регулярною, то існування їх стає безпосереднім наслідком Леми~\ref{lemma2-relations}, і в цьому випадку $(f^*_0Jh_0)(a) = f^*_0(a)Jh_0(a)$. Тут використовуються наводить позначення $f_0(a)$, $h_0(a)$ для неперервних продовжень цих функцій до $x=a$.

\begin{proof}
	Обидва твердження випливають із наступного розрахунку:
	\begin{align*}
		\left<g,h\right> - \left<f,k\right> &= \lim_{\substack{\alpha\to a+\\ \beta\to b-}} \int\limits_\alpha^\beta (g^*(x)H(x)h_0(x) - f^*_0(x)H(x)k(x))\,dx \\
		&= \lim_{\substack{\alpha\to a+\\ \beta\to b-}} \int\limits_\alpha^\beta (f_0^{*}{'}(x)Jh_0(x) - f^*_0(x)Jh'_0(x))\,dx \\
		&= \lim_{\substack{\alpha\to a+\\ \beta\to b-}} \Bigl. f^*_0Jh_0\Bigr|_\alpha^\beta.
	\end{align*}
\end{proof}